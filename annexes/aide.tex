\chapter{Trouver de l'aide}

\begin{prealable}
Comment faire lorsque l'on est coincé sur un point particulier, qu'on a relu les différents manuels plusieurs fois ? Demander de l'aide à d'autres utilisateurs de \LaTeX{}.

Voici quelques endroits pour cela. 
\end{prealable}

\section{Forums internet}

Les forums internet sur \LaTeX sont pléthores. En anglais, on pourra utiliser celui de \forme{LaTeX Community} : \url{http://www.latex-community.org/forum/}. En français, on peut utiliser celui du site \forme{Devellopez.net} \url{http://www.developpez.net/forums/f149/autres-langages/autres-langages/latex/}.


\section{Messagerie instantanée}

Il est possible de demander de l'aide sur différent salons de messagerie instantanée fonctionnant par IRC\footnote{Lointain ancêtre de Skype, MSN et autres GoogleTalk}. En général on peut y trouver de l'aide assez rapidement.

Pour se connecter à un salon de discussion IRC, on peut utiliser le plugin Chatzilla du logiciel libre Firefox.

Voici l'adresse de deux salons consacrés à \LaTeX.

\begin{description}
\item[En français]\url{irc://irc.rezosup.org/latex}
\item[En anglais]\url{irc://irc.freenode.net/latex}
\end{description}

\section{Liste de discussion}

En français, on peut utiliser celle de l'association Gutemberg : \url{http://www.gutenberg.eu.org/?Listes-de-diffusion-gerees-par}. En anglais, on utilisera la liste suivante : \url{http://groups.google.com/group/comp.text.tex/topics}.
