\chapter{Bibliographie}

\nocite{*}
\section{Manuels des packages}

Nous ne listons pas ici l'ensemble des manuels des divers package abordés. Toutefois une question peut se poser : où trouver ces manuels ?

La solution la plus simple est :
\begin{itemize}
\item Sous Mac d'ouvrir l'application \forme{Terminal} dans le dossier \forme{Utilitaires}.
\item Sous Windows d'ouvre \forme{l'invite des commandes} dans \forme{Démarrer} : \forme{Tous les programmes} : \forme{Accessoires}
\item Sous Linux -> Brendan ?
\end{itemize}

Puis de taper dans la fenêtre qui apparaît :

\begin{minted}{bash}
texdoc nomdupackage
\end{minted}

Par exemple pour le package \package{biblatex}

\begin{minted}{bash}
texdoc biblatex
\end{minted}

Taper un retour à la ligne : le manuel du package devrait s'ouvrir avec le logiciel adéquat (généralement le manuel est au format PDF).

\section{Livres généralistes}

Les livres sur \LaTeX sont pléthores. Cependant la plupart n'abordent ni \XeLaTeX, ni \package{polyglossia}, ni \package{biblatex}. C'est pourquoi je conseille de les utiliser essentiellement pour les besoins le plus avancées de mise en page.

\printbibliography[keyword=generaliste]

\section{Livres et textes sur des points spécifiques}

\printbibliography[keyword=specifique]


\section{Sites internet}

Comme pour les livres généralistes,  l'interêt en terme de contenu, par rapport à la problématique de ce livre, peut être très variable. 

\printbibliography[keyword=site]
