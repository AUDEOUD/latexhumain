\section{Installer TeXLive sous Mac Os X}

La méthode la plus simple consiste à installer MacTeX\footnote{\url{http://www.tug.org/mactex/}}. Celle-ci comprend non seulement une version de TeXLive pour Mac, mais aussi des logiciels pour faciliter la rédaction avec \LaTeX{}.

Pour installer, il suffit de télécharger le fichier d'installation sur le site, en page d'accueil. Il s'agit d'un fichier \ext{zip}, qu'il faut le cas échéant décompresser. Dedans se situe un fichier \ext{mpkg} : un double clique dessus suffit à lancer le logiciel d'installation. Il vaut mieux conserver les réglages standards d'installation.

Une fois l'installation effectuée, vous pouvez vous rendre dans le dossier \forme{TeX} du dossier \forme{Applications} de votre Mac.

Ce dossier comprend plusieurs applications :
\begin{description}
\item[BibDesk]qui est un logiciel pour gérer les fichiers bibliographiques au format BibTex.\renvoi{bibliofichier}
\item[Excalibur]qui est un logiciel de correction orthographique prévu pour reconnaître les commandes \LaTeX{}. Nous ignorons sa qualité, ne l'ayant pas essayé.
\item[LaTeXit]qui est un logiciel pour rédiger encore plus facilement des équations en \LaTeX{} et les exporter vers divers logiciels. Comme ce livre est l'un des rare sur \LaTeX{} à ne pas expliquer comment faire des équations avec \LaTeX{}, et bien, nous n'en parlerons pas.
\item[TeXLive Utility]qui sert à mettre à jour les différents modules de \LaTeX{} : classe, packages etc. Nous en parlons plus loin.
\item[TeXworks et TeXShop]qui sont deux éditeurs de textes prévus pour \LaTeX. Choissisez l'un d'entre eux pour commencer avec \LaTeX{}.\renvoi{commencer}
\end{description}
