\section{Introduction à la ligne de commande}\label{lignedecommande}

\begin{preliminaire}
La plupart des éditeurs de textes spécialisés en LaTeX  proposent un bouton pour mettre en œuvre les principales commandes de compilation :  XeLaTeX, MakeIndex, BibTeX.

Toutefois très rapidement il devient nécessaire de pouvoir faire plus : par exemple si l’on utilise le package \package{splitindex}, il faut pouvoir  executer la commande \verb|splitindex|.

Pour ce faire on utilise le terminal du système d'exploitation, qui permet d'executer directement ces commandes. Voici une breve introduction à son utilisation dans le cadre d'un usage de \XeLaTeX.
\end{preliminaire}

\subsection{La notion de repertoire courant}

Ce qu'on appelle \forme{repertoire courant} correspond à l'emplacement où l'on se situe dans l'arborescence des fichiers de l'ordinateur. Quand on veut utiliser les lignes de commande pour se servir de \XeLaTeX, la première chose à faire est de changer le repertoire courant pour se rendre dans le dossier dans lequel se situe les fichiers à compiler.
\subsection{Mac OS X et Linux}

Sur Mac OS X, le terminal se situe dans le dossier \forme{Utilitaires} du dossier \forme{Applications}. Sur Linux -> compléter.

Le repertoire courant est généralement indiqué à gauche de la ligne, le symbole  \~  représentant le dossier de départ. Pour lister son contenu, frapper \verb|ls|. Pour valider une commande, il faut frapper sur la touche \verb|Entrée|.

Pour vous déplacer dans un répertoire, il suffit de taper la commande
\verb|cd| suivie du dossier où vous souhaitez vous rendre.

Ainsi la commande \verb|cd projet-latex| vous fera pénétrer dans le
répertoire \verb|projet-latex|, ce que vous pouvez vérifier avec la
commande \verb|dir|. Pour vous déplacer dans un répertoire parent, tapez
simplement \verb|cd ..|.

Lorsque vous êtes dans le répertoire où vous souhaitez exécuter une
commande, vous pouvez la lancer de la même manière :

\begin{quote}
\verb|xelatex nomdufichieràcompiler.tex|
\end{quote}

\subsection{Windows}
Sous Windows, le terminal s'appelle \enquote{Invite de commandes}. Le langage
est différent de celui que l'on trouve chez les systèmes de type Unix comme
Linux et Mac OS X. On trouvera cependant quelques similarités.

Pour démarrer une invite de commande, pressez simultanément la touche \verb|Windows| de votre
clavier et la touche \verb|R|. Dans l'invite qui s'ouvre tapez \verb|cmd| puis
\verb|Entrée|. Vous voilà face à une console.

Toute commande, une fois tapée, est validée par une pression sur la touche \verb|Entrée|.

Le répertoire courant est indiqué à gauche du curseur clignotant.
La commande \verb|dir| vous indique ce qui se trouve dans le répertoire
courant.

Pour vous déplacer dans un répertoire, il suffit de taper la commande
\verb|cd| suivie du dossier où vous souhaitez vous rendre.

Ainsi la commande \verb|cd projet-latex| vous fera pénétrer dans le
répertoire \verb|projet-latex|, ce que vous pouvez vérifier avec la
commande \verb|dir|. Pour vous déplacer dans un répertoire parent, tapez
simplement \verb|cd ..|.

Lorsque vous êtes dans le répertoire où vous souhaitez exécuter une
commande, vous pouvez la lancer de la même manière :

\begin{quote}
\verb|xelatex nomdufichieràcompiler.tex|
\end{quote}

