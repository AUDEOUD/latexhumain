\chapter{Personnaliser en-têtes et pieds de pages}\label{entete}

\begin{intro}
Nous allons voir comment personnaliser les en-têtes et pieds de pages à l'aide du package \package{fancyhdr}.
\end{intro}


\section{Utiliser l'un des styles standards}\label{stylesentete}

\LaTeX propose quatre styles standards de page, qui définissent le contenu des en-têtes et pieds de pages. Pour utiliser l'un de ces styles, on emploie  la commande \csp{pagestyle}\marg{style}.


On peut également modifier le style d'une page en particulier grâce à la commande \csp{thispagestyle}\marg{style}.

Ces styles sont :
\begin{description}
\item[empty] pas d'en-tête ni de pied de page.
\item[plain] pas d'en-tête mais pieds de pages contenant le numéro de page en centré. C'est le style correspondant aux pages de début de chapitre\renvoi{chapitrepagestyle}. 
\item[heading] pas de pied, en-têtes contenant le titre du chapitre, de la section ou de la sous-section et le numéro de la page. C'est le style par défaut. \label{styleentetes}
\item[myheading] similaire au précédent, mais l'en-tête peut être personnalisé\footcite[Nous tenons cette information du manuel de][\pno~222, qui ne précise pas comment faire]{latex_companion}.
\end{description}



On voit vite les limites de ces styles. Ainsi, comment  avoir à la fois les titres en en-tête et les numéros des pages en pied de page ? Comment indiquer son nom ou la date en pied de page ?

\section{Premiers exemples avec \packagenoidx{fancyhdr}}\index[pkg]{fancyhdr}

Le package \package{fancyhdr} propose un autre style, qu'il est aisé de personnaliser \emph{via} des commandes spécifiques. Pour faire fonctionner \package{fancyhdr} il faut écrire les lignes suivantes dans le préambule :

\begin{latexcode}
\usepackage{fancyhdr}
\pagestyle{fancy}
\end{latexcode}

\begin{plusloins}
Les pages de début de chapitre ont automatiquement le style \forme{plain}. Pour désactiver ce style par défaut, il faut modifier le commande \cs{chapter}. Nous en parlons dans un autre chapitre\renvoi{entetechapter}. 
\end{plusloins}


On peut ensuite utiliser les six commandes de \package{fancyhdr} qui servent à définir le contenu des en-têtes et pieds de pages :

\begin{description}
\item[\csp{lhead}] reçoit comme argument le contenu de la partie gauche de l'en-tête, justifiée à gauche.
\item[\csp{chead}] reçoit comme argument la partie centrale de l'en-tête, centrée.
\item[\csp{rhead}] reçoit comme argument la partie droite de l'en-tête, justifiée à droite.
\item[\csp{lfoot}] reçoit comme argument la partie gauche du pied de page, justifiée à gauche.
\item[\csp{cfoot}] reçoit comme argument la partie centrale du pied de page, centrée.
\item[\csp{rfoot}] reçoit comme argument la partie droite du pied de page, justifiée à droite.
\end{description}



Supposons que nous souhaitions afficher le numéro des pages en pied de page centré, en indiquant également le nombre total de pages. Nous allons utiliser le package \package{totpages} qui nous permet, après deux compilations, ou plus si le nombre de pages varie entre les compilations, d'obtenir le nombre total de pages.

\begin{latexcode}
\usepackage{fancyhdr}
\pagestyle{fancy}
\usepackage{totpages}
\cfoot{{\thepage} / \ref{TotPages}}
\end{latexcode}



La commande \csp{thepage} indique la valeur du compteur \compteur{page}\renvoi{compteur}, correspondant au numéro de page.

Vous pouvez constater le résultat sur cette page.
\cfoot{{\thepage} / \ref{TotPages}\cfoot{\thepage}} 

\section{Pages recto verso et alternance gauche-droite}

Lorsqu'un travail est imprimé en recto verso, on peut souhaiter que l'en-tête et le pied de page gauches des pages impaires correspondent à l'en-tête et au pied de page droits des pages paires et \emph{vice-versa}.

Le package \package{fancyhdr} a prévu ce cas. Il propose deux commandes : 
\begin{itemize}
\item \csp{fancyhead}\oarg{position}\marg{texte d'en-tête}
\item \csp{fancyfoot}\oarg{position}\marg{texte de pied de page}
\end{itemize}

L'argument \arg{position} peut prendre une ou plusieurs des valeurs suivantes :
\begin{description}
\item[C] centre
\item[LO] gauche des pages impaires (= intérieur des pages de droite, si on écrit de gauche à droite).
\item[RO] droite des pages impaires (= extérieur des pages de droite, si on écrit de gauche à droite).
\item[LE] gauche des pages paires (= extérieur des pages  de gauche, si on écrit de gauche à droite).
\item[RE] droite des pages paires (= intérieur des pages de gauche, si on écrit de gauche à droite).
\end{description}

Ainsi pour mettre le numéro de page à l'extérieur du pied de page, il suffit d'écrire :

\begin{latexcode}
\fancyfoot[LE,RO]{\thepage}
\end{latexcode}

Évidemment, si l'on dit à \LaTeX de générer un fichier à destination d'une impression monoface\renvoi{rectoverso}, il considère qu'il n'y a que des pages recto, c'est-à-dire impaires.

\section{Titres dans l'en-tête : le mécanisme des marqueurs}

Dans la classe \classe{book}, les en-têtes contiennent par défaut les titres de chapitres sur la page de gauche et de sections sur la page de droite. En utilisant le style \verb|fancy|, on insère automatiquement les titres de chapitres du côté interne et ceux de sections dans le côté externe. 

Si on utilise \package{fancyhdr} pour ne personnaliser que le pied de page, on peut  vouloir rétablir pour le reste la présentation  par défaut. Il est nécessaire pour cela de comprendre le mécanisme des marqueurs de \LaTeX. 

Le principe de base est simple : des commandes de marquage vont stocker en mémoire des marqueurs. Lesdits marqueurs sont appelés par d'autres commandes. 

Les deux commandes de marquage sont :
\begin{itemize}
\item \csp{markboth}\marg{marqueur gauche}\marg{marqueur droit}
\item \csp{markright}\marg{marqueur droit}
\end{itemize}

Les deux commandes d'appel des marqueurs sont : 
\begin{itemize}
\item\csp{leftmark} qui retourne l'argument \arg{marqueur gauche} de la dernière commande \cs{markboth}.
\item\csp{rightmark} qui retourne l'argument \arg{marqueur droit} de la première commande \cs{markright} ou \cs{markboth}  située sur la page courante. En revanche, si la page en cours ne contient aucune de ces commandes, alors \csp{rightmark} retourne l'argument \arg{marqueur droit} du dernier \cs{markright} ou \cs{markboth} utilisé.
\end{itemize}

Concrètement, les commandes  \cs{markboth} et \cs{markright} sont appelées automatiquement dans la classe \classe{book} par les commandes \cs{chapter} et \cs{section} \emph{via} les commandes \csp{chaptermark} et \csp{sectionmark}. 

\subsection{Appeler les marqueurs dans les styles fancy}

Pour rétablir la présentation originelle, il nous faut insérer les commandes \cs{leftmark} et \cs{rightmark} dans nos commandes \cs{fancyhead}.

\begin{latexcode}
\fancyhead[LE,RO]{}
\fancyhead[RE]{\leftmark}
\fancyhead[LO]{\rightmark}
\end{latexcode}

Avec ceci, nous obtenons à nouveau le titre de chapitre à gauche et le titre de section à droite.

\begin{attention}
Les en-têtes et pieds de pages sont calculés par \LaTeX \emph{après} le reste de la page. C'est pourquoi on obtient dans l'en-tête le contenu de la dernière section de la page.
\end{attention}

\subsection[Redéfinir \oldcs{chaptermark} et \oldcs{sectionmark}]{Redéfinir les commandes \cs{chaptermark} et \cs{sectionmark}}

Supposons désormais que vous ne souhaitez plus voir les titres des sections dans les en-têtes. Le plus simple est alors de redéfinir la commande \cs{sectionmark}, pour la rendre nulle :

\begin{latexcode}
\renewcommand{\sectionmark}[1]{}
\end{latexcode}

\begin{attention}
Notez bien que nous indiquons que la commande \cs{sectionmark} prend un argument, alors même que nous ne nous en servons pas. Mais lorsque la classe \classe{book} appelle cette commande, elle lui passe bien un argument, qui est le titre de la section.
\end{attention}

Comme alors on affiche  plus que le titre du chapitre, il est peut-être inutile de préciser qu'il s'agit d'un chapitre. Il  faut alors redéfinir la commande \cs{chaptermark}, par exemple sous la forme suivante :

\begin{latexcode}
\renewcommand{\chaptermark}[1]{\markboth {\MakeUppercase{%
    \thechapter~#1}}{}}
\end{latexcode}


\begin{plusloins}
Si vous vous amusez à fouiller le fichier \fichier{book.cls}, vous verrez que la commande \cs{chaptermark} est définie deux fois. En réalité, sa définition est conditionnée par l'option de classe \option{twoside} ou \option{oneside}\renvoi{nbsides}.
\end{plusloins}

\begin{plusloins}
La commande \cs{thechapter} sert à afficher le numéro du chapitre, stocké dans un compteur \compteur{chapter}. Nous parlons plus loin des compteurs et des commandes \cs{the\meta{compteur}}\renvoi{apparencecompteur}.
\end{plusloins}

\section{Filet d'en-têtes et de pieds de pages}

Le package \package{fancyhdr} inclut un filet entre le corps du texte et l'en-tête. Il est possible de modifier l'épaisseur de ce filet, ainsi que de rajouter un filet avant le  pied de page.

On redéfinit pour cela les commandes \csp{headrulewidth} et \csp{footrulewidth}, qui doivent renvoyer une longueur\renvoi{unite}.
Par exemple, pour supprimer le filet au-dessous de l'en-tête mais en ajouter un de 0,05 mm au-dessus du pied de page, on écrit : 

\begin{latexcode}
\renewcommand{\headrulewidth}[0]{0pt}
\renewcommand{\footrulewidth}[0]{0.05mm}
\end{latexcode}
