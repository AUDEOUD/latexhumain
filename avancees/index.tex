\chapter{Formater son index}\label{styleindex}

\begin{intro}
La mise en forme par défaut de l'index n'est pas très satisfaisante. On peut heureusement la modifier pour rendre l'index plus lisible.
\end{intro}

\section{Comment faire}
Les commandes pour formater l'index sont à placer dans un fichier \fichier{monstyle.ist} que vous devez créer vous-même.

\begin{attention}
Pensez à placer ce fichier à un endroit où \LaTeX peut le trouver, par exemple à côté du fichier principal.
\end{attention}

Il faut ensuite compiler en ligne de commande suivant cette syntaxe :\renvoi{terminal}

\begin{bashcode}
makeindex -s monstyle.ist monfichier.idx
\end{bashcode}

Ou si vous utilisez \package{splitindex}\renvoi{splitindex}.

\begin{bashcode}
splitindex -m "makindex -s monstyle.ist" monfichier.idx
\end{bashcode}

L'option -s indique à \package{makeindex}  qu'il doit utiliser le fichier de style \fichier{monstyle.ist}.

\section{Quelques commandes}

Nous allons donner ici quelques exemples de commandes que l'on peut mettre dans son fichier \fichier{monstyle.ist}. 

\begin{attention}
Dans le fichier de style\ext{ist}, on n'écrit pas en \LaTeX{} mais en \TeX . Les commandes sont ainsi encadrées par des guillemets anglais \verb|"| . La barre contre-oblique, quant à elle, n'est prise en compte que si elle est précédée d'une autre contre-oblique.
\end{attention}

Le manuel de la commande MakeIndex --- que l'on peut lire en frappant dans le terminal \verb+man makeindex+\renvoi{terminal} et que l'on peut quitter en frappant la lettre  \verb|q|, ou, si l'on utilise Windows, que l'on peut consulter à cette adresse \url{http://www.fiveanddime.net/man-pages/makeindex.1.html}. --- nous indique que, pour modifier ce qui est inséré entre le premier niveau d'item (l'entrée) --- ou niveau 0 --- et le ou les numéros de page, on utilise l'option \verb|delim_0| ; de même, pour définir ce qui est inséré entre le deuxième niveau d'item (la sous-entrée) --- ou niveau 1, et les numéros, on utilise l'option \verb+delim_1+, et l'option \verb| delim_2| pour les sous-sous-entrées --- ou niveau 2 d'itemisation. Le choix par défaut est une virgule suivit d'un blanc : \verb|", "|

Ainsi, si nous désirons que le numéro de la page soit justifié à droite, et non simplement séparé de l'entrée par une virgule et un blanc, nous devons utiliser la commande \csp{hfill}\renvoi{hfill} que nous indiquons dans notre fichier de style par les lignes suivantes:

\begin{latexcode}
delim_0 "\\hfill "
delim_1 "\\hfill "
delim_2 "\\hfill "
\end{latexcode}

Nous obtenons donc: 

\begin{quotation}
\begin{tabbing}
\hspace{0,5cm}  \=  \hspace{3cm} \= \kill
Adrien\>\> 5 \\
\\
Charlemagne \>\>5 \\
\\
Ducs \\
\> Tassilon\> 5\\
\\
Evêques \\
\> Damase \> 5\\
\> Formose\>  5\\
\end{tabbing}
\end{quotation}

Si nous préférons une ligne de points, il faut logiquement remplacer \verb|"|\textbackslash\cs{hfill}\verb| "| par \verb|"|\textbackslash\cs{dotfill}\verb| "|

\begin{attention}
Les espaces situées entre les commandes et les deuxièmes guillemets fermants sont indispensables. En effet, en l'absence de ces espaces, MakeIndex accole, dans le fichier \ext{ind} généré, les commandes aux numéros de page, ce qui bloque la seconde compilation, puisque \LaTeX{} se retrouve alors face à des commandes inconnues, du style \csp{hfill1}.
\end{attention}

Si l'on continue la lecture du manuel, on apprend aussi comment faire apparaître \enquote{sq.}, ou ce que l'on veut, quand une même entrée est indexée dans trois pages ou plus à la suite. Il suffit d'indiquer \verb+suffix_3p "~sq."+

Voyons maintenant comment l'on peut insérer les lettres de l'alphabet entre les groupes d'entrées. Cette manipulation permet de rendre plus lisible un index un peu long. 

Le manuel indique que l'option par défaut \verb+headings_flag 0+ ne met pas de séparateurs entre les groupes, que l'option \verb+headings_flag 1+ permet d'obtenir des lettres majuscules comme séparateur, et l'option  \verb+headings_flag -1+ des lettres minuscules.

Définissons ensuite la manière dont vont apparaître ces lettres. On utilise la commande \verb|heading_prefix|. Supposons que nous voulions faire apparaître ces lettres séparatrices en gras. N'oublions pas que l'on code en \TeX … Nous écrivons donc :

\begin{latexcode}
heading_prefix "{\\bf "
\end{latexcode} 

Si nous voulons qu'elles apparaissent en italique, nous écrivons \verb|"|\textbackslash\cs{it}\verb|"| (on utilise \csp{bf} et \csp{it}, qui sont des commandes \TeX, et non \cs{textbf} \cs{textit}).

Nous  pouvons de même changer la taille des lettres séparatrices en ajoutant \textbackslash\cs{large} ou \textbackslash\cs{Large}, et ainsi de suite. \renvoi{taille}

Mais nous avons ouvert une accolade… il nous faut la refermer:

\begin{latexcode}
heading_suffix " }\\nopagebreak\n " 
\end{latexcode}

\textbackslash\csp{nopagebreak} évite qu'une lettre-séparatrice se retrouve seule en fin de page. Le \verb|\n| est une petite coquetterie pour permettre d'avoir, dans notre fichier \ext{ind}, un retour à la ligne après \csp{nopagebreak}.



Ce n'est pas fini. Nos lettres sont actuellement justifiées à gauche, comme les entrées de l'index. Nous voudrions qu'elles soient centrées: 

\begin{latexcode}
headings_flag 1
heading_prefix " {\\bf\\large\\hfill " 
heading_suffix " \\hfill}\\nopagebreak\n " 
\end{latexcode}

Les deux \textbackslash\csp{hfill} étirent, pour le premier, l'espace avant la lettre vers la droite, et pour le second l'espace suivant la lettre vers la gauche -- voilà nos lettres centrées.

\begin{quotation}
\begin{tabbing}
\hspace{0,5cm}  \= \hspace{1cm} \= \hspace{1,5cm} \= \kill
\>\> \large{\textbf{A}}\\
\\
Adrien\>\>\> 5 \\
\\
\>\> \large{\textbf{C}}\\
\\
Charlemagne \>\>\>5 \\
\\
\>\> \large{\textbf{D}}\\
\\
Ducs \\
\> Tassilon\>\> 5\\
\\
\>\> \large{\textbf{E}}\\
\\
Evêques \\
\> Damase \>\> 5\\
\> Formose\>\>  5\\

\end{tabbing}
\end{quotation}


Si nous préférons qu'elles soient simplement légèrement décalées vers la droite, nous pouvons utiliser, à la place de \textbackslash\cs{hfill}, \textbackslash\cs{hspace*}\verb|{1ex}| qui rajoute à gauche un espace de la taille d'un x. \renvoi{unite}\renvoi{espace}


Quand on a compris ce système, il est alors assez simple de formater à sa guise l'index. C'est pourquoi, pour les autres options, nous renvoyons ici à la lecture du manuel\footcite[On pourra aussi consulter][]{frama_index}. 

Voici tout de même, à titre indicatif, le fichier final :

\begin{latexcode}
headings_flag 1
suffix_3p "~sq."
heading_prefix "{\\bf\\large\\hspace*{1ex} " 
heading_suffix " }\\nopagebreak\n" 
delim_0 "\\dotfill "
delim_1 "\\dotfill "
delim_2 "\\dotfill "
\end{latexcode}




