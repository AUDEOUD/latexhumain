\subsection{Formater son index}

La mise en forme par défaut de l'index n'est pas très satisfaisante. On peut heureusement la modifier pour rendre l'index plus lisible.


\subsubsection{Comment faire}
Les commandes pour formater l'index sont à placer dans un fichier \fichier{monstyle.ist} que vous devez créer vous-même.

\begin{attention}
Pensez à placer ce fichier à un endroit où \LaTeX peut le trouver, par exemple dans le dossier courant. \renvoi{repcourant}
\end{attention}

Il faut ensuite taper dans le terminal :
\begin{minted}{bash}
makeindex -s monstyle.ist monfichier.idx
\end{minted}

L'option -s indique à \package{makeindex}  qu'il doit utiliser le fichier de style \fichier{monstyle.ist} .

\subsubsection{Quelques commandes}

Je vais donner ici quelques exemples de commandes que l'on peut mettre dans son fichier \fichier{monstyle.ist}. 

\begin{attention}
Dans le fichier de style\ext{ist}, on n'écrit pas en \LaTeX{} mais en \TeX . Les commandes sont ainsi encadrées par des guillemets anglais \verb|"| . La barre contre-oblique, quant à elle, n'est prise en compte que si elle est précédée d'une autre contre-oblique.
\end{attention}

Le manuel de \package{makeindex} que l'on peut lire en tapant dans le terminal \verb+man makeindex+ nous indique que, pour modifier ce qui est inséré entre le premier niveau d'item (l'entrée) - ou niveau 0 -  et le ou les numéros de pages, on utilise l'option \verb|delim_0| ; de même, pour définir ce qui est inséré entre le deuxième niveau d'item (la sous-entrée) - ou niveau 1, on utilise l'option \verb+delim_1+, et l'option \verb| delim_2| pour les sous-sous-entrées - ou niveau 2 d'itemisation. Le choix par défaut est une virgule suivit d'un blanc : \verb|", "|

Ainsi, si je veux que le numéro de la page soit justifié à droite, et non simplement séparé de l'entrée par une virgule et un blanc, je dois indiquer dans mon fichier de style les lignes suivantes:

\begin{minted}{latex}
delim_0 "\\hfill"
delim_1 "\\hfill"
delim_2 "\\hfill"
\end{minted}

J'obtiens donc : 

\begin{quotation}
\begin{tabbing}
\hspace{0,5cm}  \=  \hspace{3cm} \= \kill
Adrien\>\> 5 \\
\\
Charlemagne \>\>5 \\
\\
Ducs \\
\> Tassilon\> 5\\
\\
Evêques \\
\> Damase \> 5\\
\> Formose\>  5\\
\end{tabbing}
\end{quotation}

Si je préfère une ligne de points, il faudra logiquement remplacer \verb|"\\hfill"| par \verb+"\\dotfill"+


Si l'on continue la lecture du manuel, on apprend aussi comment faire apparaître \enquote{sq.} (ou ce que l'on voudra) quand une même entrée est indexée dans trois pages ou plus à la suite. Il suffit d'indiquer \verb+suffix_3p "~sq."+

Voyons maintenant comment l'on peut insérer les lettres de l'alphabet entre les groupes d'entrées. Cette manipulation permet de rendre plus lisible un index un peu long. 

Le manuel indique que l'option par défaut \verb+headings_flag 0+ ne met pas de séparateurs entre les groupes, que l'option \verb+headings_flag 1+ permet d'obtenir des lettres majuscules comme séparateur, et l'obtion  \verb+headings_flag -1+ des lettres minuscules.

Définissons ensuite la manière dont vont apparaître ces lettres. On utilise la commande \verb| heading_prefix  |. Supposons que je veuille faire apparaître ces lettres-séparatrices en gras. Je n'oublie pas que l'on code en \TeX … J'écris donc :

\begin{minted}{tex}
heading_prefix "\\bf"
\end{minted} 

Si je veux qu'elles apparaissent en italique, j'écris \verb| "\\it" |… (on utilise \verb+ \bf + et \verb+ \it + , qui sont des commandes \TeX, et non \verb+ \textbf + et \verb+ \textit +).

Je peux de même changer la taille des lettres-séparatices en ajoutant \verb| \\large | ou \verb| \\Large|, et ainsi de suite. 

Mais j'ai ouvert une accolade… il me faut la refermer:

\begin{minted}{latex}
\verb| heading_suffix "}\\nopagebreak\n" | 
\end{minted}

\verb|\\nopagebreak| évite qu'une lettre-séparatrice se retrouve seule en fin de page. 



Ce n'est pas fini. Mes lettres sont actuellement justifiées à gauche, comme les entrées de l'index. Je voudrais qu'elles soient centrées: 

\begin{minted}{latex}
headings_flag 1
heading_prefix "\\bf\\large\\hfill" 
heading_suffix "hfill}\\nopagebreak\n" 
\end{minted}

Les deux \verb|\\hfill| étirent, pour le premier, l'espace avant la lettre vers la droite, et pour le second l'expace suivant la lettre vers la gauche -- voilà mes lettres centrées.

\begin{quotation}
\begin{tabbing}
\hspace{0,5cm}  \= \hspace{1cm} \= \hspace{1,5cm} \= \kill
\>\> \large{\textbf{A}}\\
\\
Adrien\>\>\> 5 \\
\\
\>\> \large{\textbf{C}}\\
\\
Charlemagne \>\>\>5 \\
\\
\>\> \large{\textbf{D}}\\
\\
Ducs \\
\> Tassilon\>\> 5\\
\\
\>\> \large{\textbf{E}}\\
\\
Evêques \\
\> Damase \>\> 5\\
\> Formose\>\>  5\\

\end{tabbing}
\end{quotation}


Si je préfère qu'elles soient simplement légèrement décalées vers la droite, je peux utiliser, à la place de \verb| \\hfill |, \verb| \\hspace*{-1ex} |  qui rajoute à gauche un espace de la taille d'un -x. \renvoi{unite}\renvoi{espace}


Quand on a compris ce système, il est alors assez simple de formater à sa guise l'index. C'est pourquoi, pour les autres options, je renvoie ici à la lecture du manuel\footcite[On pourra aussi consulter][]{frama_index}. 




