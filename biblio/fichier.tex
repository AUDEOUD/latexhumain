\chapter{Remplir sa base de donnée bibliographique}

\begin{prealable}
La première étape pour gérer une bibliographique avec \logiciel{\LaTeX} est d'entrée l'ensemble des références dans une base de donnée bibliographique.

À vrai dire, cela se fait en général bien avant la rédaction du travail.

La base de donnée bibliographique est un fichier au format \logiciel{BibTeX} qui porte une extension \ext{.bib}.
\end{prealable}

\section{Strucure de base d'un fichier \ext{.bib}}

Chaque élément d'une bibliographie sera appelé entrée. Une entrée bibliographique se caractérise par :
\begin{itemize}
\item Un type d'entrée : s'agit-il d'un article, d'un livre, d'actes de colloques ?
\item Une clef unique. Cette clef permet de distinguer une entrée bibliographique d'une autre. C'est elle qui sera mentionnée dans vos fichiers \ext{.tex}
\item Des champs contenant des indications sur l'ouvrage, tel que : auteur, titre, éditeur etc. 
\end{itemize}

L'avantage de ce système est compréhensible. En effet, une correction dans la base de donnée bibliographique sera automatiquement reportée dans le travail\footnote{Moyennant une nouvelle compilation.} à chaque fois que l'entrée est mentionnée.

Un fichier \ext{.bib} est un fichier texte contenant une suite d'instruction, de la forme suivante.


\begin{minted}{latex}
@type{clef,
	Champ1 = {Valeur 1},
	Champ… = {Valeur …},
	Champn = {Valeur n}
}
\end{minted}



Par exemple

\inputminted{exemples/biblio/premierpas/urner.bib}


Qui pourra être affiché plus tard sous la forme :

\begin{quote}
\cite{Urner1952}
\end{quote}


Signifie que je déclare un livre (\verb|@book|) auquel j'attribue la clef \verb|Urner1952|. Ce livre est une oeuvre de Urner, Hans de son prénom, il s'intitule (mettre le nom proprement) a été publié en … à …

Un fichier .bib n'est rien d'autre qu'une \renvoi{logiciel:biblio} série d'entrées de ce type. Cependant, comme il est assez aisé de s'embrouiller, notamment dans les ouvertures et fermetures des accolades, il vaut la peine d'utiliser un logiciel de gestion de bibliographies, capable de gérér  des fichiers au formats \ext{.bib}. Un certain nombre sont présentés en annexe.

La documentation de \package{BibLaTeX} contient l'ensemble des champs et des types possibles. Pour éviter de répéter ce manuel, nous vous présenterons les champs en les classant par catégorie. Mais auparavant, nous dirons donnerons quelques conseils sur le choix de la clef et sur les différents types de catégories.

\section{Le choix de la clef}

Une entrée bibliographique doit avoir une clef \emph{unique}. Cette clef doit comporter uniquement des caractères alphanumériques non accentués, éventuellement avec des tirets. 

Voici le conseil que nous donnons pour le choix de la clef :
\begin{description}
\item[Pour les oeuvres contemporaines, notamment les sources secondaire] prendre le nom de l'auteur suivi de la date. Par exemple \verb|Urner1952|.
\item[Pour les oeuvres anciennes] prendre le nom de l'oeuvre  éventuellement abregée : par exemple \verb|ContraFaustum| pour parler du … Si plusieurs oeuvres d'auteurs différents portent le même nom, faire précéder du nom de l'auteur.
\item[Pour les oeuvres anonymes ou pseudépigraphiques] prendre soit le numéro d'entrée dans un index de référence, soit le numéro de placement dans une collection. Par exemple : \verb|PsAugustin1288| renvoie au texte Pseudo Augustinien qui se trouve à l'entrée … dans le …
\end{description}


Éventuellement, tout ceci ne sont que des conseils et des exemples, pas des règles absolues. A vraie dire la meilleur règle est celle qu'on se fixe, et tout dépend de la matière.

\section{Les différentes types d'entrées}
Voici les différents types d'entrées disponibles\footnote{A vrai dire il est possible de rajouter les siennes, mais nous vous le déconseillons}.

\rev{Mettre des informations}
\begin{longtable}{lll}
	Type BibTeX 				& Signification			& Explication et commentaire
	\endhead
	@article				& Article d'un périodique		& Pour les communications de colloques, voir plus bas. \\
	@book					& Livre							& Pour les actes de colloques, voir plus bas \\
	@bookinbook				& Livre dans un livre			& Par exemple lorsque un auteur ancien voit ses livres rassemblés en un seul volume. \\		
	@collection				& Ouvrage collectif mais avec des parties distinctes par auteurs.				& \\
	@manuel					& Manuel						& Pas nécessairement sous forme imprimé.\\
	@misc					& Entrée générique				& Pour tout type d'entrée non catégorisable. Exemple : tableau, manuscrit, ostrakon. \\
	@online					& Ressource internet			& Si une ressource se trouve sous forme papier et sous forme éléctronique, utiliser la champ correspondant à la forme papier. \\
	@patent					& Brevet industriel		 (?)	& \\
	@periodical				& Numéro précis d'un périodique & \\
	@proceedings			& Actes de colloques			& \\
	@reference				& Dictionnaire, Encyclopédie ou apparenté. & \\ 
	@report					& Rapport						& \\
	@masterthesis			& Mémoire de maîtrise ou de master& \\	
	@phdthesis				& Thèse de doctorat				& \\
	@unpublished			& Ouvrage non publié			& Par exemple un projet\\
	
\end{longtable}

Il existe des entrées de type @inbook ; @inproceedings ; @incollection ; @inreference pour désigner des parties précise d'une entrée de type @book ; @proceedings ; @reference. Ainsi un communication d'un colloque est-il une entrée de type @inproceedings.

\section{Les différents champs possibles}

Il n'est pas question ici de lister l'ensemble des champs disponibles : on consultera le manuel de \package{BibLaTeX} pour avoir une liste exhaustive. Nous précisons ici les champs utiles pour une première approche.

\subsection{Les champs de personnes}

Ces champs servent à désigner des personnes qui ont participé au processus de production de l'œuvre : auteur, annotateur, éditeur (scientifique) etc. Tous ces champs ne sont évidemment pas à remplir. Voici la liste  :


\begin{longtable}{p{0.2\linewidth}p{0.75\linewidth}}
	champ 		& Explication \\
	\endhead
	afterword		& Auteur(s) de la postface \\
	annotator		& Auteur(s) des annotations \\
	author		& Auteur(s) de l'œuvre    \\
	bookauthor		& Auteur(s) du livre dans lequel l'œuvre est insérée. \\
	commentator	& Auteur(s) des commentaires \\
	editor			& Éditeur(s) scientifique(s). On peut préciser son rôle grâce au champ \champ{editortype}	\\
	editora		& Éditeur(s) scientifiques ayant un autre rôle. On peut préciser son rôle grâce au champ \champ{editoratype}  \\
	editorb		& Éditeur(s) scientifiques ayant un autre rôle. On peut préciser son rôle grâce au champ \champ{editorbtype}  \\
	editorc		& Éditeur(s) scientifiques ayant un autre rôle. On peut préciser son rôle grâce au champ \champ{editorctype}  \\
	foreword		& Auteur(s) de la préface\\
	holder			& Titulaire d'un brevet industriel \\
	introduction		& Auteur(s) de l'introduction \\
	translator 		& Traducteur(s) 		\\
\end{longtable}

Lorsque des champs possèdent des valeurs identiques (par exemples les champs \champ{publisher} et \champ{translator}, \package{BibLaTeX} va  fusionner ces champs lors de l'affichage. Prenons ainsi l'entrée suivante : 

\inputminted{exemples/biblio/premierpas/augustin_editeur.bib}

Elle sera affichée ainsi : 

\begin{quote}
\cite{DoctrineChretienne}
\end{quote}

\subsubsection{Comment entrer un nom de personne}

Ces différents champs prennent comme valeur un ou plusieurs noms de personnes. S'il y a plusieurs noms, il suffit de les séparer par le mot-clef \forme{and}. Par exemple pour les auteurs de l'ouvrage que vous avez entre les mains : 

\begin{minted}{latex}
author = {Maïeul Rouquette and Enimie Rouquette}
\end{minted}

Un nom se caractérise en LaTeX par :
\begin{description}
\item[Un ou plusieurs prénoms.]L'initiale doit être en majuscule, le reste en minuscule. Exemple : \forme{Albert}. Il vaut mieux mettre le prénom complet : on peut confier le travail de limitation à une minuscule  à \package{BibLaTex}.
\item[Un nom.]L'initiale doit être en majuscule, le reste en minuscule. Exemple : \forme{Londres} et non pas \forme{LONDRES}. \package{BibLaTex} se chargera le cas échéant de mettre en majuscule.
\item[Éventuellement une particule.] Elle doit être entièrement en minuscule. Par exemple : \forme{de}, \forme{von}.
\item[Éventuellement un suffixe] avec l'initiale en majuscule. Ce type de donnée est plutôt anglo-saxonne. Exemple : \forme{Junior}.
\end{description}

En ce qui concerne la syntaxe, elle peut être :
\begin{itemize}
\item\enquote{Prénoms + (particule) + Nom}
\item\enquote{particule + Nom, (suffixe), Prénoms} 
\end{itemize}

Ainsi les entrées suivantes sont équivalentes :

\begin{minted}{latex}
Victor Marie Hugo
\end{minted}

\begin{minted}{latex}
Hugo, Victor Marie
\end{minted}

Dans le premier cas, BibTeX considère que le dernier mot commençant par une majuscule est le nom de famille. Dans le second cas, il considère l'ensemble situé avant la virgule comme le nom de famille. Ce qui est utile pour les noms composés. Ainsi pour parler de Charles De Gaulle :

\begin{minted}{latex}
De Gaulle, Charles
\end{minted}

Si je parle de Simone de Beauvoir, le \forme{de} étant une particule je peux utiliser la première syntaxe : dans ce cas BibLaTeX considère tout ce qui suit la particule comme constituant le nom :

\begin{minted}{latex}
Simone de Beauvoir
\end{minted}

Mais la seconde syntaxe fonctionne également :

\begin{minted}{latex}
de Beauvoir, Simone
\end{minted}

Pour distinguer Alexandre Dumas père d'Alexandre Dumas fils, on peut utiliser le suffixe :

\begin{minted}{latex}
Dumas, Fils, Alexandre
\end{minted}

Un cas problématique est celui des auteurs anciens, où l'on écrit souvent le prénom suivi de la ville d'exercice ou de naissance, comme par exemple pour Grégoire de Tours.

Si j'écris : 

\begin{minted}{latex}
Grégoire de Tour
\end{minted}

BibLaTeX va comprendre qu'il s'agit d'une personne prénommée Grégoire, dont le nom est Tours et la particule de. Par conséquent il va l'afficher sous la forme \forme{Tours, Grégoire, de.}. Pour éviter ce problème, il suffit d'utiliser des accolades :
\begin{minted}{latex}
{Grégoire de Tour}
\end{minted}

Pour résumer, voici un exemples de quelques entrées correctes\footnote{Le lecteur exigeant pardonnera aux auteurs de ne mettre ici que les noms d'auteurs et titres d'œuvres}.

\inputminted{exemples/biblio/premierpas/noms.bib}

\begin{quote}
\cite{HugoMiserable} \\
\cite{HugoLegende} \\
\cite{DeGaulle}	\\
\cite{BeauvoirSexe} \\
\cite{BeauvoirMemoires} \\
\cite{Dumas} \\
\cite{Gregoire} \\ 
\end{quote}

\subsection{Les champs de titre}


\begin{longtable}{p{0.2\linewidth}p{0.75\linewidth}}
	champ 		& Explication \\
	\endhead
	booksubtitle 	& Sous-titre du livre dans lequel l'entrée se situe. \\
	booktitle	 	& Titre du livre dans lequel l'entrée se situe. 		\\
	booktitleaddon	& Ajout au titre du livre dans lequel l'entrée se situe \\
	chapter			& Chapitre d'un livre. Pour les entrées de type @inbook	\\
	eventitle		& Titre du colloque, pour les entrées de type @proceedings et @inproceedings \\
	issuesubtitle	& Sous-titre d'un numéro spécifique d'un périodique. 	\\
	issuetitle		& Titre d'un numéro spécifique d'un périodique.			\\
	journalsubtitle	& Sous-titre d'un périodique.							\\
	journaltitle	& Titre d'un périodique. Le champs \champ{journal} est un alias de ce champs.							\\
	mainsubtitle	& Sous-titre d'une œuvre en plusieurs volumes.			\\
	maintitle		& Titre d'une œuvre en plusieurs volumes. Le titre du volume spécifique à notre entrée correspond au champs \champ{title}.						\\
	maintitleaddon & Ajout au titre d'une œuvre en plusieurs volumes.		\\
	origtitle		& Titre originale de l'œuvre, si traduction. N'est pas affiché en standard. \\
	reprinttitle	& Titre d'un reprint. N'est pas affiché en standard.	\\
	subtitle		& Sous-titre de l'œuvre.									\\
	title			& Titre de l'œuvre.									\\
	titleaddon		& Ajout au titre de l'œuvre. Dans ce manuel, nous conseillons de l'utiliser pour les divisions de source. \\	
\end{longtable}

Voici quelques exemples afin de comprendre comment se servir de ces champs.

\subsubsection{Un livre avec un sous-titre}

\inputminted{exemples/biblio/premierpas/saxer.bib}

\begin{quote}
\cite{Saxer1980}
\end{quote}

\subsubsection{Un livre situé dans un recueil}

\inputminted{exemples/biblio/premierpas/felix.bib}

\begin{quote}
\cite{ContraFelix}
\end{quote}

\subsubsection{Un article dans une revue}

\inputminted{exemples/biblio/premierpas/junod.bib}

\begin{quote}
\cite{Junod1992}
\end{quote}

\subsection{Les champs d'informations sur la publication}

\begin{longtable}{p{0.2\linewidth}p{0.75\linewidth}}
	champ 		& Explication \\
	\endhead
	adress			& Lieu de publication (alias du champ \champ{location}).	\\
	date			& Date de publication, sous la forme \verb|AAAA-MM-JJ/AAAA-MM-JJ|. La première date indiqué correspondant à la date de début, la seconde à la date de fin. Pour n'indiquer qu'une date de début, mettre \verb|AAAA-MM-JJ/|. On peut ne pas indiquer le mois ou le jour. On peut également utiliser les champs \champ{year} et {month} à la place. \\
	edition		& Numéro d'édition, si plusieurs éditions successives. Doit être un entier ou bien un chaîne de caractères.				\\
	eventdate		& Date du colloque pour les entrées du type @proceedings et @inproceedings. \\
	institution		& Institution dans laquelle l'œuvre a été produite. Typiquement pour les entrées de type @thesis. \\
	issue			& Spécification d'un numéro spécifique d'un périodique (par exemple \enquote{numéro d'été}). On préférera les champs \champ{year}, \champ{month}.	\\
	language		& Langue de l'œuvre. Le nom de la langue doit, idéalement, être comme indiqué dans la documentation de \package{polyglossia}.					\\
	location		& Lieu d'impression.  					\\
	month			& Mois de publication. Doit être un entier compris entre 1 et 12. \\
	number		& Numéro d'un périodique ou numéro au sein d'une collection. 	\\
	organization		& Organisation à l'origine d'un manuel ou d'une page internet.	\\
	origdate		& Date de l'édition originale.						\\
	origlanguage	& Langue originelle. Le nom de la langue doit, idéalement, être comme indiqué dans la documentation de \package{polyglossia}. \\
	origlocation		& Lieu d'impression de l'édition	originelle.		\\
	origpublisher	& Éditeur (commercial) de l'édition originelle.		\\
	part			& Pour les livres en plusieurs volumes, indique le numéro du volume physique. \\
	publisher		& Éditeur commercial.					\\
	pubstate		& Pour les œuvres non encore imprimées, indique le statut :
					\begin{description}
						\item[inpress]œuvre sous presse.
						\item[inpreparation]œuvre en préparation.
						\item[submitted]œuvre soumise à évaluation.
					\end{description}
					
					\\
	type			& Pour les entrées de type  @manual, @patent et @report, précise le type de travail.
	
	 Pour les entrées de type @manual, deux valeurs possibles :
					\begin{description}
						\item[mathesis]mémoire de master.
						\item[phdthesis]thèse de doctorat.
					\end{description}
					
					 Pour les entrées de type @patent, plusieurs  valeurs possibles : 
					 
					 
					 \begin{description}
						\item[patentde] Brevet allemand.
						\item[patenteu] Brevet européen.
						\item[patentfr] Brevet français.
						\item[patentuk] Brevet grand-breton.
						\item[patentus] Brevet États-Unien.
						\item[etc.]
					 \end{description}
					
					Pour les entrées de type @report : 
					\begin{description}
						\item[techreport]rapport technique.
						\item[resreport]rapport de recherche.
					\end{description}
					
					\\	
	url 			& Url (adresse électronique) d'une publication en-ligne.
	urldate		& Date à laquelle une publication électronique a été consultée.
	venue			& Lieu du colloque pour les entrées du type @proceedings et @inproceedings. \\
	version		& Numéro de révision d'un manuel, d'un logiciel. \\
	volume		& Volume dans une œuvre en plusieurs volumes. \\
	volumes		& Nombre de volume dans une œuvres en plusieurs volumes. \\
	year			& Année de publication 				\\
	
\end{longtable}


Évidemment tout ces champs ne sont pas à systématiquement remplir : le lecture sera mieux que nous lesquels remplir.

\subsection{Les champs d'identifications formels}

Il peut être utile d'indiquer des informations comme l'ISBN etc. Voici les champs possibles\footcite[BibLaTeX imprimera ces champs par défaut, il est toutefois possible de ne pas les afficher en passant une option au chargement du package]{explicationisbn}.

\begin{longtable}{p{0.2\linewidth}p{0.75\linewidth}}


\end{lontable}

