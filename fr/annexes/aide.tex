\chapter{Trouver de l'aide}

\begin{intro}
Comment faire lorsque l'on est coincé sur un point particulier, qu'on a relu les différents manuels plusieurs fois\footcite[Signalons au passage la possibilité de télécharger une aide sur l'ensemble des erreurs de compilation avec \LaTeX :][]{erreurscompilo} ? Demander de l'aide à d'autres utilisateurs de \LaTeX{}.

Voici quelques endroits où le faire. 
\end{intro}


\section{Forums internet}

Les forums Internet sur \LaTeX sont pléthore. En anglais, on peut utiliser celui de \forme{LaTeX Community} : \url{http://www.latex-community.org/forum/}. En français, on peut utiliser celui du site \forme{Developpez.net} \url{http://www.developpez.net/forums/f149/autres-langages/autres-langages/latex/}.


\section{Messagerie instantanée}

Il est possible de demander de l'aide sur différents salons de messagerie instantanée fonctionnant par IRC\footnote{Lointain ancêtre de Skype, MSN et autres GoogleTalk.}. En général on peut y trouver de l'aide assez rapidement.

Pour se connecter à un salon de discussion IRC, on peut utiliser le plugin Chatzilla du logiciel libre Firefox.

En français, l'adresse est \url{irc://irc.rezosup.org/latex} ; en anglais
\url{irc://irc.freenode.net/latex}.


\section{Liste de discussion}

En français, celles de l'association Gutemberg : \url{http://www.gutenberg.eu.org/?Listes-de-diffusion-gerees-par} ; en anglais,  la liste suivante : \url{http://groups.google.com/group/comp.text.tex/topics}, dont il est possible de se servir via un logiciel de gestion de \enquote{Newsgroups}, comme par exemple Mozilla Thunderbird.
