\chapter{Indiquer des références bibliographiques}

\bibverbose

\begin{intro}
Nous avons vu plus haut les différentes manières de citer un texte\renvoi{citertexte} dans le corps du travail. Nous avons également vu comment remplir une base de données bibliographique\renvoi{bddbiblio}.

Il ne nous reste plus qu'à mettre en relation les textes cités avec la base de données constituée pour avoir un travail correct, en indiquant les références des citations. C'est l'objet de ce chapitre.

\end{intro}


\section[Appel du package]{Appel du package \package{biblatex}}

La gestion bibliographique s'effectue avec le package \package{biblatex}. Dans le préambule, on appelle le package dans sa forme la plus simple par la commande:
\cs{usepackage}\verb|{biblatex}|.


Cependant, le package dispose de nombreuses options\footcite{biblatex_options}. Celle qui nous intéresse pour le moment est \option{citestyle}, qui permet de définir la manière dont les références bibliographiques sont affichées, notamment lorsqu'elles sont citées à plusieurs reprises.

Il existe un nombre important de styles de citation livrés en standard. Nous mentionnons ici les principaux\footcite[Se reporter pour plus de détails à][]{biblatex_style}:
\begin{choix}
\item[numeric]chaque entrée se voit attribuer un numéro, qui est appelé lorsque l'on renvoie à cette entrée. La bibliographie finale indique les correspondances.
\item[authortitle]sont indiqués seulement l'auteur et le titre de l'œuvre.
\item[verbose]la description complète de l'entrée est donnée la première fois, une version abrégée est affichée ensuite.
\item[verbose-ibid]la description complète de l'entrée est donnée, mais si un passage est cité plusieurs fois de suite, l'abréviation \emph{ibidem} est utilisée.
\item[verbose-note]la description complète de l'entrée est donnée à sa première mention, puis une version abrégée est utilisée.
\item[verbose-trad1; verbose-trad2; verbose-trad3]~la description complète de l'entrée est utilisée, puis les abréviations universitaires de type \emph{op. cit.}, \emph{ibidem.} etc. sont utilisées ; \package{biblatex} calcule automatiquement l'opportunité d'utiliser une abréviation universitaire, selon l'endroit où l'ouvrage aura été cité précédemment. Voir le manuel pour la description complète des différences entre ces trois styles.
\end{choix}

On comprend ici l'un des grands intérêts de \LaTeX : plus de perte de temps à se demander : \enquote{Faut-il que j'utilise ici une version abrégée de la référence ?}, \package{biblatex} le fait pour vous.

Voici donc comment doit se faire l'appel au package si nous choisissons le style de citation \verb|verbose-trad-2|:

\begin{latexcode}
\usepackage[citestyle=verbose-trad2]{biblatex}
\end{latexcode}

Pour les sciences humaines, nous recommandons les styles de la famille verbose.


\section{Appel de la base de donnée bibliographique}


Pour que \LaTeX{} sache où chercher les références bibliographiques, il faut lui signaler  quel est le fichier \ext{bib} à utiliser, pour cela il suffit d'utiliser dans le préambule la commande   :
\csp{bibliography}\marg{chemin du fichier}.

Le chemin du fichier s'indique de la même manière que n'importe quel chemin de fichier\renvoi{chemin}.

\begin{attention}
Il est possible d'appeler plusieurs fichiers bibliographiques. Nous le déconseillons, dans la mesure où cela contraint à vérifier qu'il n'y ait pas plusieurs entrées avec la même clef.
\end{attention}

\section{Citation d'un élément bibliographique}

L'ensemble des commandes de citation ont la syntaxe  : 

\csp{\meta{prefix}cite}\oarg{prenote}\oarg{postnote}\marg{clef}, \meta{prefix} indique où et comment doit apparaître la référence : directement dans le texte, entre parenthèses, en note de bas de page, etc.

Par exemple, nous souhaitons citer  un livre de Victor Saxer que nous avons entré de cette manière dans la base de données :


\inputminted{exemples/biblio/fichier/saxer.bib}

Nous écrivons \csp{cite}\verb|{Saxer1980}|. Après la troisième compilation\renvoi{3compil}, la référence apparaît selon le style choisi lors de l'appel à  \package{biblatex}. Ainsi, pour un style de la famille verbose :

\begin{quotation}
\fullcite{Saxer1980}
\end{quotation}

L'usage en sciences humaines est de citer en note de bas de page. C'est pourquoi il vaut mieux utiliser : \csp{footcite}\verb|{Saxer1980}|,\label{footcite} qui met la référence en note de bas de page, en ajoutant automatiquement le point de fin de note. On peut également décider d'utiliser la commande \csp{parencite}, pour citer entre parenthèses. 

Mais on peut aussi choisir la commande \csp{autocite}. Le mode de citation (note de bas page, citation directe, citation entre parenthèse, etc.) dépend alors du style de citation choisi : pour les styles de la famille verbose, la référence est mise en note de bas de page. 

\begin{plusloins}
Par défaut les champs bibliographiques sont séparés par des points. Si vous êtes pressé d'avoir des virgules à la place, rendez-vous au chapitre~\ref{style1}.
\end{plusloins}

\subsection{Les arguments \arg{prenote} et \arg{postnote}}

Supposons que nous souhaitons afficher un texte avant notre référence. Par exemple : \enquote{Voir également}. On utilise l'argument optionnel \arg{prenote}.

\begin{latexcode}
Blabla \autocite[Voir également][]{Saxer1980} blablabla.
\end{latexcode}

\begin{quotation}
Blabla \cite[Voir également][]{Saxer1980} blablabla.
\end{quotation}



On peut également vouloir afficher quelque chose après la référence. On utilise dans ce cas l'argument \arg{postnote}.

\begin{latexcode}
Blabla \autocite[Voir également][qui porte sur un
sujet similaire.]{Saxer1980} blabla
\end{latexcode}

\begin{quotation}
Blabla \cite[Voir également][qui porte sur un sujet similaire.]{Saxer1980} blabla
\end{quotation}

\begin{attention}
Si on utilise l'argument \arg{prenote}, il est obligatoire d'indiquer un argument \arg{postnote}, fût-il vide. En l'absence de cet argument, \package{biblatex} interprète  ce que vous pensiez être \arg{prenote} comme \arg{postnote}.
\end{attention}

\subsection{L'argument \arg{postnote} et la numérotation des pages}\label{pagespostnote}

L'argument \arg{postnote} peut servir à indiquer les pages précises du passage cité\footcite[On consultera pour plus de détails : ][]{biblatex_pages}. Il suffit pour cela qu'il contienne une valeur de type numérique, en chiffres arabes ou romains, ou bien une séquence de valeurs numériques.

Par exemple, pour citer la page 25 : 

\begin{latexcode}
\autocite[25]{Saxer1980}
\end{latexcode}

 Pour citer les pages 25 à 35 :

\begin{latexcode}
\autocite[25-35]{Saxer1980}
\end{latexcode}

Ou encore les pages 25 et 35:

\begin{latexcode}
\autocite[25 \& 35]{Saxer1980}
\end{latexcode}

Pour non seulement citer la page précise mais  aussi ajouter autre chose dans l'argument \arg{postnote}, utilisons la commande \csp{pno}, afin que \package{biblatex} insère lui même le \forme{p.} :

\begin{latexcode}
\autocite[\pno~22,  avec lequel nous sommes en désaccord.]{Saxer1980}
\end{latexcode}

\begin{quotation}
\cite[\pno~22, avec lequel nous sommes en désaccord.]{Saxer1980}
\end{quotation}

On pourra également utiliser les commandes \csp{nopp} pour ne pas afficher de préfixe de pagination,  \csp{psq} pour indiquer qu'il  faut également prendre la page suivante et \csp{psqq} pour indiquer qu'il faut prendre les pages suivantes.

\subsubsection{Problème avec le champ \champ{pages}}


Un problème se pose lorsqu'il y a déjà un champ \champ{pages} de rempli, pour un article par exemple. On se retrouve en effet avec deux numérotations de pages : celle du champ \champ{pages} et celle du passage précis que l'on cite, indiquée par l'option \arg{postnote}. Prenons l'entrée suivante :

\inputminted{exemples/biblio/fichier/junod.bib}

Et appelons-la avec le code suivant :

\begin{latexcode}
\cite[24]{Junod1992}
\end{latexcode}

On obtient ce résultat :

\begin{quotation}
\fullcite[24]{Junod1992}
\end{quotation}



Pour éviter cet inconvénient, il faut, si on utilise un style de type verbose, passer une option lors de l'appel au package \option{citepages=omit}

\begin{latexcode}
\usepackage[citestyle=verbose,citepages=omit]{biblatex}
\end{latexcode}

Désormais écrire \cs{cite}\verb|[24]{Junod1992}|
affiche correctement :

\begin{quotation}
\cite[24]{Junod1992}
\end{quotation}

En revanche le problème demeure si l'on souhaite insérer du texte dans l'argument \arg{postnote} après le numéro de page.

\begin{latexcode}
\cite[\pno~24, passage fort intéressant.]{Junod1992}
\end{latexcode}

\begin{quotation}
\cite[\pno~24, passage fort intéressant.]{Junod1992}
\end{quotation}

Comme vous pouvez le constater, le champ \champ{pages} continue à être affiché. Il s'agit d'une limite de \emph{biblatex}. Nous ne trouvons pas d'autre solution que de le contourner en indiquant le texte qui suit la référence en dehors de l'argument \arg{postnote}. 

Dans le cas d'une citation en note de bas de page, il faut donc écrire :

\begin{latexcode}
\footnote{\cite[24]{Junod1992}, passage fort intéressant.}
\end{latexcode}

\begin{attention}
Pour les sources anciennes, on aimerait pouvoir indiquer non seulement la pagination dans l'édition contemporaine mais aussi la division de source (livre, chapitre, paragraphe, etc.). Malheureusement on ne peut avec \package{biblatex} indiquer qu'un seul élément variable lors de l'utilisation d'une commande de citation.

On peut contourner ce problème en créant des sous-entrées bibliogra\-phiques\renvoi{divisionsource}, dont nous parlons dans un autre chapitre. 
\end{attention}


\begin{plusloins}
Pour les œuvres en plusieurs volumes, on peut utiliser les commandes \csp{volvcite}, \csp{pvolcite}, \csp{fvolcite}, \csp{avolcite} correspondant aux commandes respectives \cs{cite}, \cs{parencite}, \cs{footcite}, \cs{autocite}.

La syntaxe est  : 
\csp{\meta{prefix}cite}\oarg{prenote}\marg{volume}\oarg{page}\marg{clef}.
\end{plusloins}

\subsubsection{Type de pagination}

On peut préciser dans sa base de données le type de pagination d'une entrée : pagine-t-on en pages, en colonnes, etc ? On utilise  pour cela le champ \champ{pagination}, en lui donnant l'une des valeurs suivantes : 

\begin{description}
\item[page] la pagination est sous forme de pages. C'est la valeur standard.
\item[column] la pagination est sous forme de colonnes.
\item[line] la pagination est sous forme de lignes.
\item[verse] la pagination est sous forme de versets / de vers. 
\item[section] la pagination est sous forme de sections.
\item[paragraph] la pagination est sous forme de paragraphes.
\item[none] la pagination est libre.
\end{description}

Le champ  \champ{pagination} sert lorsque l'on met une indication numérique dans l'argument \arg{postnote}. En revanche si on n'utilise pas d'argument \arg{postnote} et qu'on laisse le contenu du champ \champ{pages}, il est sans effet. Prenons l'entrée suivante :

\inputminted{exemples/biblio/citation/desnos.bib}

Appelons-la en utilisant l'argument \arg{postnote}, puis sans l'utiliser.


\begin{latexcode}
\autocite[2]{desnos}

\autocite{desnos}
\end{latexcode}

% On modifie pour avoir le choix standard :
\DeclareFieldFormat{pages}{\mkpageprefix[bookpagination]{#1}}

\begin{quotation}
\cite[2]{desnos}

\cite{desnos}
\end{quotation}

% On rétablis
\DeclareFieldFormat{pages}{\mkpageprefix[pagination]{#1}}

On constate que dans le deuxième cas, \champ{pagination} n'est pas pris en compte.
Il faut dans ce cas mettre le type de pagination dans le champ \champ{bookpagination} :

\begin{latexcode}
@book{desnos,
    Author = {Robert Desnos},
    Pagination = {verse},
    Bookpagination = {verse},
    Pages = {1-3},
    Title = {La fourmi}}
\end{latexcode}

\begin{quotation}
\cite[2]{desnos}

\cite{desnos}
\end{quotation}


\begin{plusloins}
On peut éviter de dupliquer l'information dans le champ \champ{bookpagination} en insérant la ligne suivante en début de fichier \ext{tex} :

\begin{minted}{latex}
\DeclareFieldFormat{pages}{\mkpageprefix[pagination]{#1}}
\end{minted}

La commande \csp{DeclareFieldFormat} indique la manière d'afficher le champ \champ{pages} : en passant son contenu (\verb|#1|) à une fonction \csp{mkpageprefix}, à qui on indique de se baser sur le champ \champ{pagination} pour afficher le préfixe de page\footcite[Voir][]{biblatex_formating}.
\end{plusloins}

\begin{plusloins}
Il est possible de définir ses propres types de pagination en déclarant des nouvelles chaînes de langues\renvoi{i18nchaines}. Nous détaillons sur notre blog la méthode\footcite{biblio_pagination}.
\end{plusloins}

\section{Citation de plusieurs œuvres}\label{citemultiple}

Il est possible de citer plusieurs œuvres d'une traite, en  utilisant une commande dont la syntaxe de base est :

\csp{\meta{prefix}cites}\oarg{prenote}\oarg{postnote}\marg{clef}\oarg{prenote}\oarg{postnote}\marg{clef}…

On peut  citer à la suite autant d'œuvres que souhaité. Chaque élément cité a droit à son argument \arg{prenote} et \arg{postnote}, qui s'utilise de la même manière que pour les citations simples.
Voici un exemple d'utilisation : 

\begin{latexcode}
\autocites{Saxer1980}{Junod1992}
\end{latexcode}

\begin{quotation}
\cites{Saxer1980}{Junod1992}
\end{quotation}

Il est possible de préciser un texte à afficher avant la liste des références, ainsi qu'un texte à afficher après.

\begin{latexcode}
\<prefix>cites(Texte avant)(Texte après){Saxer1980}{Junod1992}
\end{latexcode}

Les séparateurs de citations sont par défaut des points-virgules. Il est possible de les modifier globalement, nous en parlons plus loin\renvoi{multicitedelim}.

Toutefois, si l'argument \arg{postnote} d'un élément de la commande de citation multiple finit par un signe de ponctuation, alors le point-virgule n'apparaît pas entre cet élément et le suivant :

\begin{latexcode}
\autocites[on consultera également :]{Saxer1980}{Urner1952}
\end{latexcode}

Donne :

\begin{quotation}
\cites[on consultera également :]{Saxer1980}{Urner1952}
\end{quotation}


\section{Choix de la forme abrégée}\label{shortfields}

Avec le style verbose, on peut faire apparaître les références sous une forme abrégée. Il existe pour cela plusieurs champs, que nous n'avons pas encore mentionnés.

\begin{choix}
    \item[shortauthor] Nom abrégé de l'auteur.
    \item[shorteditor] Nom abrégé de l'éditeur.
    \item[shorthand] Forme abrégée de la référence.
    \item[shorthandintro] Lorsqu'une entrée est citée pour la première fois, et si le champ \champ{shorthand} est utilisé, le champ \champ{shorthandintro} sert à introduire la forme abrégée. Par exemple, \enquote{cité plus tard :}.
    \item[shorttitle] Forme abrégée du titre.
\end{choix}

La commande \csp{printshorthands} permet d'imprimer la liste des abréviations.

\bibverbosetrad
