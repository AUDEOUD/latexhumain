\usepackage{doc}
\def\cmd#1{\cs{\expandafter\cmd@to@cs\string#1}}
\def\cmd@to@cs#1#2{\char\number`#2\relax}
\def\cmd#1{\cs{\expandafter\cmd@to@cs\string#1}}
\providecommand\marg[1]{%
  {\ttfamily\char`\{}\meta{#1}{\ttfamily\char`\}}}
\providecommand\oarg[1]{%
  {\ttfamily[}\meta{#1}{\ttfamily]}}
\providecommand\parg[1]{%
  {\ttfamily(}\meta{#1}{\ttfamily)}}



\newcommand*{\cf}[0]{\emph{cf}.}


\newcommand{\forme}[1]{\enquote{#1}}

\newcommand{\classenoidx}[1]{%
	\textbf{#1}%
}
\newcommandx{\classe}[2][1]{%
	% Premier argument optionel = la manière dont doit être indexé
	\classenoidx{#2}%	
	\ifthenelse{\equal{#1}{}}{\index[classe]{#2}}{\index[classe]{#1}}%
}
\newcommand{\champ}[1]{%
	\index[champ]{#1}%
	\textbf{#1}%
}

\newcommand{\fichier}[1]{%
	#1%
}
\newcommand{\environoidx}[1]{%
	\mbox{\textbf{#1}}%
}
\newcommand{\enviro}[1]{%
	\environoidx{#1}%
	\index[enviro]{#1}%
}



\newcommand*{\ext}[1]{%
	%\index{#1}%
	\textbf{.#1}%
}




\newcommand{\option}[1]{%
	\mbox{\textbf{#1}}%
}


\newcommand{\revision}[1]{%
	\marginpar{Rev : #1}%
}
\newcommand{\packagenoidx}[1]{% le style des packages, sans indexation
	\emph{#1}%
}

\newcommandx{\package}[2][1]{%
	% Premier argument optionel = la manière dont doit être indexé
	\packagenoidx{#2}%
	\ifthenelse{\equal{#1}{}}{\index[pkg]{#2}}{\index[pkg]{#1}}%		
}

\newcommandx{\renvoi}[2][2]{ (☞\,p.~\pageref{#1}, \textbf{\ref{#1}}%Eventuellement deuxième arg opt
	\ifthenelse{\equal{#2}{}}{}{; ☞\,p.~\pageref{#2}, \textbf{\ref{#2}}}%
	)%
}
% Même chose que précédemment, mais sans étoile.
\newcommandx{\renvoin}[2][2]{ ☞\,p.~\pageref{#1}, \textbf{\ref{#1}}%Eventuellement deuxième arg opt
	\ifthenelse{\equal{#2}{}}{}{; ☞\,p.~\pageref{#2}, \textbf{\ref{#2}}}%
	%
}
\newcommand{\rev}[1]{\marginpar{#1}}

\let\oldinputminted\inputminted
\renewcommand{\inputminted}[1]{%
\oldinputminted[linenos,frame=leftline,framesep=0.5em]{latex}{#1}%
}
\newminted{latex}{linenos,frame=leftline,framesep=0.5em}
\newminted{bash}{frame=leftline,framesep=0.5em}



\newcommand{\type}[1]{%
	@#1%	
}

%pour la liste de logiciels

\newcommand{\logiciel}[3]{%
	% #1 => nom du logiciel
	% #2 => Plateforme
	% #3 => Commentaires
	\subsection{#1 (#2)}
	
	#3
	
	
}
\newcommand{\bibmacro}[1]{%
	\begin{english}\textbf{#1}\end{english}%
}
\newcommand{\compteur}[1]{%
	\textbf{#1}%
}

% Logo de la famille TeX, Inspiré de xetex-bidi
\def\reflect#1{{\setbox0=\hbox{#1}\rlap{\kern0.5\wd0
  \special{x:gsave}\special{x:scale -1 1}}\box0 \special{x:grestore}}}

\let\oldTeX\TeX
\renewcommand{\TeX}[0]{\oldTeX\xspace}
\def\logo{{\leavevmode$\smash{\hbox{(X\lower.5ex
  \hbox{\kern-.125em\reflect{E})\,}\kern-.1667em \LaTeX}}$}\xspace}
\newcommand{\XeTeX}{{\leavevmode$\smash{\hbox{X\lower.5ex
  \hbox{\kern-.125em\reflect{E}}\kern-.1667em \oldTeX}}$}\xspace}
\newcommand{\XeLaTeX}{{\leavevmode$\smash{\hbox{X\lower.5ex
  \hbox{\kern-.125em\reflect{E}}\kern-.1667em \LaTeX}}$}\xspace}

% Pour gérer les basculements dans les styles biblio
\newcommand{\bibverbose}{
\renewbibmacro*{cite}{%
	 \global\togglefalse{cbx:fullcite}%
  	\global\togglefalse{cbx:loccit}%
	\usebibmacro{cite:citepages}%
	\usebibmacro{cite:full}%
      	\usebibmacro{cite:save}%
}
}
\newcommand{\bibverbosetrad}{
\renewbibmacro*{cite:title}{%
  \printtext[bibhyperlink]{%
    \printfield[citetitle]{labeltitle}%
    \setunit{\nametitledelim}%
    \bibstring[\mkibid]{opcit}}}
\renewbibmacro*{cite}{%
  \usebibmacro{cite:citepages}%
  \global\togglefalse{cbx:fullcite}%
  \global\togglefalse{cbx:loccit}%
  \bibhypertarget{cite\the\value{instcount}}{%
    \ifciteseen
      {\iffieldundef{shorthand}
         {\ifciteibid
            {\usebibmacro{cite:ibid}}
            {\ifthenelse{\ifciteidem\AND\NOT\boolean{cbx:noidem}}
               {\usebibmacro{cite:idem}}
               {\usebibmacro{cite:name}}%
             \usebibmacro{cite:title}}%
	  \usebibmacro{cite:save}}
         {\usebibmacro{cite:shorthand}}}
      {\usebibmacro{cite:full}%
       \usebibmacro{cite:save}}}}
}
