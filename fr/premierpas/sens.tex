\chapter{Mettre en sens son document (1) : premiers pas}

\begin{intro}
Nous allons maintenant voir comment \emph{mettre en sens} notre document, c'est-à-dire comment poser des balises, des repères, pour marquer le \enquote{relief} sémantique du texte.
\end{intro}

\section{Mettre en forme n'est pas mettre en sens}\label{sensforme}

Lorsque nous lisons un livre, tous les éléments ne sont pas présentés de la même manière : certains sont en gras, d'autres en italique, en souligné, en couleur, etc. 

Tout ceci constitue la \emph{mise en forme} du texte. Si notre livre est bien conçu, ces changements de forme renvoient à des changements de signification : l'italique peut indiquer un titre d'ouvrage ou bien une citation ou une simple insistance, le gras peut indiquer une notion ou une définition ou toute autre signification.

On le voit, la \emph{mise en forme} diffère de la \emph{mise en sens}. Cette dernière est idéalement faite par l'auteur du travail, tandis que l'éditeur s'occupe normalement de la mise en forme et de la mise en page.

C'est d'ailleurs ce qui se passait auparavant quand les auteurs proposaient encore des textes manuscrits à leurs éditeurs : ils indiquaient les éléments à mettre en sens par des signes, mise en sens que l'éditeur transformait en mise en forme\footnote{Dans un traitement de texte de type WYSIWYG, cette distinction se fait normalement à l'aide des styles. Bien souvent malheureusement les utilisateurs ne savent pas s'en servir.}.

Dans \LaTeX, le principe est le même : il existe des commandes de mise en sens qui sont ensuite transformées en commandes de mise en forme. Mieux : on peut définir ses propres commandes de mise en sens. L'intérêt est  évident : pouvoir changer rapidement de mise en forme pour un ensemble de données mises en sens.

Un exemple sera plus parlant. Supposons que nous écrivions un livre d'introduction à l'histoire du christianisme antique. Ce livre cite divers auteurs. Nous souhaitons mettre en valeur ces auteurs, et pour ce faire décidons de les mettre en petites capitales.

Donc, à chaque fois que nous citons un auteur, nous indiquons que nous souhaitons avoir son nom en petite capitale.
Vient le moment où nous imprimons notre livre, et nous nous rendons compte que le choix des petites capitales n'est pas le plus pertinent, mais qu'il vaudrait mieux mettre du gras. Il ne nous reste alors plus qu'à repérer toutes les petites capitales dans notre texte, à vérifier qu'il s'agit bien de petites capitales indiquant un nom d'auteur et à les remplacer par du gras --- travail fastidieux !

En revanche, si au lieu de signaler à chaque occurrence qu'il faut des petites capitales, nous signalons simplement  qu'il s'agit d'un nom d'auteur --- par exemple en écrivant : \cs{auteur}\verb|{Tertullien}| --- nous n'aurons qu'une seule ligne à changer pour indiquer que nous souhaitons avoir les noms d'auteur en gras. Mieux : nous pourrons créer très simplement un index des auteurs\renvoi{indexauteur}.

LaTeX propose quelques commandes simples de mise en sens  : par exemple celles que nous avons vues plus haut pour indiquer les niveaux de titres\renvoi{niveautitre}.

Nous allons ici présenter quelques autres commandes et environnements de mise en sens. Dans le chapitre suivant, nous en indiquerons des spécifiques aux citations\renvoi{citertexte}. Dans un troisième chapitre, nous expliquerons comment créer ses propres commandes\renvoi{creercommandes}, et nous présenterons alors la manière de mettre en forme.

\section{Commandes de mise en sens}

\subsection{Mise en valeur d'un texte}

On peut ponctuellement vouloir mettre en valeur un morceau de son écrit. Pour ce faire il existe la commande \cs{emph}\marg{texte en emphase}.
Exemple :

\begin{latexcode}
On peut se demander si des textes apocryphes 
ont été non seulement \emph{utilisés} mais aussi \emph{lus}
dans la liturgie africaine.
\end{latexcode}

Concrètement cela se traduit par un italique. 

\begin{quotation}
On peut se demander si des textes apocryphes 
ont été non seulement \emph{utilisés} mais aussi \emph{lus} dans la liturgie africaine.
\end{quotation}

Toutefois, à la différence d'une commande qui indiquerait directement de mettre le texte en italique, cette commande pourrait, si on voulait, donner un résultat différent, par exemple mettre en couleur. 

Une autre propriété intéressante est la gestion des imbrications : par défaut une commande \cs{emph} à l'intérieur d'une autre commande \cs{emph} produit un texte en caractères droits.

\subsection{Le paratexte : notes de bas de page et de marge}

\LaTeX propose deux commandes pour indiquer des paratextes\footnote{La question des apparats critiques mise à part, question que nous traiterons plus loin\renvoi{eledmac}.} : pour des notes de bas de page et des notes de marge (la position de ces dernières changeant, quand on est en recto-verso, selon que la page est paire ou impaire). Ces commandes sont, respectivement, \csp{footnote} et \csp{marginpar}.

\begin{latexcode}
Lorem\footnote{Une note de bas de page.} ipsum dolor amat.
Aliquam sagittis\marginpar{Annotation marginale} magna.
\end{latexcode}

\begin{attention}
    On serait tenté d'utiliser cette commande pour citer en note de bas de page une référence bibliographique. Il existe en fait une commande spécifique, que nous étudierons en temps voulu\renvoi{footcite}.
\end{attention}
\begin{attention}
    Certaines mauvaises langues diront qu'il s'agit ici d'une mise en forme et non pas d'une mise en sens. Ils ont partiellement raison, dans la mesure où parfois distinguer la mise en forme de la mise en sens n'est pas évident.
    
    Les personnes vraiment perfectionnistes pourront définir leurs propres commandes pour différencier les différents sens d'une note de marge ou de bas de page.
\end{attention}

\begin{plusloins}
    Certains préfèrent mettre des notes de fin de texte. Bien que nous n'approuvons guère ce choix, nous signalons qu'il est possible d'en produire à l'aide du package \package{endnotes}.
\end{plusloins}

\subsection{Listes}

\LaTeX propose trois types de listes : les listes numérotées, les listes non-numérotées et les listes de description.

\subsubsection{Les listes numérotées}

Une liste numérotée est un environnement \enviro{enumerate}.
Chaque élément de la liste est marqué par la commande \csp{item}.

\begin{latexcode}
\begin{enumerate}
    \item Premier élément
    \item Deuxième élément
    \item Troisième élément
\end{enumerate}
\end{latexcode}

\begin{quotation*}
\begin{enumerate}
    \item Premier élément
    \item Deuxième élément
    \item Troisième élément
\end{enumerate}
\end{quotation*}

\begin{plusloins}
Il existe un package \package{etaremune} proposant l'environnement  \enviro{etaremune} pour obtenir une liste numérotée à l'envers, avec le plus grand numéro en début de liste.

\end{plusloins}
\subsubsection{Les listes non-numérotées}

Une liste non-numérotée est un environnement \enviro{itemize}.
Chaque élément de la liste est marqué par la commande \cs{item}.

\begin{latexcode}
\begin{itemize}
    \item Un élément
    \item Un autre
    \item Encore un autre
\end{itemize}
\end{latexcode}

\begin{quotation*}
\begin{itemize}
    \item Un élément
    \item Un autre
    \item Encore un autre
\end{itemize}
\end{quotation*}

\subsubsection{Les listes de descriptions}

Une liste de descriptions fait correspondre une à une des valeurs. Une telle liste peut être utile pour des lexiques, des glossaires, des chronologies, etc. Pour chaque couple, la première valeur est passée comme argument à la commande \cs{item}. Les listes de définitions sont des environnements \enviro{description}.


\begin{latexcode}
\begin{description}
    \item[325]Concile de Nicée.
    \item[381]Concile de Constantinople.
    \item[431]Concile d'Éphèse.
\end{description}
\end{latexcode}

\begin{quotation*}
\begin{description}
    \item[325]Concile de Nicée.
    \item[381]Concile de Constantinople.
    \item[431]Concile d'Éphèse.
\end{description}
\end{quotation*}

\subsection{Imbrication des listes}

Il est possible d'imbriquer des listes, quels que soient leurs types. On ne peut, par défaut, avoir plus de quatre niveaux d'imbrication.

\begin{latexcode}
\begin{itemize}
    \item Un élément de premier niveau
    \begin{enumerate}
            \item Premier sous élément
            \item Second sous élément
    \end{enumerate}
    \item Un autre élément de premier niveau
\end{itemize}
\end{latexcode}

\begin{quotation*}
\begin{itemize}
    \item Un élément de premier niveau
    \begin{enumerate}
            \item Premier sous élément
            \item Second sous élément
    \end{enumerate}
    \item Un autre élément de premier niveau
\end{itemize}
\end{quotation*}

\begin{plusloins}
On a parfois besoin de personnaliser l'aspect des listes, ou bien encore d'arrêter une numérotation de liste pour la reprendre plus loin. Bertrand Masson a écrit un excellent tutoriel sur la manière d'utiliser le package \package{enumitem} pour arriver à ces fins\footcite{bebert_liste}. 
\end{plusloins}
