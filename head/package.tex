\usepackage{fontspec}
\usepackage{xunicode}
\usepackage{polyglossia}
%\usepackage[french]{babel}
\setmainlanguage{french}
\setotherlanguage{english}
\setotherlanguage[variant=ancient]{greek}
\usepackage{minted}
\usepackage[citestyle=verbose-trad2,bibstyle=verbose,backend=biber]{biblatex}
%\listingscaption{Code}
\usepackage{fontspec}
\setmainfont[Mapping=tex-text,Ligatures=Common]{Linux Libertine O}
\usepackage{placeins}%Pour que les flottants ne flottent pas trop
\usepackage{ifthen}
\usepackage{xargs}
\usepackage{xspace}
\usepackage{xcolor}
\usepackage{csquotes}
\usepackage{longtable}
\usepackage{hyperref}
\hypersetup{colorlinks=true, citecolor=black, filecolor=black, linkcolor=black, urlcolor=black, bookmarks=true,pdftitle={XeLaTeX pour les sciences humaines}, pdfauthor={Maïeul ROUQUETTE}, pdfcreator={PdfLaTeX}}
%\usepackage[landscape,a4paper]{geometry}
\usepackage[strict]{changepage}
\usepackage{tikz}
\usepackage{array}
\usepackage{multirow}
\usepackage{bidi} % nécessaire pour la command \XeLaTeX
\usepackage{csvtools}

%Appel des fichiers .bib
\bibliography{biblio_fichiers/biblio_exemple}		 % Dans ce fichier, stocker les bibliographies d'exemples qui sont simplement  cités, sans que soit affiché le code .bib
\bibliography{exemples/biblio/premierpas/crossref} % Pour expliquer crossref
\bibliography{exemples/biblio/premierpas/augustin} % Pour expliquer l'intérêt des crossref dans la gestion des division
\bibliography{exemples/biblio/premierpas/urner}	% Un premier exemple de bibliographie : l'entrée Urner
\bibliography{exemples/biblio/premierpas/augustin_editeur} % Pour expliquer la fusion des auteurs et des éditeurs
\bibliography{exemples/biblio/premierpas/noms} % Pour expliquer la syntaxe des noms
\bibliography{exemples/biblio/premierpas/saxer} % Pour expliquer la syntaxe des titres
\bibliography{exemples/biblio/premierpas/felix} % Pour expliquer la syntaxe des titres
\bibliography{exemples/biblio/premierpas/junod} % Pour expliquer la syntaxe des titres
\bibliography{exemples/biblio/premierpas/histoire} % Pour expliquer la syntaxe des titres