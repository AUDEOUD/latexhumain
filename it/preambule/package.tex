\usepackage{fontspec}
\usepackage{xunicode}
\usepackage{polyglossia}
\setmainlanguage{french}
\setotherlanguage{english}
\setotherlanguage[variant=ancient]{greek}
\usepackage{minted}				%Pour les exemples de code colorés

\usepackage[singletitle=true,citestyle=verbose-trad2,bibstyle=verbose,backend=biber,citepages=omit]{biblatex}


\usepackage{placeins}				%Pour que les flottants ne flottent pas trop
\usepackage{verse}					%Pour les vers
\usepackage{ifthen}				% Pour des test conditionnels
\usepackage{xargs}				% Pour des arguments optionnels
\usepackage{xspace}				% Pour la gestion des espaces
\usepackage{xcolor}				% Pour les couleurs
\usepackage{csquotes}				% Pour les guillemets
\usepackage{longtable}				% Pour les tableaux longs
\usepackage{tikz}					% Pour les dessins
\usetikzlibrary{shapes}
\usepackage{array}					% Pour les types de colonne sup.
\usepackage{multirow}				% Pour les fusion de lignes
\usepackage[nobreak=true,backgroundcolor=shadecolor,linecolor=white,innertopmargin=0.5\baselineskip,innerbottommargin=0.5\baselineskip]{mdframed}		% Pour les fonds grisés
\usepackage{framed}
\usepackage{fancyhdr}				% Pour les entetes et pied
\usepackage{graphicx}				% Pour insérer des images
\usepackage{subscript}				% Pour insérer des textes en index

\usepackage{bibleref-french}			% Pour les citations bibliques
\usepackage{enumitem}				% Pour les styles de listes

\usepackage{eledmac, eledpar}			% Pour l'édition critiques
\usepackage{multicol}				%Pour une gestion avancées de plusieurs colonnes, pour l'index. nécéssaire d'appeler  avant le package bidi, sinon certaines options de imakeidx ne fonctionnent pas
\usepackage{totpages}				% Pour le nb total de pages
\usepackage[splitindex,noautomatic]{imakeidx}		%Pour plusieurs index. Il est nécéssaire de passer par splitindex, car trop de donnée à indexer.
\indexsetup{level=\section,noclearpage}
\makeindex[title=Commandes,name=cs]
\makeindex[title=Classes,name=classe]
\makeindex[title=Champs bibliographiques,name=champ]
\makeindex[title=Environnements,name=enviro]
\makeindex[title=Packages,name=pkg]

\usepackage{hyperref}				% Pour les liens hypertextes
\hypersetup{colorlinks=true, citecolor=black, filecolor=black, linkcolor=black, urlcolor=black, bookmarks=true,pdftitle={(Xe)LaTeX applicato alle scienze umane}, pdfauthor={Maïeul ROUQUETTE con la collaborazione di Brendan CHABANNES ed Enimie ROUQUETTE. Versione italiana a cura di Giacomo LANZA}, pdfcreator={XeLaTeX}}
