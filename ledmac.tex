\chapter{Édition critique et bilingue : les packages \package{ledmac} et \package{ledpar}}

\section{Faire une édition critique avec \package{ledmac}}

\subsection{Numéroter un texte}

\cs{beginnumbering} 
\cs{endnumbering}
>encadre le texte à numéroter

\cs{pstart}
\cs{pend}
> début et fin de la numérotation. A l'intérieur de begin et endnumbering,  on peut interrompre puis reprendre là numérotation où on l'avait interrompu : seul les passage entre pstart et pend sont numérotés.

la do dit qu'il faut à chaque paragraphe remettere pstart/pend , et qu'on peut empêcher cette inconvénient en créant un groupe et avec la commande autopar. Pourtant marche très bien sur plusieurs paragarphes à la suit sans passer par cette commande.. à voir!


Ledmac numérote par défaut toutes les 5 lignes. On eut changer cela:
\cs{firstlinenum}\marg{nbre} et \cs{linenumincrement}\marg{nbre}.
firstlinenum: la première ligne à être numérotée ; linenumincrement: définit la fréquence avec laquelle les lignes seront numérotés. ex: \cs{linenumincrement}\marg{1} \cs{firstlinenum}\marg{1} : toutes les lignes seront numérotées en commençant par la première. 

Un moyen pour ne pas encombrer la mémoire qd texte trop long : faire de plus petites sections numérotées. Mais si on met endnumbering ,  avec beginnumbering on recommencera la numérotation à zéro.  Utiliser à la place pausenumbering - resumenumbering > avec resume numbering la nmérotation reprend où elle s'était arrêté. Si on utilise ces commande essentiellement pour des question de mémoire conseil de faire une commande exprès: \verb|\newcommand{\memorybreak}{\pausenumbering\resumenumbering}|> plus qu'à mettre memorybreak de temps en temps dans le texte ..

\cs{lineation}\marg{arg} > \arg{page} > numérotation recommence à chaque page \arg{section} (par défut) : numérotation recommence à chaque section.
Possible aussi de changer l'endroit où apparit le numéro de page (par défaut ds la marge de gauche): \cs{linenummargin}\marg{arg}, dt les arguments peuvent être left, right, inner, ou outer

\cs{startsub} \cs{endsub} > ligne numérotés 10.1, 10.2, 10.3,.. et non 10, 11, 12 
\setline{⟨num⟩} and \advanceline{⟨num⟩} > modifier le numéro d'une ligne 
\cs{skipnumbering} > ne pas prendre en compte dans la numérotation telle ligne
On peut aussi changer façon dont apparaissent es numéro des lignes (voir doc)


\subsection{L'apparat critique}



\section{Mettre deux textes en vis-à-vis: le \package{ledpar}}
\prealable{Le package \package{ledpar}, qui sert à mettre un texte et sa traduction en vis-à-vis, fonctionne avec \package{ledmac}}