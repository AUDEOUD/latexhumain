
\chapter{Faire un index}


\section{Faire un index simple avec \package{MakeIndex}}


\subsection{Principe de base}
\begin{prealable}

Le package se charge avec \verb|\usepackage{makeidx}|. Il faut ensuite placer dans le préambule la commande \commande{makeindex} pour que \LaTeX{}  puisse créer un index.

\end{prealable}



\subsubsection{Indexer son document}


On indexe son document avec  la commande \verb+\index{+\emph{entrée}\verb+}+. L'entrée apparaîtra dans l'index sous la forme indiquée par l'argument \emph{entrée}, suivit du numéro de la page où cette commande est placée dans le texte. 

Dans l'exemple qui suit, l'index comportera ainsi cinq entrées: \enquote{Charlemagne}, \enquote{Adrien}, \enquote{Tassilon}, \enquote{Formose} et \enquote{Damase}:
%Changer l'exemple.
\begin{listing}[ht]
\begin{minted}[linenos]{latex}

Tandis que Charle\index{Charlemagne} était à Rome, il convint
avec le pape Adrien\index{Adrien} qu’ils enverraient de concert
des ambassadeurs à Tassilon, duc de Bavière\index{Tassilon}(...)
Les hommes choisis et envoyés dans cette ambassade furent, de 
la part du pape, les évêques Formose\index{Formose} et 
Damase\index{Damase}(…).

\end{minted}
\caption{Indexer son texte}
\end{listing}
%Eginahrd, Annales

 Lorsqu'une entrée est référencée deux fois dans la même page, cette page n'est indiquée qu'une seule fois. 

\begin{attention}
Il vaut mieux accoler la commande \verb+\index{entrée}+ directement au  mot à indexer, sans laisser d'espace, pour éviter toute ambiguité en cas de changement de page.

\end{attention}

%% Ne faudrait-il pas créer un environnement mettre "Remarque"? qui irait mieux que anecdote par ex pour certains endroits plutôt que "Anecdote" ou attention? Ici, par exemple( ou ds le chapitre notion de base, sur babel)
L'indexation d'un texte n'est pas automatique: il faut placer \verb+\index+ à chaque endroit  que l'on veut référencer. Pour éviter d'oublier, dans cette démarche fastidieuse, de référencer certaines occurences, on  peut bien sûr créer une commande  spécifique.

Ainsi, pour indexer automatiquement tous les noms propres d'un texte, déclarons la commande suivante:\\
 \verb+\newcommand\auteur[2]{#1~\textsc{#2}\index{#2, #1\xspace}}+\\
Il suffira ensuite, au cours de la rédaction de son texte, de taper par exemple \verb|\auteur{Victor}{Hugo}| pour obtenir \enquote{Victor \textsc{Hugo}} dans le cours de son texte, indexé sous l'entrée \enquote{Hugo, Victor}.


\subsubsection{Générer l'index}


 Lors de la première compilation, la commande \commande{\makeindex} indique à \LaTeX{} de créer un fichier de type \fichier{exemple.idx}, contenant la liste de toutes les entrées que l'on a placées. Chaque entrée se présente de la manière suivante:\\
\verb+\indexentry{nom de l'entrée}{numéro de la page}+.



On compile ensuite le document en tapant dans le terminal \\ \verb+\makeindex exemple.idx + \\
-il est possible de ne pas indiquer l'extension .idx. \LaTeX{} génère alors un fichier \fichier{exemple.ilg} qui contient les message de compilation de l'index, et un fichier \fichier{exemple.ind} qui contient l'index formaté et se présente ainsi:


\begin{verbatim}
\begin{theindex}

  \item entrée,4
  \item autre entrée, 5--7
  \item troisième entrée, 3, 8	

\end{theindex} 

\end{verbatim}

Il suffit alors de compiler à nouveau son document pour y intégrer son index, qui apparaîtra à l'emplacement que l'on aura indiqué par la commande  \verb|\printindex |.


%essayer de faire marcher showidx

\subsection{Aller plus loin}
\subsubsection{Créer des subdivisions}

Il est possible, avec \package{MakeIndex},  de créer des subdivisions et des subsubdivisons  pour chaque entrée de l'index. La subdivision se crée de la manière suivante: \verb+\index{entrée!sous-entrée}+. La sous-sous-entrée, logiquement, se crée ainsi: \verb+\index{entrée!sous-entrée!sous-sous-entrée}+. \package{MakeIndex} ne permet cependant que ces trois niveaux d'indexation.

On peut ainsi référencer autrement notre premier exemple, en créant les entrées \enquote{Evêques} et \enquote{Ducs} que l'on subdivise:

\begin{minted}[linenos]{latex}

Tandis que Charle\index{Charlemagne} était à Rome, il convint 
avec le pape Adrien\index{Adrien} qu’ils enverraient de concert 
des ambassadeurs à Tassilon, duc de Bavière\index{Ducs!Tassilon}.
Les hommes choisis et envoyés dans cette ambassade furent, de la 
part du pape, les évêques Formose\index{Evêques!Formose}
et Damase\index{Evêques!Damase}(…).

\end{minted}

Supposons que ce texte soit à la page 5; on obtiendra ainsi dans l'index:
\begin{tabbing}
\hspace{0,5cm}  \= \kill
Adrien, 5 \\
\\
Charlemagne, 5 \\
\\
Ducs \\
\> Tassilon, 5\\
\\
Evêques \\
\> Damase, 5\\
\> Formose, 5\\

\end{tabbing}




Bien entendu, on pourrait encore rajouter une subdivision, en distinguant par exemple \enquote{Clercs} et \enquote{Laïcs}, et dans la première catégorie en distinguant \enquote{Evêques} et \enquote{Papes}. 
%Montrer le résultat?

 
\subsubsection{Faire des références croisées}

Pour qu'une entrée dans l'index renvoie à une autre entrée, on utilise la commande  \verb+\index{entrée|see{entrée à laquelle on renvoie}}+. Ainsi, la commande \verb+\index{Tassilon|see{Ducs}+ donnera dans l'index:

Tassilon, voir Ducs\\
La traduction de \emph{see} change selon la langue indiquée comme \verb+ \setmainlanguage+


%Si on met  \index{nom propre|see{nom générique}} dans le corps du texte, marche sans problème. En revanche si on veut semi automatiser l'indexation pour certains nom  en définissant une nouvelle commandande  type \newcommand\Nom{\index{nom propre|see{nom générique}}Nom\xspace}, on va avoir dans l'index: nom propre, voir nom générique, voir nom générique, voir nom générique,.. (répété autant de fois que \Nom apparaît ds le document. 
% Bidouillage qui marche: définir la commande \newcommande\Nom{\index{nom générique}Nom\xspace} puis ailleurs une seule fois indiquer \index{nom propre|see{nom générique}.
%Pas sûre que ce soit utile de l'indiquer: si on emploi la commande |see c'est pour des cas particuliers: on ne va pas en faire une commande automatique.

  

\subsubsection{Créer des entrées sur plusieurs pages}

Si l'on veut référencer dans l'index non pas un mot mais un passage, il faut placer la commande \index{entrée|(} au début du passage à indexer et la commande  \index{entrée|)} à la fin. Si le passage commence à la page x et se termine à la page y, on aura ainsi indiqué dans l'index: \\
entrée x-y


\subsubsection{Entrées formatées}

Makeindex ne gère pas correctement les accents indiquées dans la commande \verb+\index{...}+ : il classe les mots commançant par un accent à la fin de l'index. La commande \verb+\index{entrée@entrée formatée}+ permet de résoudre ce problème. Ainsi, si l'on veut créer une entrée \enquote{écrivains}, qui soit triée à \enquote{e}, il faut taper \verb+\index{ecrivains@écrivains}+

La commande \verb+ \index{entrée@entrée formatée}+ permet donc de classer une entrée là où l'on veut dans l'index. Pour faire apparaître dans l'index, par exemple, les empereurs romains dans l'ordre chronologique et non dans l'ordre alphabétique, on peut créer les commandes suivantes:\\
\verb+\index{Empereurs!empereur1@Auguste}+,\\ \verb+\index{Empereurs!empereur2@Tibère}+\\
\verb+\index{Empereurs!empereur3@Claude}+\\
et ainsi de suite. On obtiendra alors
\begin{tabbing}
\hspace{0,5cm} \= \kill
Empereurs\\
\> Auguste, x\\
\> Tibère, z\\
\> Claude, y \\

\end{tabbing}

Cette commande est aussi utile pour mettre en évidence une entrée dans l'index en modifiant son aspect. En mettant \verb+\index{entrée@\textbf{entrée formatée}+, l'entrée apparaît en gras dans l'index. Ceci est valable pour toutes les commandes modifiant la fonte. Modifions ainsi comme suit la commande \verb+\auteur+ que l'on a crée:\\
\verb+\newcommand\auteur[2]{#1~\textsc{#2}\index{#2 #1@\textsc{#2}, #1\xspace}}+\\
Désormais, taper \verb|\auteur{Victor}{Hugo}| permettra d'obtenir \textsc{Hugo},~Victor dans l'index.
 
 
%To put a !, @, or | character in an index entry, quote it by preceding the character with a ". More precisely, any a character is said to be quoted if it follows an unquoted " that is not part of a \" command. A quoted !, @, or | character is treated like an ordinary character rather than having its usual meaning. The " preceding a quoted character is deleted before the entries are alphabetized. 
 
\subsubsection{Formater le numéro des pages}

Il peut arriver, lorsqu'une entrée est très souvent représentée dans un texte indexé, que l'on veuille mettre en valeur une de ses occurences, en faisant apparaître dans l'index le numéro de la page où se trouve cette occurence en gras: on utilise alors la commande \verb+\index{entrée|textbf}+. De même, pour faire apparître le numéro de la page en italique va-t-on utiliser la commande \verb|\index{entrée|textit}|…

\begin{attention}
il s'agit bien de \verb+ |textbf +, non de \verb+ |\textbf +

\end{attention}



%+ protéger un index fragile, "customiser" son index
%Recall that special characters like \ may appear in the argument of an \index command only if that command is not itself contained in the argument of another command. This is most likely to be a problem when indexing items in a footnote. Even in this case, robust commands can be placed in the “@” part of an entry, as in \index{gnu@{\it gnu}}, and fragile commands can be used if protected with the \protect command.3

\subsection{Formater son index}

La mise en forme par défaut de l'index n'est pas très satisfaisante. On peut heureusement la modifier pour rendre l'index plus lisible.


\subsubsection{Comment faire}
Les commandes pour formater l'index sont à placer dans un fichier \fichier{monstyle.ist} que vous devez créer vous-même.

\begin{attention}
Pensez à placer ce fichier à un endroit où \LaTeX peut le trouver, par exemple dans le dossier courant 

(est-ce qu'on explique qq part comment indiquer un chemin ds le terminal? ds le chap sur les packages?)
\end{attention}

Il faut ensuite taper dans le terminal :\\
\verb+ makeindex -s monformatage.ist monfichier.idx +\\
La commande -s indique à \package{makeindex}  qu'il doit utiliser le fichier de style \fichier{monstyle.ist} .

\subsubsection{Quelques commandes}

Je vais donner ici quelques exemples de commandes que l'on peut mettre dans son fichier \fichier{monstyle.ist}. 

\begin{attention}
Dans le fichier de style\ext{ist}, on n'écrit pas en \LaTeX{} mais en \TeX .Les commandes sont ainsi encadrées par des guillemets " . La barre contre-oblique, quant à elle, n'est prise en compte que si elle est précédée d'une autre contre-oblique.
\end{attention}

Le manuel de \package{makeindex} que l'on peut lire en tapant dans le terminal \verb+ man makeindex + nous indique que, pour modifier ce qui est inséré entre le premier niveau d'item (l'entrée) - ou niveau 0 -  et le ou les numéros de pages, on utilise l'option\verb| delim_0 +; de même, pour définir ce qui est inséré entre le deuxième niveau d'item (la sous-entrée) - ou niveau 1, on utilise l'option\verb+ delim_1 +, et l'obtion \verb+ delim_2 + pour les sous-sous-entrées - ou niveau 2 d'itemisation. Le choix par défaut est une virgule suivit d'un blanc : \verb+ ", " +

Ainsi, si je veux que le numéro de la page soit justifié à droite, et non simplement séparé de l'entrée par une virgule et un blanc, je dois indiquer dans mon fichier de style les lignes suivantes:

\begin{verbatim}
delim_0 "\\hfill"
delim_1 "\\hfill"
delim_2 "\\hfill"
\end{verbatim}

Si je préfère une ligne de points, il faudra logiquement remplacer \verb|"\\hfill"| par \verb+ "\\dotfill" +


Si l'on continue la lecture du manuel, on apprend aussi comment faire apparaître \enquote{ff.} (ou ce que l'on voudra) quand une même entrée est indexée dans trois pages ou plus à la suite. Il suffit d'indiquer \verb+ suffix_3p "~ff." +
\\

Voyons maintenant comment l'on peut insérer les lettres de l'alaphabet entre les groupes d'entrées. Cette manipulation permet de rendre plus lisible un index un peu long. 

Le manuel indique que l'option par défaut \verb+ headings_flag 0 + ne met pas de séparateurs entre les groupes, que l'option \verb+ headings_flag 1 + permet d'obtenir des lettres majuscules comme séparateur, et l'obtion  \verb+ headings_flag -1 + des lettres minuscules.

Définissons ensuite la manière dont vont apparaître ces lettres. On utilise la commande \verb| heading_prefix  |. Supposons que je veuille faire apparaître ces lettres-séparatrices en gras. Je n'oublie pas que l'on code en \TeX … J'écris donc:\\
\verb| heading_prefix "\\bf" | . Si je veux qu'elles apparaissent en italique, j'écris \verb| "\\it" |… (on utilise \verb+ \bf + et \verb+ \it + , qui sont des commandes \TeX, et non \verb+ \textbf + et \verb+ \textit +).
Je peux de même changer la taille des lettres-séparatices en ajoutant \verb| \\large | ou \verb| \\Large|, et ainsi de suite. 

Mais j'ai ouvert une accolade… il me faut la refermer:\\
\verb| heading_suffix "}\\nopagebreak\n" |

\verb| \\nopagebreak| évite qu'une lettre-séparatrice se retrouve seule en fin de page. 

%Est-ce que le \n est nécessaire? Pas trop sûre... vérifier, essayer de l'expliquer.

Ce n'est pas fini. Mes lettres sont actuellement justifiées à gauche, comme les entrées de l'index. Je voudrais qu'elles soient centrées: \\

\begin{verbatim}
headings_flag 1
heading_prefix "\\bf\\large\\hfill" 
heading_suffix "hfill}\\nopagebreak\n" 
\end{verbatim}
Les deux \verb| \\hfill | étirent, pour le premier, l'espace avant la lettre vers la droite, et pour le second l'expace suivant la lettre vers la gauche -- voilà mes lettres centrées.

%faire un envoi sur la différence entre \hfill et \hfil? Sur les longueur ebsolues et relatives?
Si je préfère qu'elles soient simplement légèrement décalées vers la droite, je peux utiliser, à la place de \verb| \\hfill |, \verb| \\hspace*{-1ex} |  qui rajoute à gauche un espace de la taille d'un -x. 


Quand on a compris ce système, il est alors assez simple de formater à sa guise l'index. C'est pourquoi, pour les autres options, je renvoie ici à la lecture du manuel.



%man makeindex: indique les différentes obtions. En détailler quelques unes? ex: aligner les numéros des pages au bout de la ligne (avec ou sans points dans l'espace blanc); Ajouter les lettres , etc
%heading_prefix "{\\bfseries\\hfil "
%heading_suffix "\\hfil}\\nopagebreak\n" >un seul { car on est en Tex. Mais latex trouve que manque une accolade > la refermer la fois d'après.

\section{\package{SplitIndex}, ou comment faire plusieurs index}

\begin{prealable}
Nous supposons que vous avez appris à vous servir de \package{MakeIndex} avant d'apprendre à utiliser \package{SplitIndex}, car les principes de base sont les mêmes.
\end{prealable}

