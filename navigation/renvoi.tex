\chapter{Renvois internes}\label{label}

\begin{intro}
Dans ce court chapitre nous allons nous intéresser à la manière de faire des renvois à l'intérieur d'un document.
Il s'agit de permettre d'afficher des textes du type : \enquote{Nous renvoyons au chapitre~N. page~P.}
\end{intro}

\section{Étiqueter des  emplacements : \cs{label}}

Pour pouvoir faire des renvois internes, il est nécessaire de placer des \enquote{étiquettes} aux endroits vers lesquels on souhaite renvoyer.
Cet étiquetage  se fait avec la commande \csp{label}\marg{etiquette}.

\begin{plusloins}
Lors de la relecture, il peut être intéressant d'avoir les étiquettes affichées en marge. On utilisera pour ce faire le package \package{showlabels}.
\end{plusloins}

\section{Se servir des étiquettes}

Après avoir avoir placé des étiquettes, on peut y renvoyer. 
Il suffit d'insérer une commande de renvoi à l'endroit souhaité. 
Il est toutefois nécessaire de compiler deux fois avec \XeLaTeX.
À la première compilation,   \XeLaTeX
note les emplacements des étiquettes dans un fichier. À la seconde compilation, il lit ce fichier afin de procéder aux renvois. 

\begin{attention}
    Si vous modifiez votre texte, il faudra de nouveau compiler deux fois. En effet, les numéros de pages, de titres, de légendes, etc. peuvent avoir changé. Il faut donc  que \XeLaTeX{} les apprenne à nouveau. En résumé, pour avoir un document correct il faut :
    \begin{enumerate}
        \item Compiler avec \XeLaTeX{};
        \item Compiler avec  Biber afin d'incorporer les références bibliographiques;
        \item Compiler avec \XeLaTeX{};
        \item Re-compiler avec \XeLaTeX{}, car l'ajout des références bibliographiques a pu modifier le positionnement des étiquettes.
    \end{enumerate}
    
    Tout ceci peut paraître bien compliqué et risque d'entraîner des oublis. C'est pourquoi il est conseillé d'utiliser le programme latexmk dont nous parlons en annexe\renvoi{latexmk}.
\end{attention}

\subsection{Renvoyer à une page}

Pour renvoyer au numéro de page correspondant à l'étiquette \arg{etiquette}, on utilise la commande \csp{pageref}\marg{etiquette}.

\begin{latexcode}
blabla \label{etiquette}
…
Nous renvoyons à la page~\pageref{etiquette}.
\end{latexcode}

\subsection{Renvoyer à un numéro de section}

Pour renvoyer à un numéro de section, on utilise la commande \csp{ref}.

\begin{latexcode}
\section{Section} \label{etiquette}
…
Nous renvoyons à la section~\ref{etiquette}.
\end{latexcode}
 
\subsection{Renvoyer à un titre de section}\label{renvoititre}

Pour renvoyer à un titre de section, on utilise la commande \csp{nameref}.

\begin{latexcode}
\section{Section} \label{etiquette}
…
Nous renvoyons à la section  \enquote{\nameref{etiquette}}.
\end{latexcode}

Toutefois cette commande n'est pas disponible en standard : elle est proposée par le package \package{hyperref}. Il faut donc charger ce package dans le préambule. Nous documentons plus loin quelques fonctionnalités de ce package\renvoi{hyperref}.


\section[Où placer la commande \oldcs{label} ?]{Où placer la commande \cs{label} ?}

Pour le moment, nous n'avons vu que des renvois vers des sections, mais le système de renvois est beaucoup plus souple.

Une étiquette permet de renvoyer à tout élément numéroté, comme un titre, une note de bas de page, une légende de flottant. Elle peut aussi renvoyer à un endroit précis en indiquant la page.

\begin{itemize}
\item Si l'étiquette \cs{label} est placée \emph{dans}  une commande  d'élément numéroté, elle renvoie à cet élément. Par exemple, pour renvoyer à une figure\renvoi{legende} :

\begin{latexcode}
\begin{figure}[paramètre de placement]
    Insertion de la figure
    \caption{Légende\label{figure}}
\end{figure} 
…

Nous renvoyons à la figure~\ref{figure} située p.~\pageref{figure}.
\end{latexcode}

\item Si la commande est placée ailleurs, elle renvoie à la page courante et à la section courante\footnote{En réalité il est possible de la placer immédiatement après un élément numéroté pour y renvoyer, mais cela ne s'applique pas aux notes de bas de pages.}.
\end{itemize}

\section{Comment nommer ses étiquettes ?}

Vous êtes bien sûr libre de trouver votre propre système de dénomination des étiquettes. Toutefois il est conseillé d'avoir quelque chose de la forme : \begin{english}\meta{forme}:\meta{nom}\end{english}, où \arg{forme} désigne le type d'élément vers lequel on renvoie.

Exemples :

\begin{english}
\begin{latexcode}
\footnote{Blabla \label{note:nom}}
\section{Titre \label{section:nom}}
\end{latexcode}
\end{english}


