\usepackage{doc}
\def\cmd#1{\cs{\expandafter\cmd@to@cs\string#1}}
\def\cmd@to@cs#1#2{\char\number`#2\relax}
\def\cmd#1{\cs{\expandafter\cmd@to@cs\string#1}}
\providecommand\marg[1]{%
  {\ttfamily\char`\{}\meta{#1}{\ttfamily\char`\}}}
\providecommand\oarg[1]{%
  {\ttfamily[}\meta{#1}{\ttfamily]}}
\providecommand\parg[1]{%
  {\ttfamily(}\meta{#1}{\ttfamily)}}


\newindex[Classes]{classe}
\newindex[Champs bibliographiques]{champ}
\newindex[Environnements]{enviro}
\newindex[Packages]{pkg}
\newcommand*{\cf}[0]{\emph{cf}.}


\newcommand{\forme}[1]{\enquote{#1}}

\newcommand{\classenoidx}[1]{%
	\textbf{#1}%
}
\newcommand{\classe}[1]{%
	\classenoidx{#1}%	
	\sindex[classe]{#1}%
}
\newcommand{\champ}[1]{%
	\sindex[champ]{#1}%
	\textbf{#1}%
}

\newcommand{\fichier}[1]{%
	#1%
}
\newcommand{\environoidx}[1]{%
	\textbf{#1}%
}
\newcommand{\enviro}[1]{%
	\environoidx{#1}%
	\sindex[enviro]{#1}%
}



\newcommand*{\ext}[1]{%
	%\index{#1}%
	\textbf{.#1}%
}




\newcommand{\option}[1]{%
	\textbf{#1}%
}


\newcommand{\revision}[1]{%
	\marginpar{Rev : #1}%
}
\newcommand{\packagenoidx}[1]{% le style des packages, sans indexation
	\emph{#1}%
}

\newcommand{\package}[1]{%
	\packagenoidx{#1}%
	\sindex[pkg]{#1}%		
}

\newcommand{\renvoi}[1]{\marginpar{p. \pageref{#1}}}

\newcommand{\rev}[1]{\marginpar{#1}}
% Redefinir les commandes de minted
\let\oldinputminted\inputminted
\renewcommand{\inputminted}[1]{%
\oldinputminted[linenos]{latex}{#1}%
}





%pour la liste de logiciels

\newcommand{\logiciel}[3]{%
	% #1 => nom du logiciel
	% #2 => Plateforme
	% #3 => Commentaires
	\subsection{#1 (#2)}
	
	#3
	
	
}
\newcommand{\bibmacro}[1]{%
	\textenglish{\textbf{#1}}%
}
\newcommand{\compteur}[1]{%
	\textbf{#1}%
}

\let\oldlatex\LaTeX
\renewcommand{\LaTeX}[0]{%
	\oldlatex\xspace}
\let\oldtex\TeX
\renewcommand{\TeX}[0]{%
	\oldtex\xspace}

\let\oldxelatex\XeLaTeX
\renewcommand{\XeLaTeX}[0]{%
	\oldxelatex\xspace}