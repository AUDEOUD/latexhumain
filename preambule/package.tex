\usepackage{fontspec}
\usepackage{xunicode}
\usepackage{polyglossia}
%\usepackage[french]{babel}
\setmainlanguage{french}
\setotherlanguage{english}
\setotherlanguage[variant=ancient]{greek}
\usepackage{minted}

\usepackage{doc}
\usepackage[citestyle=verbose-trad2,bibstyle=verbose,backend=biber,citetracker=false,citepages=omit]{biblatex}
\setmainfont[Mapping=tex-text,Ligatures=Common]{Linux Libertine}
\setsansfont[Mapping=tex-text,Ligatures=Common]{Linux Biolinum}
\setmonofont[Scale=0.75]{DejaVu  Sans Mono}

\usepackage{placeins}%Pour que les flottants ne flottent pas trop
\usepackage{verse}
\usepackage{ltxdockit}
%\usepackage{btxdockit}	% Pour documenter les différents champs
\usepackage{ifthen}
\usepackage{xargs}
\usepackage{xspace}
\usepackage{xcolor}
\usepackage{csquotes}
\usepackage{longtable}
\usepackage{hyperref}
\hypersetup{colorlinks=true, citecolor=black, filecolor=black, linkcolor=black, urlcolor=black, bookmarks=true,pdftitle={XeLaTeX pour les sciences humaines}, pdfauthor={Maïeul ROUQUETTE}, pdfcreator={PdfLaTeX}}
%\usepackage[landscape,a4paper]{geometry}
\usepackage{tikz}
\usetikzlibrary{shapes}
\usepackage{array}
\usepackage{multirow}
\usepackage{bidi} % nécessaire pour la command \XeLaTeX
\usepackage{csvtools}
% Bibleref
\usepackage{bibleref-french}
\usepackage{enumitem}

\biblerefstyle{jerusalem}

%Appel des fichiers .bib
\bibliography{biblio_fichiers/latex}		% La bibliographie sur \LaTeX.

\bibliography{biblio_fichiers/biblio_exemple}		 % Dans ce fichier, stocker les bibliographies d'exemples qui sont simplement  cités, sans que soit affiché le code .bib
\bibliography{exemples/biblio/biber/crossref} % Pour expliquer crossref
\bibliography{exemples/biblio/biber/augustin} % Pour expliquer l'intérêt des crossref dans la gestion des division
\bibliography{exemples/biblio/fichier/urner}	% Un premier exemple de bibliographie : l'entrée Urner
\bibliography{exemples/biblio/fichier/augustin_editeur} % Pour expliquer la fusion des auteurs et des éditeurs
\bibliography{exemples/biblio/fichier/noms} % Pour expliquer la syntaxe des noms
\bibliography{exemples/biblio/fichier/saxer} % Pour expliquer la syntaxe des titres
\bibliography{exemples/biblio/fichier/felix} % Pour expliquer la syntaxe des titres
\bibliography{exemples/biblio/fichier/junod} % Pour expliquer la syntaxe des titres
\bibliography{exemples/biblio/style1/afrique.bib} % pour expliquer les histoire sur les styles