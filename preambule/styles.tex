% Réglage généraux de style
\setmainfont[Mapping=tex-text,Ligatures=Common]{Linux Libertine O}
\setsansfont[Mapping=tex-text,Ligatures=Common]{Linux Biolinum O}
\setmonofont[Scale=0.75]{DejaVu  Sans Mono}
\newcommand*{\headlongtable}[1]{\hfill\textbf{#1}\hfill}


% Pour les entêtes

\pagestyle{fancy}        
\fancyfoot[C]{\thepage}
\renewcommand{\chaptermark}[1]{\markboth {
     #1}{}}
\renewcommand{\sectionmark}[1]{\markright{
    #1}{}}
\makeatletter
\fancyhead[LE]{\@chapapp~\thechapter}
\makeatother
\fancyhead[RE]{\emph\leftmark}
\fancyhead[LO]{\emph\rightmark}
\fancyhead[RO]{§\,\thesection}
\csappto{clearpage}{\thispagestyle{empty}}

% Réglage pour la bibliographie
\defbibheading{bibliography}{}

\DeclareDataInheritance{*}{*}{
	\noinherit{keywords}
}
\DeclareFieldFormat{pages}{\mkpageprefix[pagination]{#1}}
\DefineBibliographyStrings{french}{byeditor		= \iffieldequalstr{usera}{1}{{éd\adddot}}{{dir\adddotspace}}}
\DeclareBibliographyAlias{url}{online}
\DeclareFieldAlias{bookpagination}{pagination}
\renewcommand{\newunitpunct}[0]{\addcomma\addspace}

\renewcommand{\subtitlepunct}[0]{\addspace\addcolon\addspace}
\renewcommand{\mkibid}[1]{\emph{#1}}

% Réglage pour les listes, avec le package enumitem
\setitemize[1]{label=--}
\setenumerate{itemsep=0pt}
\setitemize{itemsep=0pt}
\setdescription{itemsep=0pt}


% Pour les épigrammes
\makeatletter
\newcommand*{\epigramme}[0]{
	\begin{list}{}{
		\leftmargin 0.25\linewidth
		\listparindent \z@
		\listparindent \z@
                     \itemindent    \listparindent
                     \rightmargin   0.05\linewidth
                     \parsep        \z@ \@plus\p@
		}
		\item\ifthenelse{\equal{\epi}{}}{}{
		\enquote{\epi}}
	
	\begin{flushright}{\small\episource}\end{flushright}
	\end{list}

	\def\episource{}
	\def\epi{}
}
\patchcmd{\@endpart}{\vfil\newpage}{\vfil\epigramme\newpage}{}{}
\makeatother