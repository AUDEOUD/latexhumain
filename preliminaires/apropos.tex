\phantomsection\addcontentsline{toc}{section}{Au sujet de ce livre}\section*{Au sujet de ce livre}

Il va de soi que ce livre a été composé avec \XeLaTeX. Outre les packages dont il traite, j'ai utilisé afin de le composer les packages \package{multicol} pour gérer finement le double colonnage de l'index ; \package{ltxdockit} et \package{minted} pour les citations de code ; \package{framed} pour les boîtes colorées, et enfin, \package{bidi} pour les commandes \XeLaTeX et \XeTeX.



Ce livre est diffusé sous licence \emph{Creatives Commons - Paternité - Partage des Conditions Initiales à l'Identique}. Sommairement\footnote{Pour les détails, je renvoie au texte intégral de la licence : \url{http://creativecommons.org/licenses/by-sa/2.0/fr/legalcode}.}, cela signifie que vous pouvez le diffuser, le dupliquer, le publier et même le modifier si vous respectez deux conditions:
\begin{enumerate}
\item que vous citiez mon nom\footnote{Et que vous ne portiez  pas atteinte à mes droits moraux.};
\item que vous offriez les mêmes droits aux destinataires de vos diffusions\footnote{Les images de pas et d'éclair servant à indiquer les encarts sont tirées du domaine public et (légèrement) modifiées par mes soins. Elles ne sont donc pas affectées par ces règles. Voir \url{http://www.openclipart.org/detail/154855/green-steps-by-netalloy} et \url{http://thenounproject.com/noun/high-voltage/}.}.
\end{enumerate}

Bien sûr, si vous souhaitez me soutenir, vous pouvez acheter cet ouvrage en version papier, ou simplement m'envoyer un petit mot (vous trouverez aisément comment me contacter sur Internet).

Si vous souhaitez améliorer cette œuvre, soyez le bienvenu. Le code est mis à disposition sur GitHub \url{}, un service fonctionnant à l'aide de l'outil de travail collaboratif Git\renvoi{svn} mais disposant d'une interface d'édition en ligne. N'hésitez pas à me demander un accès à l'édition du projet ! 

