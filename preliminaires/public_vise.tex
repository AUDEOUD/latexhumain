\chapter{Introduction : but, visée et public de ce livre}
\section{Un manque important}

Ce livre est le fruit de deux ans de travail et d'utilisation quotidienne de \LaTeX pour la rédaction de mon mémoire de master. Il vient selon nous combler un vide. En effet, si les ouvrages sur \LaTeX sont nombreux, rares sont ceux destinés plus spécifiquement aux sciences humaines. À vrai dire, à part quelques articles ici ou là, je n'ai rien trouvé.


Si le mot \LaTeX a déjà été entendu par des oreilles humanistes, il évoque -- sauf rares exceptions --- au mieux un outil pour les sciences dites dures, au pire la sève d'un arbre ou un plastique aux nombreuses applications. 

De nombreuses raisons pourraient expliquer ce quasi-vide.
\begin{itemize}
\item la tendance générale des chercheurs en sciences humaines à mal connaître ou méconnaître l'outil informatique ;
\item le fait que \LaTeX paraît \emph{au premier abord} peu convivial ;
\item le fait que pendant longtemps \LaTeX ne disposait pas d'outils pour gérer convenablement et simplement une bibliographie selon les normes propres aux sciences humaines : note de bas de page, distinction entre sources primaires et secondaires etc ;
\item le fait que pendant longtemps la gestion des caractères non latins n'était pas des plus simples en \LaTeX ;
\item le fait que les éditeurs de sciences humaines prennent rarement des textes formatés en \LaTeX, parce que les auteurs les rédigent rarement en \LaTeX, parce que les éditeurs les prennent rarement en \LaTeX \ldots
\end{itemize}

Alors que les chercheurs en sciences humaines sont des spécialistes de l'écrit, pendant longtemps seuls des logiciels de types WYSIWYG\footnote{Voir glossaire.}, comme OpenOffice.org ou Microsoft Word ont été utilisés pour la rédaction des travaux universitaires.

Et c'est là un paradoxe : en effet, comme nous le verrons, les logiciels de type WYSIWYG, souffrent de deux défauts majeurs qui devraient inciter les écrivains à changer :
\begin{itemize}
\item Ils produisent une typographie de médiocre qualité. Certes un chercheur ne publie pas lui-même ses travaux mais passe en général par un éditeur. Pour autant, l'apprentissage de la recherche commence par un mémoire de master, puis par une thèse, que l'étudiant édite lui-même pour les soumettre à son directeur. Or, de même qu'un travail manuscrit écrit lisiblement gagne la sympathie de son correcteur\footnote{Combien de fois n'ai-je pas perdu de points pour \enquote{graphie illisible} ?},  une typographie correcte rend la lecture plus agréable.
\item Leur module de gestion bibliographique manque de souplesse. Je ne connais personne qui m'ait déclaré utiliser ceux livrés en standard avec leur traitement de texte.
\end{itemize}

\section{Pourquoi utiliser (Xe)\LaTeX en sciences humaines}

Un des grands arguments avancés en faveur de \LaTeX est qu'il incite à penser \enquote{structure d'abord, forme ensuite}. Pour ma part, j'estime que les logiciels de type WYSIWYG permettent tout à fait de faire cela, pour peu qu'on s'en serve correctement. 

Il est vrai cependant qu'ils incitent à la confusion. Toutefois, je considère que ce genre de confusion ne devrait pas exister dans le domaine des sciences humaines, dans la mesure où une part essentielle du programme de licence consiste à apprendre à faire des plans. Tout chercheur en sciences humaines devrait penser spontanément \enquote{structure}.\revision{Mettre un lien vers le débat sur mon site}

En revanche, en ce qu'il permet de séparer forme et sens, il présente un avantage indéniable. Bien qu'encore une fois cela soit tout à fait possible avec un logiciel de type WYSIWYG en utilisant les styles,  il s'avère malheureusement que peu de gens les utilisent. Par ailleurs, \LaTeX permet de faire beaucoup plus que les simples styles, si tant est qu'on veuille prendre la peine d'un apprentissage.

Pour ma part, j'estime que la raison majeure d'utiliser \LaTeX est la possibilité d'avoir une gestion très souple de la bibliographie et de son rendu, grâce au \emph{package} \package{BibLaTeX}. 

Une autre raison est qu'un traitement de texte est en général bien plus long à lancer qu'un éditeur de texte, que ce dernier plante moins souvent lors de la phase de rédaction, et qu'il est plus rapide à réagir. Qui ne s'est jamais plaint de son Word ou son Openoffice.org pris de lenteur ?

Enfin, la dernière raison est le rendu typographique final de \LaTeX, bien meilleur que celui d'un logiciel de traitement de texte.


\section{Publics visés par cet ouvrage}

Cet ouvrage vise donc trois publics distincts.

Tout d'abord, les étudiants et chercheurs en sciences humaines qui ne sont pas rebutés par l'idée d'apprendre un nouvel outil informatique, qui, s'il leur semblera faire perdre du temps au début, leurs fera gagner un précieux temps à l'usage.

Ensuite, les éditeurs de revues et de livres de sciences humaines, pour les inciter à prendre en compte le format \LaTeX dans leur choix de format de fichier, et pour leur montrer l'avantage de ce format par rapport aux autres formats. J'espère montrer que \LaTeX permet de résoudre nombre de problèmes d'édition, notamment en ce qui concerne les normes bibliographiques, puisqu'il distingue aisément le sens et la forme.

Enfin les utilisateurs de \LaTeX venant des sciences dites \enquote{dures} pour leur montrer les spécificités éditoriales des sciences humaines et la nécessité de packages adaptés.

