\chapter{Mettre en sens (2) : l'art de citer en LaTeX}\label{citertexte}


Nous évoquerons dans ce chapitre les citations \emph{explicites et textuelles}, c'est à dire celles où l'auteur du travail ne se contente pas de renvoyer à une source ou à une étude, mais cite des extraits de cette source.

Ces citations peuvent se faire de deux manières : dans le corps du propos, elle est alors normalement entourée de guillemets\footnote{Il existe cependant deux exceptions courantes à cette règle : les citations dans des alphabets non latins, qui se voient privées de guillemets, et celles d'un texte en latin, qu'on \enquote{prive} de guillemets mais à qui on accordes des italiques. Cette coutume, notamment en ce qui concerne le latin, nous semble bien étrange -- quoique parfaitement explicable d'un point de vue historique -- puisqu'elle accorde à des langues un statut particulier qui leur ferait échapper aux règles typographiques communes.}, ou bien dans un paragraphe spécifique : elle est alors généralement présentée avec des marqueurs typographiques particuliers : changement de la taille de police, de la marge etc.

Nous avions vu plus haut qu'il fallait à ce sujet séparer sens et forme. Nous allons donc présenter ici les commandes servant à marquer des citations.

\begin{attention}
Toute citation se doit d'être accompagnée d'une référence, généralement en note de bas de page. Toutefois toute référence n'accompagne pas nécessairement une citation textuelle. C'est pourquoi nous renvoyons pour la gestion des références bibliographiques à la partie qui lui est consacré.\renvoi{bibliographie}

\end{attention}

\section{Citation dans le corps du texte}\label{guillemets}

Les citations dans le corps d'un texte sont normalement entourées de guillemets français : \verb|« |\ldots \verb| »|. Lorsque qu'on cite un texte qui cite un texte, la citation dans la citation s'entoure de guillemets courbes \verb|“”|. 

\begin{quotation}
	Comme le dit très justement xxx : \enquote{Lorsque yyy déclare \enquote{zzz} il ne déclare rien du tout}.
\end{quotation}

Les claviers disponibles sur nos ordinateurs ne disposent généralement que des guillemets anglais\footnote{Deux exceptions notables sont la disposition de clavier Bépo, et la gestion des claviers sous GNU/Linux (essayez la combinaison de touche \texttt{AltGr + w} et \texttt{AltGr + x}.} : \verb|" "|. 
La plupart des logiciels WYSIWYG convertissent automatiquement les guillemets vers des guillemets français. Rares sont les éditeurs de texte qui le proposent\footnote{C'est d'ailleurs à nos yeux une des raisons qui fait qu'un site internet d'un quotidien national dit \enquote{de référence} n'utilise pas de guillemet français sur sa page d'accueil (à la date du 10 avril 2011), les rédacteurs ne prenant pas le temps de taper les combinaisons complexes de touches nécessaires à la frappe de guillemets français}. 

En outre, en vertu du principe de séparation du sens et de la forme, évoqué plus haut\renvoi{sensforme}, il est plus pertinent d'utiliser une commande spécifique pour indiquer une citation dans le corps du texte.

Nous allons donc utiliser le package \package{csquotes} qui propose des commandes  pour les citations.

\begin{minted}{latex}
	\usepackage{csquotes}
\end{minted}

Le package propose une première commande utile : \cs{\enquote{citation}}, qui sert pour les citations en ligne.

\begin{minted}{latex}
	Comme le dit très justement xxx : \enquote{Lorsque yyy déclare \enquote{zzz} il ne déclare rien du tout}.
\end{minted}


\begin{quotation}
	Comme le dit très justement xxx : \enquote{Lorsque yyy déclare \enquote{zzz} il ne déclare rien du tout}.
\end{quotation}


Nous constatons que le package s'occupe automatiquement de choisir les bons guillemets. Par défaut on ne peut imbriquer que deux niveaux de citation. Toutefois une option du package permet d'avoir plus de niveaux de citation. Par exemple pour en avoir 3 : 

\begin{minted}{latex}
	\usepackage[maxlevel=3]{csquotes}
\end{minted}

Le package propose d'autres options : consultez le manuel.

\section{Citation dans un bloc séparé}


LaTeX propose en standard trois environnements pour citer dans un bloc séparés.

\subsection{L'environnement \enviro{quote}}

Il est prévu pour des courtes citations d'un paragraphe.

\begin{minted}{latex}
	\begin{quote}
	Le corps de Pierre gît à Rome, disent les hommes, le corps de Paul gît à Rome, le corps de Laurent aussi, les corps d'autres martyrs y gisent, mais Rome est misérable, elle est dévastée, affligée, saccagée, incendiée.\footcite[6]{AugustinSermo296}
	\end{quote}
\end{minted}


	\begin{quote}
	Le corps de Pierre gît à Rome, disent les hommes, le corps de Paul gît à Rome, le corps de Laurent aussi, les corps d'autres martyrs y gisent, mais Rome est misérable, elle est dévastée, affligée, saccagée, incendiée.
	\end{quote}

\subsection{L'environnement \enviro{quotation}}

Il est prévu pour des citations plus longues.

\begin{minted}{latex}
	\begin{quotation}
	Que rien exceptées les écritures canoniques ne soit lu en église sous le nom d’écritures divines.

Les écritures canoniques sont : Genèse, Exode, Lévitique, Nombres, Deutéronome, Josué fils de Noun, Juges, Ruth, 4 livres des règnes, 2 livres des paralipoménes, Job, psautier, cinq livres de Salomon, 12 livres des prophètes mineurs, de même Isaïe, Jérémie, Ézechiel, Daniel, Tobie, Judith, Esther, 2 livres d’Esdras, 2 livres des Maccabées.

Du nouveau testament sont : 4 évangiles, un livre des actes des apôtres, 14 lettres de l’apôtre Paul, 2 de Pierre, 3 de Jean, 1 de Jude, 1 de Jacques, l’apocalypse de Jean.

Que l’Église d'outre-mer soit consultée pour la confirmation de ce canon.

De plus, qu'il soit permis de lire les passions des martyrs, lorsqu'on célèbre leurs anniversaires.
	\end{quotation}
\end{minted}

	\begin{quotation}
Que rien exceptées les écritures canoniques ne soit lu en église sous le nom d’écritures divines.

Les écritures canoniques sont : Genèse, Exode, Lévitique, Nombres, Deutéronome, Josué fils de Noun, Juges, Ruth, 4 livres des règnes, 2 livres des paralipoménes, Job, psautier, cinq livres de Salomon, 12 livres des prophètes mineurs, de même Isaïe, Jérémie, Ézechiel, Daniel, Tobie, Judith, Esther, 2 livres d’Esdras, 2 livres des Maccabées.

Du nouveau testament sont : 4 évangiles, un livre des actes des apôtres, 14 lettres de l’apôtre Paul, 2 de Pierre, 3 de Jean, 1 de Jude, 1 de Jacques, l’apocalypse de Jean.

Que l’Église d'outre-mer soit consultée pour la confirmation de ce canon.

De plus, qu'il soit permis de lire les passions des martyrs, lorsqu'on célèbre leurs anniversaires.
\footcite{BreveHippone}
	\end{quotation}

\subsection{L'environnement \enviro{verse}}

Il est prévu pour des citations de poèmes. Sa particularité est de gérer spécifiquement les fins de lignes, afin de continuer à distinguer les strophes les unes des autres.

\begin{minted}{latex}
\begin{verse}

Demain, dès l'aube, à l'heure où blanchit la campagne,\\
Je partirai. Vois-tu, je sais que tu m'attends.\\
J'irai par la forêt, j'irai par la montagne.\\
Je ne puis demeurer loin de toi plus longtemps.\\

Je marcherai les yeux fixés sur mes pensées,\\
Sans rien voir au dehors, sans entendre aucun bruit,\\
Seul, inconnu, le dos courbé, les mains croisées,\\
Triste, et le jour pour moi sera comme la nuit.\\

Je ne regarderai ni l'or du soir qui tombe,\\
Ni les voiles au loin descendant vers Harfleur,\\
Et, quand j'arriverai, je mettrai sur ta tombe\\
Un bouquet de houx vert et de bruyère en fleur.\\

\end{verse}
\end{minted}

\begin{verse}


Demain, dès l'aube, à l'heure où blanchit la campagne,\\
Je partirai. Vois-tu, je sais que tu m'attends.\\
J'irai par la forêt, j'irai par la montagne.\\
Je ne puis demeurer loin de toi plus longtemps.\\

Je marcherai les yeux fixés sur mes pensées,\\
Sans rien voir au dehors, sans entendre aucun bruit,\\
Seul, inconnu, le dos courbé, les mains croisées,\\
Triste, et le jour pour moi sera comme la nuit.\\

Je ne regarderai ni l'or du soir qui tombe,\\
Ni les voiles au loin descendant vers Harfleur,\\
Et, quand j'arriverai, je mettrai sur ta tombe\\
Un bouquet de houx vert et de bruyère en fleur.\footcite{demain}\\

\end{verse}

\section{Citations tronquées et modifiées}

Le package \package{csquotes} proposent deux commandes spécifiques pour signaler qu'une citation a été tronquée ou modifiée.

\subsection{Citation tronquée}

La commande \cs{textelp{}} signale un texte tronqué. Si on lui passe un argument, on signale qu'on ajoute un texte après la troncature.

\begin{minted}{latex}
	\begin{quotation}
	Que rien exceptées les écritures canoniques ne soit lu en église sous le nom d’écritures divines.
\textelp{Suit la liste des écritures canoniques.}

Que l’Église d'outre-mer soit consultée pour la confirmation de ce canon.

De plus, qu'il soit permis de lire les passions des martyrs, lorsqu'on célèbre leurs anniversaires.
	\end{quotation}
\end{minted}

	\begin{quotation}
	Que rien exceptées les écritures canoniques ne soit lu en église sous le nom d’écritures divines.
\textelp{Suit la liste des écritures canoniques.}

Que l’Église d'outre-mer soit consultée pour la confirmation de ce canon.

De plus, qu'il soit permis de lire les passions des martyrs, lorsqu'on célèbre leurs anniversaires.\footcite{BreveHippone}
	\end{quotation}

\begin{anedocte}
On peut décider de la manière dont la troncature et l'ajout sont signalés : consulter le manuel.
\end{anedocte}

\subsection{Citation modifiée}

Pour signaler une modification dans une citation, on utilise  \cs{textins{}}.
\begin{minted}{latex}
\begin{quotation}
	Comme le disait très justement xxx : \enquote{Lorsque yyy \textins{a déclaré} \enquote{zzz} il \textins{n'a rien déclaré} du tout}.
\end{quotation}
\end{minted}

\begin{quotation}
	Comme le disait très justement xxx : \enquote{Lorsque yyy \textins{a déclaré} \enquote{zzz} il \textins{n'a rien déclaré} du tout}.
\end{quotation}

La commande \cs{textins*{}} est une variante, servant pour déclarer les changements mineurs nécessaires au nouveau contexte d'énonciation : mise en majuscules, changement de personne etc.
