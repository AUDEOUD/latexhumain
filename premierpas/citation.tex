\chapter{Mettre en sens (2) : l'art de citer en LaTeX}\label{citertexte}

\begin{prealable}
Nous évoquerons dans ce chapitre les citations \emph{explicites et textuelles}, c'est à dire celles où l'auteur du travail ne se contente pas de renvoyer à une source ou à une étude, mais cite des extraits de cette source.

Une citation peut se faire de deux manières : dans le corps du propos, elle est alors normalement entourée de guillemets, ou bien dans un paragraphe spécifique : elle est alors généralement présentée avec des marqueurs typographiques particuliers : changement de la taille de police, de la marge etc.

Nous avions vu plus haut qu'il fallait séparer sens et forme. Nous allons donc présenter ici les commandes servant à marquer des citations.\renvoi{sensforme}

\end{prealable}
\begin{attention}
Toute citation se doit d'être accompagnée d'une référence, généralement en note de bas de page. Toutefois toute référence n'accompagne pas nécessairement une citation textuelle. C'est pourquoi nous renvoyons pour la gestion des références bibliographiques à la partie qui lui est consacrée.\renvoi{bibliographie}

\end{attention}

\section{Citation dans le corps du texte}\label{guillemets}

Les citations dans le corps d'un texte sont normalement entourées de guillemets français : \verb|«»|. Lorsque qu'on cite un texte qui cite un texte, la citation dans la citation s'entoure de guillemets courbes \verb|“”|. 

\begin{quotation}
    Comme le dit très justement xxx : \enquote{Lorsque yyy déclare \enquote{zzz} il ne déclare rien du tout}.
\end{quotation}

Les claviers disponibles sur nos ordinateurs ne disposent généralement en accès direct que des guillemets anglais\footnote{Une exception notable est la disposition de clavier Bépo.} : \verb|"|. 
La plupart des logiciels WYSIWYG convertissent automatiquement ces guillemets en guillemets français. Rares sont les éditeurs de texte qui le proposent\footnote{C'est d'ailleurs à nos yeux une des raisons qui fait qu'un site internet d'un quotidien national dit \enquote{de référence} n'utilise pas de guillemet français sur sa page d'accueil (à la date du 10 avril 2011), les rédacteurs ne prenant pas le temps de taper les combinaisons complexes de touches nécessaires à la frappe de guillemets français}. 

En outre, en vertu du principe de séparation du sens et de la forme, évoqué plus haut\renvoi{sensforme}, il est plus pertinent d'utiliser une commande spécifique pour indiquer une citation dans le corps du texte.

Nous allons donc utiliser le package \package{csquotes} qui propose des commandes  pour les citations.



Le package propose une première commande utile : \csp{enquote}\marg{citation}, qui sert pour les citations en ligne.

\begin{latexcode}
Comme le dit très justement xxx : \enquote{Lorsque yyy déclare \enquote{zzz} il ne déclare rien du tout}.
\end{latexcode}


\begin{quotation}
Comme le dit très justement xxx : \enquote{Lorsque yyy déclare \enquote{zzz} il ne déclare rien du tout}.
\end{quotation}


Nous constatons que \package{csquotes} s'occupe automatiquement de choisir les bons guillemets. Par défaut on ne peut imbriquer que deux niveaux de citation. Toutefois une option du package permet d'avoir plus de niveaux de citation. Par exemple pour en avoir 3 : 

\begin{latexcode}
\usepackage[maxlevel=3]{csquotes}
\end{latexcode}

Le package propose d'autres options et commandes : consultez le manuel\footcite{csquotes}.

\section{Citation dans un bloc séparé}


LaTeX propose en standard trois environnements pour citer dans un bloc séparé.

\subsection{L'environnement \environoidx{quote}}\sindex[enviro]{quote}

Il est prévu pour des courtes citations d'un paragraphe.

\begin{latexcode}
\begin{quote}
Le corps de Pierre gît à Rome, disent les hommes,
le corps de Paul gît à Rome, le corps de Laurent aussi,
les corps d'autres martyrs y gisent,
mais Rome est misérable,
elle est dévastée, affligée, saccagée, incendiée.
\end{quote}
\end{latexcode}


    \begin{quote}
    Le corps de Pierre gît à Rome, disent les hommes, le corps de Paul gît à Rome, le corps de Laurent aussi, les corps d'autres martyrs y gisent, mais Rome est misérable, elle est dévastée, affligée, saccagée, incendiée\footcite[6]{AugustinSermo296}.
    \end{quote}

\subsection{L'environnement \environoidx{quotation}}\sindex[enviro]{quotation}

Il est prévu pour des citations plus longues.

\begin{latexcode}
\begin{quotation}
Que rien exceptées les écritures canoniques ne soit lu en église sous le nom d’écritures divines.

Les écritures canoniques sont : Genèse, Exode, Lévitique, Nombres, Deutéronome,
Josué fils de Noun, Juges, Ruth, 4 livres des règnes, 2 livres des paralipoménes,
Job, psautier, cinq livres de Salomon, 12 livres des prophètes mineurs,
de même Isaïe, Jérémie, Ézechiel, Daniel,
Tobie, Judith, Esther,
2 livres d’Esdras, 2 livres des Maccabées.

Du nouveau testament sont :
4 évangiles, un livre des actes des apôtres,
14 lettres de l’apôtre Paul, 2 de Pierre,
-3 de Jean, 1 de Jude, 1 de Jacques,
l’apocalypse de Jean.

Que l’Église d'outre-mer soit consultée pour la confirmation de ce canon.

De plus, qu'il soit permis de lire les passions des martyrs,
lorsqu'on célèbre leurs anniversaires
\end{quotation}
\end{latexcode}

    \begin{quotation}
Que rien exceptées les écritures canoniques ne soit lu en église sous le nom d’écritures divines.

Les écritures canoniques sont : Genèse, Exode, Lévitique, Nombres, Deutéronome,
Josué fils de Noun, Juges, Ruth, 4 livres des règnes, 2 livres des paralipoménes,
Job, psautier, cinq livres de Salomon, 12 livres des prophètes mineurs,
de même Isaïe, Jérémie, Ézechiel, Daniel,
Tobie, Judith, Esther,
2 livres d’Esdras, 2 livres des Maccabées.

Du nouveau testament sont :
4 évangiles, un livre des actes des apôtres,
14 lettres de l’apôtre Paul, 2 de Pierre,
-3 de Jean, 1 de Jude, 1 de Jacques,
l’apocalypse de Jean.

Que l’Église d'outre-mer soit consultée pour la confirmation de ce canon.

De plus, qu'il soit permis de lire les passions des martyrs,
lorsqu'on célèbre leurs anniversaires\footcite{BreveHippone}.
    \end{quotation}

\begin{attention}
Le fond gris que vous constatez ici n'est pas standard avec \LaTeX : il est propre au livre que vous avez sous les yeux.
\end{attention}

\subsection{L'environnement \environoidx{verse} et le package \packagenoidx{verse}}\sindex[enviro]{verse}\sindex[pkg]{verse}


L'environnement  \enviro{verse} permet de citer de façon rapide et simple des poèmes, en gérant notamment le rejet en cas de vers trop long. Chaque vers, à l'exception du dernier, doit se terminer par \verb|\\|. Si le vers est composé de plusieurs strophes, il suffit de sauter une ligne entre chaque strophe.    

Cet environnement est très limité: il ne permet pas de numéroter ni d'indenter les vers, ni encore de rajouter un titre. Si l'on est amené à citer fréquemment des vers, il vaut mieux appeler dans le préambule le package \package{verse}. La citation du poème se fait de la même manière.


Avec ce package pour numéroter les vers cités il faut indiquer \csp{poemlines}\marg{nbr} :  \arg{nbr} 
définit la fréquence avec laquelle les vers sont numérotés. 


Citons pour  exemple un poème entier, dont les vers sont numérotés, de façon assez traditionnelle, une fois sur cinq. En frappant ceci:

\begin{latexcode}
\begin{verse}
\poemlines{5}

Demain, dès l'aube, à l'heure où blanchit la campagne,\\
Je partirai. Vois-tu, je sais que tu m'attends.\\
J'irai par la forêt, j'irai par la montagne.\\
Je ne puis demeurer loin de toi plus longtemps.\\

Je marcherai les yeux fixés sur mes pensées,\\
Sans rien voir au dehors, sans entendre aucun bruit,\\
Seul, inconnu, le dos courbé, les mains croisées,\\
Triste, et le jour pour moi sera comme la nuit.\\

Je ne regarderai ni l'or du soir qui tombe,\\
Ni les voiles au loin descendant vers Harfleur,\\
Et, quand j'arriverai, je mettrai sur ta tombe\\
Un bouquet de houx vert et de bruyère en fleur.

\end{verse}
\end{latexcode}

on obtient cela :

\begin{verse}
\poemlines{5}

Demain, dès l'aube, à l'heure où blanchit la campagne,\\
Je partirai. Vois-tu, je sais que tu m'attends.\\
J'irai par la forêt, j'irai par la montagne.\\
Je ne puis demeurer loin de toi plus longtemps.\\

Je marcherai les yeux fixés sur mes pensées,\\
Sans rien voir au dehors, sans entendre aucun bruit,\\
Seul, inconnu, le dos courbé, les mains croisées,\\
Triste, et le jour pour moi sera comme la nuit.\\

Je ne regarderai ni l'or du soir qui tombe,\\
Ni les voiles au loin descendant vers Harfleur,\\
Et, quand j'arriverai, je mettrai sur ta tombe\\
Un bouquet de houx vert et de bruyère en fleur.\footcite{demain}

\end{verse}



Si l'on ne cite pas un poème en entier, mais simplement un passage, il faut indiquer à quel vers commence le passage cité, afin que la numérotation des vers que l'on cite soit la bonne.
On utilise pour cela la commande \csp{setverselinenums}\marg{premier vers}\marg{premier vers numéroté} : \arg{premier vers} indique le numéro du premier vers que l'on cite, \arg{premier vers numéroté} où doit commencer la numérotation. Ainsi, \cs{setverselinenums}\marg{12}\marg{15} indique que l'extrait que l'on cite commence par le douzième vers du poème, et que le premier vers numéroté sera le quinzième. 
Avec \cs{setverselinenums}\marg{12}\marg{12}, le premier vers cité sera le premier numéroté.

Ne citant que la troisième strophe, par exemple, il faut frapper :

\begin{latexcode}
\begin{verse}
\poemlines{5}
\setverselinenums{9}{10}


Je ne regarderai ni l'or du soir qui tombe,\\
Ni les voiles au loin descendant vers Harfleur,\\
Et, quand j'arriverai, je mettrai sur ta tombe\\
Un bouquet de houx vert et de bruyère en fleur.

\end{verse}
\end{latexcode}

pour obtenir, correctement numéroté, 

\begin{verse}
\poemlines{5}
\setverselinenums{9}{10}


Je ne regarderai ni l'or du soir qui tombe,\\
Ni les voiles au loin descendant vers Harfleur,\\
Et, quand j'arriverai, je mettrai sur ta tombe\\
Un bouquet de houx vert et de bruyère en fleur.

\end{verse}

\begin{plusloins}
Dans la typographie française, il est d'usage de mettre un crochet droit [ au début d'un rejet. Ni l'environnement ni le package \package{verse} ne suivent cette règle. Pour obtenir le crochet droit, il faut indiquer charger le package  \package{gmverse} en lui passant l'option \option{squarebr} :

\begin{latexcode}
\usepackage}[squarebr]{gmverse}
\end{latexcode}

Puis insérer, une fois commencé l'environnement \enviro{verse}, la commande \csp{versehangrightsquare}.
\end{plusloins}

Le package \package{verse} permet aussi d'indenter de façon très souple une strophe. On indique avec la commande \csp{indentpattern}\marg{$n_1 n_2 n_x$} l'indentation de chaque vers contenu dans la strophe encadrée dans l'environnement \enviro{patverse}: \arg{$n_1$} correspond au premier vers, \arg{$n_2$} au deuxième et ainsi de suite. Le premier vers n'est jamais indenté, mais il suffit de le faire précéder de \csp{vin}. Avec le code suivant :

\begin{latexcode}
\begin{verse} 
\poemlines{1}
\indentpattern{010110} 
\begin{patverse} 

\vin There was a young lady of Ryde \\
 Who ate some apples and died.  \\
 The apples fermented \\
 Inside the lamented \\
 And made cider inside her inside. 

\end{patverse}  
\end{verse}
\end{latexcode}

on obtient donc:
  
\begin{verse} \poemlines{1}

\indentpattern{010110} \begin{patverse} 

\vin There was a young lady of Ryde \\
 Who ate some apples and died.  \\
 The apples fermented \\
 Inside the lamented \\
 And made cider inside her inside\footcite{EdwardLear}. 

\end{patverse}  \end{verse} 
 
 Si l'on a besoin de répéter une indentation tout le long d'un poème, on remplace \enviro{patverse} par \enviro{patverse*}. Donc pour citer un long extrait en hexamètre dactylique, plutôt que d'indiquer l'indentation pour chaque vers, il suffit  d'indiquer \cs{indentpattern}\verb|{01}| et d'insérer tout le poème  dans l'environnement \enviro{patverse*} pour avoir un vers sur deux indenté.
 
 
 Le package \package{verse} offre bien d'autres possibilités qui dépassent la simple citation de poésie dans le cours d'un texte. À l'instar d'un autre package, \package{poemscol}, c'est un véritable outil pour éditer de la poésie (nous vous renvoyons ici aux manuels de ces packages). Ni l'un ni l'autre ne permettent en revanche de faire des éditions bilingues: pour cela, il faut utiliser les packages \package{ledmac} et \package{ledpar}. \renvoi{ledmac}


\section{Citations tronquées et modifiées}

Le package \package{csquotes} proposent deux commandes spécifiques pour signaler qu'une citation a été tronquée ou modifiée.

\subsection{Citation tronquée}

La commande \csp{textelp}\arg{texte} signale un texte tronqué. L'argument \arg{texte} est inséré après la troncature, entre crochets (par défaut) ; pour ne rien insérer, le laisser vide.

\begin{latexcode}
\begin{quotation}
Que rien exceptées les écritures canoniques ne soit lu
en église sous le nom d’écritures divines.
\textelp{Suit la liste des écritures canoniques.}

Que l’Église d'outre-mer soit consultée pour la confirmation de ce canon.

De plus, qu'il soit permis de lire les passions des martyrs,
lorsqu'on célèbre leurs anniversaires.
\end{quotation}
\end{latexcode}

    \begin{quotation}
    Que rien exceptées les écritures canoniques ne soit lu en église sous le nom d’écritures divines.
\textelp{Suit la liste des écritures canoniques.}

Que l’Église d'outre-mer soit consultée pour la confirmation de ce canon.

De plus, qu'il soit permis de lire les passions des martyrs, lorsqu'on célèbre leurs anniversaires\footcite{BreveHippone}.
    \end{quotation}

\begin{plusloins}
On peut décider de la manière dont la troncature et l'ajout sont signalés : consulter le manuel\footcite{csquotes_ellipses}.
\end{plusloins}

\subsection{Citation modifiée}

Pour signaler une modification dans une citation, on utilise  \csp{textins}\marg{texte modifié}.

\begin{latexcode}
\begin{quotation}
Comme le disait très justement xxx : \enquote{Lorsque yyy \textins{a déclaré}
\enquote{zzz} il \textins{n'a rien déclaré} du tout}.
\end{quotation}
\end{latexcode}

\begin{quotation}
    Comme le disait très justement xxx : \enquote{Lorsque yyy \textins{a déclaré} \enquote{zzz} il \textins{n'a rien déclaré} du tout}.
\end{quotation}

La commande \csp{textins*} est une variante, servant pour déclarer les changements mineurs nécessaires au nouveau contexte d'énonciation : mise en majuscules, changement de personne etc.
