\chapter{Insérer des éléments non textuels}

\begin{prealable}
	Dans ce chapitre nous allons examiner comment insérer des éléments qui ne font pas parties du flux du texte : images, graphismes, tableaux de données.
\end{prealable}

\section{Insérer des images}

Les images insérables avec Xe\LaTeX doivent être au format JPEG (extension \ext{jpg}) ou PNG (extension \ext{png}) ou PDF (extension \ext{jpg}).

L'insertion d'une image se fait de la manière suivante :
\begin{minted}{latex}
\includegraphics[param. opti.]{chemin de l'image}
\end{minted}

Le chemin de l'image s'indique de la même manière que le chemin des fichiers inclus, vu plus haut.\renvoi{chemin}

Les paramètres optionnels peuvent être les suivants :

\begin{longtable}{l||l|l}
	Paramètres & Signification & Exemple	\\
	\hline
	\endhead
	width		& Largeur 	& \verb|width=10cm| 	\\
	height		& Hauteur	& \verb|height=10cm|	 \\
	scale		& Redimension proportionelle & \verb|scale=0.5|\\
	
\end{longtable}

\begin{anedocte}
Le centimètre n'est pas la seule  unité de mesure disponible en \LaTeX. Nous en parlons en annexe de ce livre.
\end{anedocte}

\begin{attention}
	Évidemment, il est plus correct d'insérer une légende sous une image. Nous expliquons comment faire plus loin.\renvoi{legende}
\end{attention}

\section{Insertion de graphisme et de schémas}

On peut vouloir insérer divers type de graphisme ou de schémas : graphismes statistiques, arbres généalogiques, arbres de familles de manuscrits etc.

Deux solutions s'offrent à nous :
\begin{enumerate}
\item Utiliser un logiciel externe qui exporterait le graphisme dans une image, qu'on insérerait comme n'importe quelle image.
\item Utiliser les possibilités de \LaTeX et du paquet \package{TikZ}.
\end{enumerate}

La première solution paraît de prime abord plus simple, puisqu'elle ne nécessite pas d'apprendre des nouveaux éléments en \LaTeX. Toutefois elle nécessite de gérer plus de fichiers\footnote{Prenons un graphisme généré par un tableur : il faudra conserver la feuille du tableur plus l'image du graphisme généré par celui-ci.}. En outre, les images exportés par ces logiciels étant bien souvent de type bitmap et non pas vectorielle%
\footnote{Il existe deux manières en informatiques de coder une image : en codant point par point (image bitmap) ou bien en codant simplement les formes, par exemple des cercles ou des segments (images vectorielle). La première méthode est adaptée à des images complexes, comme par exemple une photos,  le seconde méthode est conseillée pour les images \enquote{simple} tel que les graphisme. La méthode vectorielle possède l'avantage d'être plus facilement redimensionnable  sans perte de qualité et de permettre aisément la sélection de texte.},
on perd la possibilité de sélectionner dans le PDF généré par LaTeX les textes, et on risque d'avoir des problèmes de redimensionnement.

Nous allons donc ici vous présenter une possibilité du \package{TikZ} : la représentation d'une famille de manuscrits. 

Nous avons choisi cette exemple car il est relativement simple. Mais \package{TikZ} permet de faire bien plus, comme par exemple de produire des diagrammes statistiques : nous renvoyons à d'autres ouvrages pour en savoir plus\footcites[Outre le manuel on pourra lire][]{tikzimpatient}[on pourra également consulter le site ][]{tikzexample}.
Ce package possède une syntaxe spécifique. 

\begin{attention}
	Bien que le package s'appelle \package{TikZ}, il faut dans le préambule l'appeler \emph{sans majuscules}.
	\begin{minted}{latex}
	\usepackage{tikz}
	\end{minted} 
\end{attention}

\subsection{Arbre représentant des familles de manuscrits}

Notre exemple est volontairement simple. Nous disposons de six familles de manuscrits d'un même texte. La  famille A a engendré les familles B, C, D. Les familles E et F pour leurs parts sont issus de la famille D. 

Le code est le suivant 
\inputminted{exemples/premierpas/nontextuels/genealogie.tex}[linenos=true]

Ce qui donne le résultat de la figure\ref{figure:genealogie}\renvoi{figure:genealogie}. \begin{figure}[hp]
\centering
\begin{tikzpicture}
	\node {A}
		child { node {B}}
		child { node {C}}
		child { node {D}
			 child {
				node{E}
				}
			 child {
				node{F}
				}
			}	
	;
\end{tikzpicture}
\caption{Exemple de généalogie}
\label{figure:genealogie}	
\end{figure}
Analysons le code :

\begin{description}
\item[Ligne 1]l'environnement \enviro{tikzpicture} est l'environnement utilisé pour insérer une figure TikZ. Il peut recevoir de nombreux arguments pour modifier certains aspects, tels que épaisseurs par défaut des lignes, tailles du texte, orientation du schéma. Nous renvoyons à la documentation.
\item[Ligne 2] un nœud TikZ, matérialisé ici par le commande \commande{node} est un bloc de texte. On pourrait passer des options à cette commande, afin de modifier certains aspects (par exemple afficher un contour au texte).
\item[Lignes 3 et 4] \verb|child| désigne ici une opération : associer un fils au nœud précédent. Ce fils est lui même un nœud (\verb|node|)
\item[Lignes 5 à 12]même chose que précédemment, sauf que le nœud D se voit attribuer des fils (E et F). Notez le nombre d'accolade et leurs imbrications : c'est ce qui permet à \package{TikZ} de construire correctement la généalogie.
\item[Ligne 13] le point-virgule est important.Il est obligatoire après chaque commande \package{TikZ} (ici \commande{node}).
\item[Ligne 14] fin de graphisme \package{TikZ}
\end{itemize}

Évidemment un tel code est relativement complexe. C'est pourquoi nous conseillons de commenter attentivement le code\footnote{Pour des raisons de place nous ne l'avons pas fait ici} et de prêter extrêmement attention  aux accolades. Par ailleurs nous recommandons de mettre chaque graphisme \package{TikZ} dans un fichier séparé.

Pour faciliter la création de graphisme TikZ il est possible d'utiliser des logiciels WYSIWYG\renvoi{logiciels:tikz}.

Notre arbre généalogique est relativement simple : il n'y a pas de consanguinité. Si cela avait été le cas la construction aurait été plus complexe : on n'aurait pas pu utiliser les opérations de types \verb|child| et il aurait fallu positionner précisément les éléments en utilisant un système de coordonnée.



\section[La notion de flottants]{Où insérer les éléments non textuels ? : la notion de flottants}
\label{legende}
Nous avons vu comment insérer des éléments non textuels. Mais vous constaterez rapidement que la mise en forme n'est pas toujours des meilleurs, l'élément s'insérant dans le texte à l'endroit précis où il a été appelé, ce qui peut entraîner des espaces blancs disgracieux.

En outre, ces éléments non textuels disposent habituellement d'une légende.

Pour résoudre ces deux problèmes --- positionnement esthétique et légende --- \LaTeX utilise la notion de flottant. \emph{Un flottant est donc un élément non textuel que LaTeX essaie d'insérer au meilleur endroit du point de vue de l'esthétique et qui dispose (éventuellement) d'une légende.}

Il existe deux types principaux de flottant :
\begin{itemize}
	\item Les figures.
	\item Les tableaux.
\end{itemize}

La syntaxe pour insérer le premier est la suivante :

\begin{minted}{latex}
\begin{figure}[paramètre de placement]
	Insertion de la figure ou de l'image suivant les codes montrés plus haut.
	\caption{Légende}
\end{figure} 
\end{minted}

Celle pour insérer le second est la suivante :
\begin{minted}{latex}
\begin{table}[paramètre de placement]
	Insertion d'un tableau suivant le codes montrés plus haut.
	\caption{Légende}
\end{table} 
\end{minted}

Si la commande \commande{caption} est insérée au dessus de la figure, de l'image ou du table, la légende sera située au dessus de l'élément et non en dessous. Il est possible, mais déconseillée, de ne pas mettre de légende.

Un exemple de flottant se situe  dans ce chapitre : la figure \ref{figure:genealogie}\renvoi{figure:genealogie}.
\begin{attention}
	On pourrait vouloir changer les intitulés comme \emph{figure} et \emph{tableau}. Tout ceci est possible, nous en parlerons plus loins dans le chapitre consacrée aux chaînes de langue.\renvoi{chainedelangue}.
	
	De même on pourrait souhaiter renvoyer automatiquement vers le numéro et la page d'une figure : nous en parlons dans le chapitre consacré à la navigation à l'intérieur d'un document.\renvoi{label}
\end{attention}

\subsection{Choix de l'emplacement du flottant}

Le paramètre de placement indique à \LaTeX comment placer idéalement les flottants. Ce sont des paramètres indicatifs que le compilateur essaiera autant que possible de suivre, sans pour autant être contraint. Ces valeurs sont les suivantes :

\begin{longtable}{l|l}
	Valeur & Signification	\\
	\hline
	\endhead
	h 	& À l'emplacement de l'appel au flottant 	\\
	t 	& En haut d'une page				\\
	b 	& En bas d'une page				\\
	p 	& Sur une page dédiée aux flottants		\\
\end{longtable}


Si le système des flottants permet généralement de conserver une mise en page correcte tout en n'éloignant pas trop l'élément de son emplacement, il arrive parfois qu'il l'éloignement soit trop important.

Pour remédier à ce problème, on pourra utiliser la commande \commande{FloatBarrier} du paquet \package{placeins}. 
Tout les flottants appelés précédemment seront systématiquement placés avant cette commande.

\subsection{Sous-figure}

Il est possible d'insérer au sein d'une figure des sous-figures, chacune d'entre elles disposant d'une légende, en plus de la légende principal.
Pour ce faire il faut utiliser le paquet \package{subfig}.

La syntaxe est la suivante :  
\begin{minted}{latex}
\begin{figure}[paramètre de placement]
	\subfloat[Légende 1]{Code de la figure 1}
	\subfloat[Légende 2]{Code de la figure 2}
	…
	\subfloat[Légende N]{Code de la figure N}
	\caption{Légende principale}
\end{figure} 
\end{minted}

\section{Création de tableau}

La création de tableau en \LaTeX bien que de prime abord simple nécessite pour autant une extrême rigueur.

\subsection{Création depuis un fichier \ext{cvs}}

Évidemment une telle syntaxe est bien compliquée et risque d'entraîner des erreurs. Il est toutefois possible d'utiliser un fichier au format \ext{cvs}, que n'importe quel tableur est capable de produire.

Le format \ext{cvs} est un format standard de données tabulaires. Malheureusement il ne dispose ni de mise en forme ni de possibilité de fusion de cellule.

\begin{anedocte}Une solution peut être alors d'utiliser la macro Calc2LaTeX pour le logiciel OpenOffice.org, mais nous n'avons pas testé cette macro.
\end{anedocte}

Il vous donc exporter votre tableau au format cvs. La première ligne correspondant à l'entête.

Toutefois pour des données simples, comme par exemple un tableau statistiques, il suffit amplement. Il est possible de créer à partir d'un fichier \ext{cvs} un environnement \enviro{tabular} ou\enviro{longtable}.

Le premier se crée  avec la commande

\begin{minted}{latex}
\CSVtotabular{fichier}{alignement}{premiere}{milieu}{dernier}
\end{minted}

le second avec la commande

\begin{minted}{latex}
\CSVtolongtable{fichier}{alignement}{premiere}{milieu}{dernier}
\end{minted}

\begin{description}
\item[fichier]est le nom du fichier \ext{.cvs}. La méthode pour indiquer le chemin du fichier est la même que habituelle.\renvoi{chemin}
\item[alignement] indique les alignements de colonnes, comme vu précédemment\renvoi{alignement}.
\item[premiere] indique le contenu de la première ligne (dans le tableau que vous allez généré).
\item[milieu] indique le contenu des lignes du milieu.
\item[dernier] indique le contenu de la dernière ligne.
\end{description}

Pour les arguments 

Exemple avec le fichier \fichier{csv.csv} :


\inputminted{exemples/premierpas/nontextuels/cvstolongtable.tex}


Le résultat est le tableau \ref{cvstolongtable}\renvoi{cvstolongtable}.

\begin{table}[p]
\CSVtolongtable{exemples/premierpas/nontextuels/csv.csv}{l|c|r|}{
\emph{Colonne A}	& \emph{Colonne B} & \emph{Colonne C} \\ 
\hline}
{\insertbyname{A}&\insertbyname{B}&\insertbyname{C}\\}
{\emph{\insertbyname{A}}&\emph{\insertbyname{B}}&\emph{\insertbyname{C}}\\}
\caption{Exemple d'utilisation de la commande \commande{cvstolongtable}}
\label{cvstolongtable}
\end{table}


\begin{anedocte}Pour afficher un graphique statistique, nous recommandons d'utiliser le package \package{TikZ}\renvoi{TikZ} et les nombreux packages dérivés. Toutefois le \package{cvstools} possède une fonction pour tracer un diagramme circulaire à partir d'un fichier \ext{cvs}. Nous renvoyons à sa documentation.

\end{anedocte}

	