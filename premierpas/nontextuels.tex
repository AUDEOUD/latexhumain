\chapter{Insérer des éléments non textuels}

\begin{prealable}
	Dans ce chapitre nous allons examiner comment insérer des éléments qui ne font pas parties du flux du texte : images, graphismes, tableaux de données.
\end{prealable}

\section[La notion de flottants]{Où insérer les éléments non textuels ? : la notion de flottants}

Nous avons vu comment insérer des éléments non textuels. Mais vous constaterez rapidement que la mise en forme n'est pas toujours des meilleurs, l'élément s'insérant dans le texte à l'endroit précis où il a été appelé, ce qui peut entraîner des espaces blancs disgracieux.

En outre, ces éléments non textuels disposent habituellement d'une légende et on aimerait que cette légende serve ensuite pour construire une table des figures.

Pour résoudre ces deux problèmes --- positionnement esthétique et légende --- \LaTeX utilise la notion de flottant. \emph{Un flottant est donc un élément non textuel que LaTeX essaie d'insérer au meilleur endroit du point de vue de l'esthétique et qui dispose (éventuellement) d'une légende.}
