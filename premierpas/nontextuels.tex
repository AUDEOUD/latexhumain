\chapter{Insérer des éléments non textuels}

\begin{intro}
    Dans ce chapitre nous allons examiner comment insérer des éléments qui ne font pas partie du flux du texte : images, graphismes, tableaux de données.
\end{intro}

\section{Insérer des images}

Les images insérables avec \XeLaTeX sont au format JPEG (extension~\ext{jpg}), PNG (extension~\ext{png}) ou PDF (extension~\ext{pdf}). Il est nécessaire d'utiliser le package \package{graphicx}.


L'insertion d'une image se fait de la manière suivante : \csp{includegraphics}\oarg{param. opti.}\marg{chemin de l'image}.

Le chemin de l'image s'indique de la même manière que le chemin des fichiers inclus\renvoi{chemin}.
Les principaux paramètres optionnels sont les suivants\footcite[Pour les autres, on consultera][]{graphicx_options} :

\begin{longtable}{|p{0.2\linewidth}|p{0.6\linewidth}|p{0.2\linewidth}|}
    \hline
    \centering\textbf{Paramètres} & \centering\textbf{Signification} & \centering\textbf{Exemple}    \tabularnewline
    \hline
    \endhead
    \hline
    \endfoot
    angle        & Angle exprimé en degrés. Un nombre positif indique une rotation anti-horaire, un nombre négatif une rotation horaire.     & \verb|angle=90| \\
    height        & Hauteur    & \verb|height=10cm|     \\
    scale        & Redimension proportionnelle & \verb|scale=0.5|\\
    width        & Largeur     & \verb|width=10cm|     \\
\end{longtable}

\begin{plusloins}
Le centimètre n'est pas la seule  unité de mesure disponible en \LaTeX. Nous en parlons en annexe de ce livre\renvoi{unite}.
\end{plusloins}

\begin{plusloins}
Le package \package{graphicx} propose de nombreux outils pour manipuler graphiquement du contenu, textuel ou non. Consultez le manuel\footcite{graphicx}.
\end{plusloins}

\begin{attention}
    Évidemment, il est plus correct d'insérer une légende sous une image. Nous expliquons comment faire plus loin\renvoi{legende}.
\end{attention}

\section{Insérer des graphismes et des schémas}

On peut vouloir insérer divers types de graphismes ou de schémas : graphiques statistiques, arbres généalogiques, \emph{stemma codicum} etc.
Deux solutions s'offrent à nous :
\begin{enumerate}
\item Utiliser un logiciel externe qui exporterait le graphisme dans une image, qu'on insérerait comme n'importe quelle image.
\item Utiliser les possibilités de \LaTeX et du package\package{TikZ}.\label{TikZ}
\end{enumerate}

La première solution paraît de prime abord plus simple, puisqu'elle ne nécessite pas d'apprendre de nouveaux éléments en \LaTeX. Toutefois il faut avec elle gérer plus de fichiers\footnote{Prenons un graphisme généré par un tableur : il faudra conserver la feuille du tableur plus l'image du graphisme généré par celui-ci.}. En outre, les images exportées par ces logiciels étant bien souvent de type bitmap et non pas vectoriel%
\footnote{Il existe deux manières en informatique de coder une image : en codant point par point (image bitmap) ou bien en codant simplement les formes, par exemple des cercles ou des segments (image vectorielle). La première méthode est adaptée à des images complexes, comme les photographies,  la seconde méthode est conseillée pour les images \enquote{simples} telles que les graphismes. La méthode vectorielle possède l'avantage d'offrir une image plus facilement redimensionnable  sans perte de qualité et de permettre aisément la sélection de texte.},
on perd la possibilité de sélectionner dans le PDF généré par \LaTeX les textes, et on risque d'avoir des problèmes de redimensionnement.

Nous allons donc ici vous présenter une des possibilités du package \package{TikZ} : la représentation d'une famille de manuscrits. 

Nous avons choisi cet exemple car il est assez simple, tout en étant fréquent en sciences humaines. Mais \package{TikZ} permet de faire bien plus, comme produire des diagrammes statistiques : nous renvoyons à d'autres ouvrages pour en savoir plus\footnote{Outre le manuel \cite{tikz}, on pourra lire \cites{tikzimpatient}[on pourra également consulter le site][]{tikzexample}}.
Ce package possède une syntaxe spécifique. 

\begin{attention}
    Bien que le package s'appelle \package{TikZ}, il faut dans le préambule l'appeler \emph{sans majuscules} : \cs{usepackage}\verb+{tikz}+.
 
\end{attention}


Nous disposons de six familles de manuscrits d'un même texte. La  famille A a engendré les familles B, C, D. Les familles E et F pour leur part sont issues de la famille D. 

Le code est le suivant: 
\inputminted{exemples/premierpas/nontextuels/genealogie.tex}

Ce qui donne le résultat de la figure~\ref{figure:genealogie}. Analysons le code :

\begin{figure}[h]
\centering
\begin{tikzpicture}
	\node {A}
		child { node {B}}
		child { node {C}}
		child { node {D}
			 child {
				node{E}
				}
			 child {
				node{F}
				}
			}	
	;
\end{tikzpicture}
\caption{Exemple de \emph{stemma}}
\label{figure:genealogie}
\end{figure}



\begin{description}
\item[Ligne 1] L'environnement \enviro{tikzpicture} est l'environnement utilisé pour insérer une figure TikZ. Il peut recevoir de nombreux arguments pour modifier certains aspects, tels que l'épaisseur par défaut des lignes, la taille du texte, l'orientation du schéma. Nous renvoyons à la documentation.
\item[Ligne 2] Un nœud TikZ, matérialisé  par le commande \csp{node}, est un bloc de texte. On pourrait passer des options à cette commande, afin de modifier certains aspects comme la taille du texte.
\item[Lignes 3 et 4] \verb|child| désigne ici une opération : associer un fils au nœud précédent. Ce fils est lui même un nœud (\verb|node|).
\item[Lignes 5 à 12]Même chose que précédemment, sauf que le nœud D se voit attribuer des fils (E et F). Notez le nombre d'accolades et leurs imbrications : c'est ce qui permet à \package{TikZ} de construire correctement le \emph{stemma}.
\item[Ligne 13] Le point-virgule est obligatoire après chaque commande \package{TikZ} (ici \cs{node}).
\item[Ligne 14] Fin du graphisme \package{TikZ}.
\end{description}

Évidemment un tel code est relativement complexe. C'est pourquoi nous conseillons de le commenter attentivement\footnote{Pour des raisons de place nous ne l'avons pas fait ici.} et de prêter extrêmement attention  aux accolades. Par ailleurs nous recommandons de mettre chaque graphisme \package{TikZ} dans un fichier séparé.


Notre \emph{stemma} est relativement simple : il n'y a pas de \enquote{consanguinité}. Si cela avait été le cas la construction aurait été plus complexe : on n'aurait pas pu utiliser les opérations de types \verb|child| et il aurait fallu positionner précisément les éléments en utilisant un système de coordonnées\footcite[Nous avons publié sur notre site Internet une série d'articles sur les \emph{stemma} avec TikZ :][]{stemma}.



\section{Création de tableaux}

\begin{attention}
La création de tableaux en \LaTeX nécessite  une extrême rigueur. 
C'est à nos yeux l'un des éléments les plus complexes de \LaTeX. C'est pourquoi nous ne proposerons ici qu'un bref aperçu : on trouvera aisément des exemples plus détaillés dans d'autres ouvrages ou sur Internet\footcite[On pourra consulter des fichiers assez didactiques :][]{bebert_tableaux}.
\end{attention}

\subsection{Syntaxe de base}

La création d'un tableau simple se fait grâce à l'environnement \enviro{tabular}, d'après la syntaxe suivante :

\begin{latexcode}
\begin{tabular}{<type de colonnes>}
    Cellule & Cellule & Cellule \\
    Cellule & Cellule & Cellule \\
\end{tabular}
\end{latexcode}

Le caractère \verb|&| marque un changement de cellule. Le \verb|\\| marque un changement de ligne.

L'argument \arg{type de colonnes} de l'environnement \enviro{tabular} sert à préciser :
\begin{enumerate}
    \item Le nombre de cellules.
    \item Le formatage du texte à l'intérieur des cellules.
    \item Éventuellement les séparateurs verticaux des cellules.
    
\end{enumerate}

Il existe trois types possibles de colonnes,  qui précisent l'alignement des textes : \label{typecolonnes}

\begin{description}
\item[l] Alignement à gauche.
\item[c] Centré.
\item[r] Alignement à droite.
\end{description}

Supposons que nous ayons un tableau à trois colonnes. Nous souhaitons que les cellules de la première colonne soient alignées à gauche, celles de la seconde centrées et celles de la dernière alignées à droite\footnote{Il s'agit ici d'une dérogation au principe de séparation de fond et de forme. Toutefois le lecteur pourra aisément, lorsqu'il l'aura assimilée, contourner cette dérogation et revenir dans le droit chemin.}.  
Il nous faut  écrire :

\inputminted{exemples/premierpas/nontextuels/tableau3colonnes.tex}


\begin{tabular}{l|c|r}
Première cellule à gauche 
& Première cellule au centre 
& Première cellule à droite \\
Seconde cellule à gauche 
& Seconde cellule au centre 
& Seconde cellule à droite \\
\end{tabular}



Par défaut, le tableau va prendre la largeur minimale nécessaire à son affichage. 

On peut utiliser l'environnement \enviro{tabular*} qui propose un argument supplémentaire, correspondant à la largeur totale du tableau :

\begin{latexcode}
\begin{tabular*}{<largeur du tableau>}{<type de colonnes>}
    Cellule & Cellule & Cellule \\
    Cellule & Cellule & Cellulle \\
\end{tabular}
\end{latexcode}


Toutefois il peut être pratique d'indiquer des tailles de colonnes : en particulier lorsque nos cellules contiennent des paragraphes.

Pour ce faire trois types supplémentaires de colonnes nous sont proposés par le package \package{array}\footnote{En réalité le premier type ne nécessite pas que l'on charge le package.}. Ces types, à la différences des précédents, déterminent une largeur de colonne. Il faut pour les utiliser connaître  les unités de mesure\renvoi{unite} de \LaTeX. 

Ces trois types de colonnes sont :
\begin{description}
    \item[p\marg{largeur}] texte justifié. La plus haute ligne du texte se trouve sur la ligne de base.
    \item[m\marg{largeur}] texte justifié. La ligne du milieu du texte se trouve sur la ligne de base.
    \item[b\marg{largeur}] texte justifié. La plus basse ligne du texte se trouve sur la ligne de base.
\end{description}

La ligne de base est une ligne spécifique, correspondant \emph{grosso-modo} au milieu d'une cellule. Exemple :

\inputminted{exemples/premierpas/nontextuels/tableau3colonnesarray.tex}
\begin{tabular}{p{3cm} | m{3cm} | b{3cm} | c}
Long texte dont la plus haute ligne 
se trouve sur la ligne de base.  
& 
Long texte dont la ligne du milieu 
se trouve sur la ligne de base. 
&
Long texte dont la plus basse ligne 
se trouve sur la ligne de base.
&
Ligne de base
\end{tabular}




\begin{plusloins}
Il est possible de créer ses propres types de colonnes, afin par exemple d'en avoir avec une typographie particulière. Consultez le manuel du package \package{array}\footcite{array_colonnes}.
\end{plusloins}

\begin{plusloins}
On peut utiliser le package \package{slahshbox} afin d'avoir une cellule divisée en deux dans un sens diagonal, pour les tableaux à doubles entrées.
\end{plusloins}

\begin{plusloins}
Le package \package{tabularx} propose un environnement \package{tabularx} permettant de calculer automatiquement la taille des colonnes.
\end{plusloins}

\begin{plusloins}
Il existe également un package \package{tabbing} permettant de gérer des alignements de texte sans utiliser un tableau\footcite{tabbing}.
\end{plusloins}

\begin{plusloins}
On peut simuler certaines fonctionnalités d'un tableur à l'aide du package \package{spreadtab}\footcite{spreadtab}.
\end{plusloins}

\subsubsection{Séparation des cellules}

On peut tracer des filets de séparation des cellules : pour un filet  horizontal, il suffit d'utiliser la commande \csp{hline}. Pour un filet  vertical, il faut mettre le caractère \verb+|+\footnote{Ce caractère, appelé \forme{pipe} ne doit pas être confondu avec un L minuscule.} entre les types de cellules.

On trace des filets horizontaux plus petits en utilisant la commande \csp{cline}\verb|{<col1>-<col2>}|, où \arg{col1} et \arg{col2} doivent être remplacés par les numéros de colonnes.

\inputminted{exemples/premierpas/nontextuels/tableauseparateur.tex}
\begin{tabular}{l||c|r}
\hline
A & B & C \\
\hline
A1 & B1 & B2\\

\end{tabular}


\subsubsection{Fusion de cellules}

Il est possible de fusionner des cellules. Pour une fusion de colonnes, il faut utiliser la commande suivante :
\csp{multicolumn}\marg{nombcol}\marg{type}\marg{contenu}.

\begin{description}

\item[\arg{nombcol}] est le nombre de colonnes à fusionner.
\item[\arg{type}] est le type de colonne qui en résulte, selon la syntaxe vue plus haut.
\item[\arg{contenu}] est le contenu de la ligne fusionnée.
\end{description}

Pour une fusion de lignes, il faut charger le package \package{multirow}. Celui-ci propose une commande \cs{multirow} dont voici la syntaxe simplifiée\footcite[Simplifiée car nous ne présentons pas les arguments optionnels : nous renvoyons à la documentation :][]{multirow} :

\csp{multirow}\marg{lignes}\marg{largeur}\marg{contenu}.


\begin{description}
\item[\arg{lignes}] est le nombre de lignes concernées par la fusion.
\item[\arg{largeur}] est la largeur du texte. On peut mettre \verb|*| pour laisser \LaTeX{} s'en charger.
\item[\arg{contenu}] est le contenu de la cellule issue de la fusion.
\end{description}

\inputminted{exemples/premierpas/nontextuels/tableaufusion.tex}
\begin{tabular}{|c|c|c|}
\hline
	\multicolumn{3}{|c|}{Colonnes fusionnées = ABC0} \\
	\hline
	A1 & B1 & C1 \\
	\hline
	\multirow{2}{*}{Lignes fusionnés = A2-A3}
	& B2 & C2 \\
	\cline{2-3}
	& B3 & C3 \\
\hline
\end{tabular}

On remarquera les cellules vides correspondant aux anciennes cellules A2 et A3. La commande \cs{multirow} ne modifie pas la structure du tableau : elle indique simplement que le texte se situe sur plusieurs colonnes.

\subsection{Long tableau}

Si vous fabriquez ainsi un long tableau, vous constaterez que les cellules dépassent dans la marge inférieure, jusqu'à sortir de la page, au lieu, comme il semblerait logique, de continuer sur les pages suivantes.

Pour éviter cela, il faut  utiliser l'environnement \enviro{longtable} du package éponyme\index[pkg]{longtable}.

\begin{latexcode}
\usepackage{longtable}
…
\begin{longtable}{types colonnes}
…
\end{longtable}
\end{latexcode}

La syntaxe d'un environnement \enviro{longtable} est identique à celle d'un environnement \enviro{tabular}. Toutefois le package propose deux  commandes supplémentaires pour préciser les en-têtes et pieds de tableau, qui sont répétés sur chaque pages. Ces commandes sont, respectivement, \csp{endhead} et \csp{endfoot}. Il existe aussi aussi une commande \csp{endfirsthead} pour préciser l'en-tête  spécifique à la première page du tableau et une commande \csp{endlastfoot} pour préciser le pied  spécifique à la dernière page du tableau.

Ces quatre commandes sont optionnelles, elles  doivent s'utiliser en début de tableau : il faut placer les éléments correspondants \emph{avant} chacune des commandes, de la manière suivante :

\begin{latexcode}
\begin{longtable}{types colonnes}
Cellules d'en-tête.
\endhead
Cellules d'en-tête pour la première page.
\endfirsthead
Cellules de pied.
\endfoot
Cellules de pied pour la dernière page.
\endlastfoot
Cellules de corps de tableau.
\end{longtable}
\end{latexcode}

\begin{attention}
Avec \enviro{longtable}, il faut compiler deux fois le fichier \ext{tex}. 
En effet, \XeLaTeX{} --- comme d'ailleurs \LaTeX{} --- n'arrive pas à déterminer la taille des colonnes à la première compilation : il procède à un calcul page par page. Cependant il stocke la taille maximale des colonnes dans un fichier temporaire, portant l'extension \ext{aux}. Lors de la seconde compilation, il se sert de ce fichier temporaire pour indiquer la taille des colonnes sur chacune des pages.
\end{attention}


\begin{plusloins}
Pour afficher un graphique statistique, nous recommandons d'utiliser le package \package{TikZ}\renvoi{TikZ}. Toutefois la maîtrise de cet outil peut se révéler relativement complexe. C'est pourquoi nous listons ici quelques packages utiles : 
\begin{itemize}
\item Le package \package{csvpie} propose une fonction pour tracer un diagramme circulaire à partir d'un fichier \ext{csv}\footcite[Sa documentation se trouve dans celle du package \package{csvtools} :][]{csvtools_pie}. 
\item Le package \package{datatool} propose un certain nombre de graphiques statistiques. Il faut enregistrer les données selon une forme liée au package. Nous renvoyons à sa documentation\footcite{datatool}.
\end{itemize}
\end{plusloins}


\section[La notion de flottants]{Où insérer les éléments non textuels ?  la notion de flottants}
\label{legende}
Nous avons vu comment insérer des éléments non textuels. Mais vous constaterez rapidement que la mise en forme n'est pas toujours des meilleures, l'élément s'insérant dans le texte à l'endroit précis où il a été appelé, ce qui peut entraîner des espaces blancs disgracieux.
En outre, ces éléments non textuels disposent habituellement d'une légende, que nous ne savons pas encore mettre.

Pour résoudre ces deux problèmes --- positionnement esthétique et légende --- \LaTeX utilise la notion de flottant. Un flottant est donc un élément non textuel que LaTeX essaie d'insérer au meilleur endroit du point de vue de l'esthétique et qui dispose (éventuellement) d'une légende.

Il existe deux types principaux de flottant :
\begin{itemize}
    \item Les figures, générées par l'environnement \enviro{figure}.
    \item Les tableaux, générés par l'environnement \enviro{table}.
\end{itemize}

La syntaxe pour insérer le premier type est la suivante :

\begin{latexcode}
\begin{figure}[<paramètre de placement>]
    Insertion de la figure ou de l'image suivant les codes montrés plus haut.
    \caption{<Légende>}
\end{figure} 
\end{latexcode}

Celle pour insérer le second est la suivante :

\begin{latexcode}
\begin{table}[<paramètre de placement>]
    Insertion d'un tableau suivant les codes montrés plus haut.
    \caption{<Légende>}
\end{table} 
\end{latexcode}

Si la commande \cs{caption} est insérée au-dessus de la figure, de l'image ou du tableau, la légende sera située au-dessus de l'élément et non en-dessous. Il est possible, mais déconseillé, de ne pas mettre de légende.

Nous avons un exemple de flottant   dans ce chapitre : la figure~\ref{figure:genealogie}.
\begin{attention}
    On pourrait vouloir changer les intitulés comme \emph{figure} et \emph{tableau}. Tout ceci est possible : nous en parlons plus loin, dans le passage consacré aux chaînes de langue de \package{polyglossia}\renvoi{i18npoly}.
    
    De même on pourrait souhaiter renvoyer automatiquement au numéro et la page d'une figure : nous en parlons dans le chapitre consacré à la navigation à l'intérieur d'un document\renvoi{label}.
\end{attention}

À noter que, comme les titres de partie, la commande \csp{caption} peut recevoir un argument optionnel qui compte pour titre dans la table des figures ou la table des tableaux\renvoi{tablefigure} : 

\cs{caption}\oarg{Légende pour la table}\marg{Légende}. 


\begin{plusloins}

Il est parfois  utile de centrer le contenu du flottant. Il s'agit ici d'une exception au principe de séparation du fond de la forme. Pour procéder à ce centrage, il faut utiliser la commande \csp{centering}.

\begin{latexcode}
\begin{figure}[<paramètre de placement>]
    \centering
    Insertion de la figure ou de l'image suivant les codes montrés plus haut.
    \caption{<Légende>}
\end{figure} 
\end{latexcode}


Pour personnaliser les styles des légendes, la méthode la plus simple reste encore d'utiliser le package \package{caption}. Nous renvoyons à sa documentation\footcite{caption}.
Pour utiliser une légende en dehors d'un flottant, on peut utiliser le package \package{ccaption}\footcite{ccaption}, qui permet d'autres choses encore.

Si l'on souhaite avoir deux éléments flottants en vis-à-vis, sur la page gauche et sur la page droite, on peut  se servir du package \package{dpfloat}\footcite{dpfloat}.

Le package \package{endfloat}\footcite{endfloat} permet, quant à lui, de placer systématiquement les flottants à la fin du travail.

Pour avoir des flottants orientés dans un autre sens que celui de la page, on utilisera le package \package{rotfloat}\footcite{rotfloat}.

Il est possible d'insérer au sein d'une figure des sous-figures, chacune d'entre elles disposant d'une légende, en plus de la légende principale; il faut utiliser le  \package{subcaption}\footcite{subcaption}.
\end{plusloins}



\subsection{Choix de l'emplacement du flottant}

Le \arg{paramètre de placement} indique à \LaTeX comment placer idéalement les flottants. Il s'agit d'un paramètre indicatif que le compilateur essaie autant que possible de suivre, sans pour autant être contraint. Les valeurs sont les suivantes :

\begin{longtable}{|l|l|}
    \hline
    \headlongtable{Valeur} & \headlongtable{Signification}    \\
    \hline
    \endhead
    \hline
    \endfoot
    h     & À l'emplacement de l'appel au flottant     \\
    t     & En haut d'une page                \\
    b     & En bas d'une page                \\
    p     & Sur une page dédiée aux flottants        \\
\end{longtable}


Si le système des flottants permet généralement de conserver une mise en page correcte tout en n'éloignant pas trop l'élément de son emplacement, il arrive parfois que l'éloignement soit trop important.

La commande \csp{FloatBarrier} du package \package{placeins} permet de remédier à ce problème\footcite{placeins}. 
Tous les flottants appelés avant la commande sont placés avant celle-ci.






    
