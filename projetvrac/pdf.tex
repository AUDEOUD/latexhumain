Un autre avantage de \XeLaTeX{} est la production directe d'un document au format PDF. Lorsque l'on reçoit des documents sous forme numérique, il est bien fréquent qu'à l'ouverture du fichier la mise en page soit perdue, que l'impression soit de mauvaise qualité, ou même qu'il soit tout bonnement impossible d'ouvrir le fichier.

Le \emph{Portable Document Format} permet de pallier à ces inconvénients. Il s'agit d'un format \emph{ouvert}, c'est-à-dire que son créateur, la société Adobe, a publié toutes les spécifications nécessaires à la création de logiciels pouvant lire ce format. Par conséquent, il existe de nombreux lecteurs de PDF utilisables sur un grand nombre de systèmes d'exploitations\footnote{Pour un aperçu des lecteurs libres et gratuits disponibles pour votre système d'exploitation si vous n'en disposez pas déjà d'un, rendez-vous à la page \url{http://pdfreaders.org/index.fr.html} (en français).}.

Le format PDF est conçu pour être lisible de manière universelle. Il embarque en effet dans un même fichier non seulement le texte et les éventuelles images, mais aussi les indications de mise en page et les polices de caractères. Être aussi complet vous garanti que celui qui visionnera votre document ou bien le lira sous sa forme imprimée verra exactement ce que \emph{vous} aviez en tête lorsque vous l'avez composé, ce qui constitue un avantage indéniable sur les formats de fichiers tels que celui de Microsoft Word, par exemple. En outre, vous avez la certitude que le PDF que vous conservez est pérenne. Dans la mesure où chacun est libre de concevoir un logiciel permettant la lecture des PDF et que les spécifications sont librement accessibles, vous avez la garantie de ne pas voir le sort de vos publications dépendre de la bonne volonté d'un vendeur de logiciels au cours des années à venir, ce qui fait du PDF un format de choix pour archiver des publications.

La médaille a hélas son revers: format de fichier lisible universellement, le PDF est fort difficile à modifier confortablement: il n'est pas prévu pour cet usage. C'est pourquoi, si vous souhaitez travailler en collaboration sur un ouvrage en utilisant \XeLaTeX{}, il vous faudra travailler différemment.\renvoi{travail collaboratif}