Le premier niveau correspond à des parties
\begin{enumerate}
\item Introduction
	
	\begin{enumerate}
	\item Préface
	\item Préambule : public visée, objectifs du livre			-> à refaire
	\item LaTeX vs Traitement de Texte					-> fait
		\begin{enumerate}
		\item Rédaction d'un texte vs mise en forme
		\item Inconvénients d'un traitement de texte
		\item Avantages de LaTex
		\begin{enumerate}
			\item Pour le rédacteur
			\item Pour l'éditeur
			\item Pour la manipulation du texte
		\end{enumerate}
		\item Le principe de la compilation  (? )
		\item Pourquoi (Xe)LaTex ?
		\end{enumerate}
	\item Organisation de ce livre						> dire essaie détailler, mais renvoie à d'autre manuels. + pavé du milieux
	\end{enumerate}
\item Débuter avec LaTeX
\begin{enumerate}
	\item Commencer avec LaTeX						-> fait
	\begin{enumerate}
		\item Un premier document
		\item Structure d'un document LaTeX
		\begin{enumerate}
			\item La classe d'un document
			\item L'appel aux packages
			\item Choisir le Français comme langue par défaut
			\item Le corps du document
			\item Titre, auteur et date : la notion de commande
			\item Le corps du Texte, la manière de le rédiger
			%\begin{enumerate}
			%	\item Analyse de notre exemple
			%	\item Allons plus loins
			%\end{enumerate}
			\item Un commentaire
			\item La notion d'environnement			
			\item conclusion	
		\end{enumerate}
	\end{enumerate} 
	\item{Structurer son document}						-> fait
		\begin{enumerate}	
		\item Les différents niveaux de titre
		\item Fractionner son document en plus fichiers : la notion d'inclusion
		\end{enumerate}
	\item Mettre en sens son document (1) : découverte
	\begin{enumerate}
		\item Mettre en sens vs mettre en forme
		\item Les principales commandes
		\begin{enumerate}
			\item Emph (≠ italique)					-> fait
			\item Notes de bas de pages, Notes de marge	-> fait
			\item Les listes						-> fait
		\end{enumerate}
	\end{enumerate}
	\item Mettre en sens (2) : l'art de citer					-> fait
		\begin{enumerate}
		\item Citer dans le corps du texte	
		\item Citer dans un bloc à part
		\end{enumerate}
	\item Mettre en sens (3) : inventer ses propres commandes	-> fait
		\begin{enumerate}
		\item Déclarer sa commande
		\item Élèments de mise en forme dans une commande
		\end{enumerate}
	\item Gérer les langues avec Xunicode					 > fait
		\begin{enumerate}
		\item Déclarer les changements de langues
		\item Des textes dans des caractères non latins
		\begin{enumerate}
			\item La mauvaise méthodes : changer de police
			\item La bonne méthode : Unicode
			\item Écrire de droite à gauche et en boustréphodon
		\end{enumerate}
		\end{enumerate}
	\item Des documents non textuels
		\begin{enumerate}
		\item Insérer des images						> fait
		\item Les tableaux							> fait > voir comment trier automatique
		\item Inserer des graphismes (TikZ)				> fait 3 exemples de manuscrits
		\item Gestion des flottants					> fait
		\end{enumerate}
\end{enumerate}

\item{Gérer une bibliographie}
\begin{enumerate}
\item Introduction									> fait
	\begin{enumerate}
		\item{Principe général}
		\item{Une triple compilation}
	\end{enumerate}
\item Remplir un fichier .bib 						> fait, je n'ai pas pris tout à fait ce plan
	\begin{enumerate}
	\item La notion de clef \& syntaxe de base\footnote{Renvoi vers logiciels.}
	\item Les différents champs
		\begin{enumerate}
			\item Les champs de personnes
			\item Les champs standards : titre, éditeur commerciale etc.
			\item Les champs pour la pagination
			\item Les champs pour l'indexation
			
			\item Les champs d'organisation
			\item Les champs personalisés				> laissé tomber ou faire remarquer
		\end{enumerate} 
	\end{enumerate}
\item{Introduire des références bibliographiques}
	\begin{enumerate}
	\item Le Package BibLaTex et ses différentes options de styles
	\item Les commandes de citations simples
	\item Les commandes de citations mutltiples
	\end{enumerate}
\item Une bibliographie							-> fait
	\begin{enumerate}
	\item Commande de base
	\item Personaliser l'entête
	\item Trier la bibliographie : sous parties, distinction sources primaires vs secondaires etc.
	\item La commande printshorthands				> elle ne sert à rien, je documente pas
	\end{enumerate}

\item Personaliser l'affichage des références bibliographiques	-> fait
	\begin{enumerate}
	\item Les commandes de styles
	\item La notion de macro bibliographique : un exemple : n'afficher qu'une fois la pagination.
	\item Créer un fichier de style pour son journal (usage avancée)	> en notes, en remarques
	\end{enumerate}	
\end{enumerate}

\item Des outils pour naviguer
\begin{enumerate}
\item Des renvois internes					> fait
\item Un sommaire / table des matières			> ne pas parler de minitoc
\item Des index							> en cours
\end{enumerate}

\item Outils supplémentaires pour les sciences humaines
\begin{enumerate}
\item Bibref : gérer les citations bibliques			> faire avec splitindex ?
\item Bredele : une classe spécialement prévues pour les sciences humaines >non
\item Ledmac pour des éditions critiques			> Enimie	
\item Ledpar en //							> Enimie
\item Introduction à Beamer					> fait
\end{enumerate}

\item Gérer finement le rendu final
\begin{enumerate}
\item Gestion de la taille du papier et des marges		> inutile
\item Personaliser l'apparence des titres			> oui
\item En-tête et pieds de pages					> oui
\item Page de garde						> on prend les styles de bredele
\item Styles d'index et styles de	sommaires			> fait sauf sommaire mais à relire

\end{enumerate}

\item{Annexes}
\begin{enumerate}
\item Installer LaTex								=> fait
\item Installer / mettre à jour des packages							-> à faire pour mac / windows. Pour Linux sans doute préciser qu'il faut utiliser les logiciels de gestions.
\item Les principaux éditeurs							-> fait mais peut être à dev
\item Logiciels de gestions bibliographiques					-> fait mais peut être à commente	\item Les unités de mesures en LaTeX						> fait + redefinir longueur + hspace
\item Suivi des révisions et travail collaboratif : introduction à SVN	> fait
\item Comptage des mots et des caractèes					> ???
\item Trouver de l'aide + FAQ Anglaise
\item Index : packages, commande
\item Glossaire des notions :
	- macro
	- envrionnement
	-commande
	-package
	- argument
	- clef bibliographique et clef d'index
	- champ (?)
	- preambule
	- commentaires
\item Bibliographie : texdoc + man.
	- 1. Ouvrage généraux LaTeX
	- 2. LaTeX et SHS
	- 3. fiches et tutoriels
\end{enumerate}
\end{enumerate}
