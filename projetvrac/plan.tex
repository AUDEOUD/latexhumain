Le premier niveau correspond à des parties
\begin{enumerate}
\item Introduction
	
	\begin{enumerate}
	\item Préface
	\item Préambule : public visée, objectifs du livre			-> à refaire
	\item LaTeX vs Traitement de Texte
		\begin{enumerate}
		\item Rédaction d'un texte vs mise en forme
		\item Inconvénients d'un traitement de texte
		\item Avantages de LaTex
		\begin{enumerate}
			\item Pour le rédacteur
			\item Pour l'éditeur
			\item Pour la manipulation du texte
		\end{enumerate}
		\item Le principe de la compilation  (? )
		\item Pourquoi (Xe)LaTex ?
		\end{enumerate}
	\item Organisation de ce livre
	\end{enumerate}
\item Débuter avec LaTeX
\begin{enumerate}
	\item Commencer avec LaTeX						-> fait
	\begin{enumerate}
		\item Un premier document
		\item Structure d'un document LaTeX
		\begin{enumerate}
			\item La classe d'un document
			\item L'appel aux packages
			\item Choisir le Français comme langue par défaut
			\item Le corps du document
			\item Titre, auteur et date : la notion de commande
			\item Le corps du Texte, la manière de le rédiger
			%\begin{enumerate}
			%	\item Analyse de notre exemple
			%	\item Allons plus loins
			%\end{enumerate}
			\item Un commentaire
			\item La notion d'environnement
			\item conclusion	
		\end{enumerate}
	\end{enumerate} 
	\item{Structurer son document}						-> fait
		\begin{enumerate}	
		\item Les différents niveaux de titre
		\item Fractionner son document en plus fichiers : la notion d'inclusion
		\end{enumerate}
	\item Mettre en sens son document (1) : découverte
	\begin{enumerate}
		\item Mettre en sens vs mettre en forme
		\item Les principales commandes
		\begin{enumerate}
			\item Emph (≠ italique)					-> fait
			\item Notes de bas de pages, Notes de marge	-> fait
			\item Les listes						-> fait
		\end{enumerate}
	\end{enumerate}
	\item Mettre en sens (2) : l'art de citer					-> fait
		\begin{enumerate}
		\item Citer dans le corps du texte	
		\item Citer dans un bloc à part
		\end{enumerate}
	\item Mettre en sens (3) : inventer ses propres commandes	-> fait
		\begin{enumerate}
		\item Déclarer sa commande
		\item Élèments de mise en forme dans une commande
		\end{enumerate}
	\item Gérer les langues avec Xunicode					 > fait
		\begin{enumerate}
		\item Déclarer les changements de langues
		\item Des textes dans des caractères non latins
		\begin{enumerate}
			\item La mauvaise méthodes : changer de police
			\item La bonne méthode : Unicode
			\item Écrire de droite à gauche et en boustréphodon
		\end{enumerate}
		\end{enumerate}
	\item Des documents non textuels
		\begin{enumerate}
		\item Insérer des images						> fait
		\item Les tableaux							> en cours	
		\item Inserer des graphismes (TikZ)				> fait mais peut être à develloper
		\item Gestion des flottants					> fait
		\end{enumerate}
\end{enumerate}

\item{Gérer une bibliographie}
\begin{enumerate}
\item Introduction
	\begin{enumerate}
		\item{Principe général}
		\item{Une triple compilation}
	\end{enumerate}
\item Remplir un fichier .bib
	\begin{enumerate}
	\item La notion de clef \& syntaxe de base\footnote{Renvoi vers logiciels.}
	\item Les différents champs
		\begin{enumerate}
			\item Les champs de personnes
			\item Les champs standards : titre, éditeur commerciale etc.
			\item Les champs pour la pagination
			\item Les champs pour l'indexation
			\item Les champs pour le tri
			\item Les champs d'organisation\footnote{Trouver un titre plus générique} : Library, Keywords, Annotation etc.
			\item Les champs personalisés
		\end{enumerate} 
	\end{enumerate}
\item{Introduire des références bibliographiques}
	\begin{enumerate}
	\item Le Package BibLaTex et ses différentes options de styles
	\item Les commandes de citations simples
	\item Les commandes de citations mutltiples
	\end{enumerate}
\item Une bibliographie							-> fait
	\begin{enumerate}
	\item Commande de base
	\item Personaliser l'entête
	\item Trier la bibliographie : sous parties, distinction sources primaires vs secondaires etc.
	\item La commande printshorthands				> elle ne sert à rien, je documente pas
	\end{enumerate}

\item Personaliser l'affichage des références bibliographiques
	\begin{enumerate}
	\item Les commandes de styles
	\item La notion de macro bibliographique : un exemple : n'afficher qu'une fois la pagination.
	\item Créer un fichier de style pour son journal (usage avancée)
	\end{enumerate}	
\end{enumerate}

\item Des outils pour naviguer
\begin{enumerate}
\item Des renvois internes
\item Un sommaire
\item Des index
\end{enumerate}

\item Outils supplémentaires pour les sciences humaines
\begin{enumerate}
\item Bibref : gérer les citations bibliques
\item Bredele : une classe spécialement prévues pour les sciences humaines
\item ? pour des éditions critiques
\item Introduction à Beamer
\item Xargs
\end{enumerate}

\item Gérer finement le rendu final
\begin{enumerate}
\item Gestion de la taille du papier
\item Personaliser l'apparence des titres
\item En-tête et pieds de pages
\item Créer ses environnements
\end{enumerate}

\item{Annexes}
\begin{enumerate}
\item Installer LaTex
\item Installer des packages
\item Les principaux éditeurs
\item Logiciels de gestions bibliographiques
\item Logiciels WYSIWYG pour faire des graph.
\item Logiciels pour prise de notes (?)
\item Les erreurs de compilations
\item Les unités de mesures en LaTeX
\item Suivi des révisions et travail collaboratif : introduction à SVN
\item Index : packages, commande
\item Glossaire des notions
\item Bibliographie
\end{enumerate}
\end{enumerate}