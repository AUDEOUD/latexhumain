Sous Windows, le terminal s'appelle \endquote{Invite de commandes}. Le langage
est différent de celui que l'on trouve chez les systèmes de type Unix comme
GNU/Linux et Mac OS X. On trouvera cependant quelques similarités.

Pour démarrer une invite de commande, pressez simultanément la touche \verb|Windows| de votre
clavier et la touche \verb|R|. Dans l'invite qui s'ouvre tapez \verb|cmd| puis
\verb|Entrée|. Voilà face à une console.

Le principe est le même que pour une console sous Unix: il va vous falloir
naviguer entre des répertoires et exécuter des commandes. Toute commande, une
fois tapée, est validée par une pression sur la touche \verb|Entrée|.

Le répertoire courant est indiqué à gauche du curseur clignotant.
La commande \verb|dir| vous indique ce qui se trouve dans le répertoire
courant.

Pour vous déplacer dans un répertoire, il suffit de taper la commande
\verb|cd| suivie du dossier où vous souhaitez vous rendre.

Ainsi la commande \verb|cd projet-latex| vous fera pénétrer dans le
répertoire \verb|projet-latex|, ce que vous pouvez vérifier avec la
commande \verb|dir|. Pour vous déplacer dans un répertoire parent, tapez
simplement \verb|cd ..|.

Lorsque vous êtes dans le répertoire où vous souhaitez exécuter une
commande, vous pouvez la lancer de la même manière:

\begin{quote}
\verb|xelatex nomdufichieràcompiler.tex|
\end{quote}

