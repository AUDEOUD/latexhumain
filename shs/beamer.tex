\chapter{Présentations avec \LaTeX{} : \classenoidx{Beamer}}\label{beamer}\sindex[classe]{beamer}

\begin{intro}
Si \LaTeX{} permet d'écrire du texte, il offre également la possibilité de créer facilement des diapositives de présentation pour un exposé ou une conférence\footnote{Pour remplacer Powerpoint ou OpenOffice Impress}. Ceci se fait à l'aide d'une classe spécifique : \classe{Beamer}.

Nous présenterons ici les bases de Beamer afin de bien préparer des supports de présentation. Nous renvoyons à d'autres ouvrages pour plus de détails\footcites[On consultera pour une introduction plus approfondie][]{intro_beamer}[pour les usages avancées on se servira du manuel : ][]{beamer}.

Le PDF résultant de la compilation d'un fichier Beamer peut être affiché sous forme de diaporama contrôlable à l'aide des logiciels classiques de lecture de PDF.
\end{intro}

\section{Préambule et premieres diapositives}

\classe{Beamer} étant une classe \LaTeX{}, le document doit commencer par :

\begin{latexcode}
\documentclass{beamer}
\end{latexcode}

Le reste du préambule contient le même contenu qu'un fichier \ext{tex} classique : appels aux packages, définitions d'éventuelles commandes etc.

Toutefois il est possible de définir la manière dont les diapositives vont s'afficher, en utilisant dans le préambule la commande \csp{usetheme}. Par exemple pour utiliser le thème \forme{Darmstadt} :

\begin{latexcode}
\usetheme{Darmstadt}
\end{latexcode}

Un certains nombres de thèmes sont décrits dans le manuel de Beamer\footcite{beamer_theme}.

Avec Beamer, l'unité de base  est la diapositive, appelée \textenglish{\emph{frame}}. Une diapositive se crée avec l'environnement \emph{frame} et peut contenir n'importe quel contenu \LaTeX{}.
Par exemple :

\begin{latexcode}
\begin{frame}
    \begin{itemize}
        \item Élément 1
        \item Élément 2
        \item Élément 3
        \item Etc.
    \end{itemize}
\end{frame}
\end{latexcode}

À noter que l'environnement  \enviro{frame} peut être remplacé par la commande éponyme\sindex[cs]{frame@\oldcs{frame}|hyperpagetextbf} :

\begin{latexcode}
\frame{
    \begin{itemize}
        \item Élément 1
        \item Élément 2
        \item Élément 3
        \item Etc.
    \end{itemize}
}
\end{latexcode}

\section{Diapositives de titre}

Il est possible de structurer le document Beamer grâce aux commandes \cs{section} et \cs{subsection}\renvoi{niveautitre}\footcite[Le manuel de \classe{Beamer} déconseille d'utiliser  \cs{subsubsection} parce que \enquote{c'est le diable}. \\emph{Cf}. :][]{beamer_diable} et d'utiliser dans le préambule les commandes de description : \cs{title}, \cs{author} et \cs{date}\renvoi{notioncommande}.

Toutefois ces informations servent uniquement à structurer le document, afin de permettre d'afficher le plan. C'est pourquoi il faut les entrer \emph{en-dehors} des diapositives\footnote{Un  autre intérêt est aussi de permettre d'avoir un seul document pour une version \forme{article} et une version \forme{diapositive} d'un même sujet.}.

En revanche, pour les afficher dans une diapositive, il faut utiliser les commandes \csp{sectionpage} et \csp{subsectionpage} pour les niveaux de titres et \csp{titlepage} pour les informations sur la date, l'auteur et le titre de la conférence.

\inputminted{exemples/shs/beamer/structure.tex}

\section{Affichage différé : les couches Beamer}

Afficher tous les éléments d'une diapositive en une seule fois présente peu d'intérêt : il est plus intéressant de les afficher petit à petit afin de les commenter au fur et à mesure. \classe{Beamer} utilise pour cela la notion de \forme{\textenglish{slides}}, que nous traduisons par celle de \emph{couches}. Il s'agit de dire à une commande de ne s'exécuter que sur une ou plusieurs couches spécifiques.

La syntaxe de l'affichage couche par couche (en anglais \textenglish{\emph{overlay specification}} est  : \verb|<a-b>| où \verb|a| désigne la couche de départ et \verb|b| désigne la couche d'arrivée. On peut également séparer des numéros de couches spécifiques par des virgules. Voici des exemples pour mieux comprendre :

\begin{latexcode}
\begin{frame}
\begin{itemize}
    \item<1>Élément affiché uniquement sur la première couche.
    \item<2->Élément affiché à partir de la deuxième couche.
    \item<2-4>Élément affiché sur les couches 2 à 4.
    \item<-4>Élément affiché jusqu'à la couche 4, incluse.
    \item<1,3-5>Élément affiché sur les couches 1 et 3 à 5.
\end{itemize}
\end{frame}
\end{latexcode}

Toutes les commandes \LaTeX{} ne réagissent pas nécessairement aux spécifications de couche : par exemple la commande \cs{emph} ne réagit pas : si vous utilisez une spécification de couche avec cette commande, la spécification apparaît sur la diapositive.

On peut vouloir obtenir un PDF où toutes les couches d'une diapositive apparaissent d'un seul coup, afin par exemple d'avoir une version papier\footcite[À noter qu'il est possible de fabriquer un PDF destiné à l'impression sur transparents -- en cas de défaillance du vidéo-projecteur -- en précisant quelles couches doivent être séparées sur les transparents. Voir :][]{beamer_trans}. Pour ce faire, il suffit d'utiliser l'option de package \option{handout}.

\begin{latexcode}
\documentclass[handout]{beamer}
\end{latexcode}

\package{Beamer} propose des commandes et des environnements pour des usages avancés des spécifications de couche, notamment pour remplacer des éléments par d'autres éléments, tout en accordant à chacun la même place sur la diapositive, pour éviter d'avoir des \enquote{effets de déplacement} d'un élément d'une couche à l'autre. 
Ceci dépassant cette introduction, nous renvoyons au manuel\footcite{beamer_overlays}.
\section{Mise en valeur}

\subsection{Blocs}

Il est possible de mettre en valeur certains éléments de plusieurs manières. La première  consiste à les mettre dans des blocs pour les encadrer. \classe{Beamer} propose plusieurs blocs standards, qui prennent à la fois un titre et un contenu. Ces blocs\footcites[Il est possible de créer ses propres blocs, voir : ][]{beamer_box}[il existe aussi des éléments proches des blocs que sont les encadrés, pour mettre en valeur des définitions, des théorèmes, voir:][]{beamer_theorems} sont  : bloc standard, bloc d'exemple et bloc d'alerte.

\begin{latexcode}
\begin{block}{Un bloc normal} 
    Un texte à encadrer
\end{block}
\begin{alertblock}{Attention} 
    Ce texte est important
\end{alertblock}
\begin{exampleblock}{Exemple} 
    Un exemple des propos tenus
\end{exampleblock}
\end{latexcode}

Évidemment, il est possible de spécifier des niveaux de couches; pour ce faire la spécification doit se trouver après le titre du bloc :

\begin{latexcode}
\begin{alertblock}{Attention}<2>
    Un message d'alerte sur la couche 2.
\end{alertblock}
\end{latexcode}

L'apparence des blocs dépend du thème choisi.

\subsection{La commande alert}

La commande \csp{alert} sert à mettre en valeur une partie d'un texte. Elle peut recevoir une spécification de couche. Cela est très utile pour indiquer à l'auditoire le sujet dont on parle. 

\begin{latexcode}
\begin{frame}
\begin{description}
    \item\alert<1>{Nous parlons d'abord d'une chose}
    \item\alert<2>{Nous parlons ensuite d'une autre}
    \item\alert<3>{Et enfin d'une troisième}
\end{description}
\end{frame}
\end{latexcode}

\begin{plusloins}

Le package \package{spot} permet des mises en valeurs sous forme de tache lumineuse.

\end{plusloins}
\section{Notes de conférence}

Il peut être utile pour le conférencier d'avoir, en plus des diapositives, des notes pour lui servir d'aide-mémoire au cours de son exposé.

Pour ce faire il suffit d'utiliser la commande \csp{note} dans une diapositive.

\begin{latexcode}
\begin{frame}
\note{Un élément à se rappeler pour le signaler au public}.
    
Le contenu proprement dit de la diapositive.
\end{frame}
\end{latexcode}

On peut afficher plusieurs notes sous forme de liste en passant  à la commande \csp{note} un argument optionnel ayant pour valeur \contenuarg{item}.

\begin{latexcode}
\begin{frame}
\note[item]{Un élément à se rappeler pour le signaler au public}.
\note[item]{Un autre  élément}
    
Le contenu proprement dit de la diapositive.
\end{frame}
\end{latexcode}


Les notes n'apparaissent pas sur la diapositive, mais sont reportées après, dans une page spéciale du fichier PDF. Cela n'est pas très pratique : on souhaiterait avoir d'une part les diapositives, d'autre part les notes. 

Pour ce faire il suffit d'utiliser dans le préambule la commande \csp{setbeameroption} avec le bon argument.

\begin{latexcode}
\setbeameroption{hide notes}
\end{latexcode}

Signifie que nous produisons un PDF contenant uniquement les diapositives. \emph{A contrario} :

\begin{latexcode}
\setbeameroption{show only notes}
\end{latexcode}

indique que nous désirons un fichier ne contenant que les notes.

Il est également possible de produire un fichier contenant les diapositives et les notes, mais capable, avec le bon logiciel, de les envoyer sur deux écrans différents. Nous renvoyons au manuel pour plus de détails\footcite{beamer_2ecrans}.

\section{Écrire son article dans le même fichier que la présentation}

Il peut être judicieux d'écrire l'article correspondant à la présentation dans le même fichier que celui-ci. Pour ce faire, la solution est  simple : il suffit d'écrire le contenu de l'article entre les environnements (ou les commandes) \enviro{frame}\footcites[À noter qu'on peut aussi introduire des miniatures des diapositives  dans l'article, voir :][]{beamer_diapo_article}.

\begin{latexcode}
\section{Introduction}

Introduction de notre article
\begin{frame}
    Diapositive pour l'introduction de la présentation
\end{frame}

\section{Première partie}

Texte de la première partie
\begin{frame}
    Diapositive pour la première partie
\end{frame}

etc.
\end{latexcode}

\subsection{Diapositives ou article ?}

Par défaut, Beamer va afficher les textes situés entre les diapositives dans des diapositives spécifiques. Pour éviter cela, il suffit de passer l'option \option{ignorenonframetext} lors de l'appel de la classe.

\begin{latexcode}
\documentclass[ignorenonframetext]{beamer}
\end{latexcode}

Pour obtenir uniquement la version article, il suffit de changer la classe du document et de charger le package \package{beamerarticle}.

\begin{latexcode}
\documentclass{article}
\usepackage{beamerarticle}
\end{latexcode}

\subsection{Automatiser les réglages}\renvoi{inclusion}

Plutôt que de changer systématiquement les premières lignes du document, autant utiliser les possibilités d'inclusion de fichier offertes par \LaTeX{}. 

Pour ce faire, créer trois fichiers :  \fichier{article.tex}, \fichier{presentation.tex} et \fichier{contenu.tex}.
Dans \fichier{article.tex} mettre :

\begin{latexcode}
\documentclass{article}
\usepackage{beamerarticle}
\input{contenu}
\end{latexcode}

Dans \fichier{presentation.tex} mettre :

\begin{latexcode}
\documentclass[ignorenonframetext]{beamer}
\input{contenu}
\end{latexcode}

Dans \fichier{contenu.tex} mettre les appels aux packages communs aux deux versions ainsi que le contenu de la présentation / de l'article.



