\chapter{Gestion des références bibliques avec \packagenoidx{Bibleref}}\sindex[pkg]{Bibleref}

\begin{prealable}

On peut évidemment considérer la Bible comme une référence bibliographique comme une autre, et inclure ses livres dans la bibliographie générale.
Toutefois, la présentation d'une référence biblique diverge bien souvent de celle d'une référence bibliographique.
C'est pourquoi on peut utiliser le package \package{bibleref} et sa traduction \package{bibleref-french}.
\end{prealable}


\section{Chargement du package et options}

Pour que nos références bibliques soient en français, il suffit simplement d'appeler le package \package{bibleref-french}. Celui-ci charge alors le package \package{bibleref}.


\begin{minted}{latex}
\usepackage{bibleref-french}
\end{minted}

\subsection{Forme catholique ou protestante}
Le livre que les protestants appellent \forme{Esaïe} est appelé \forme{Isaïe} chez les catholiques. Pour choisir la forme protestante, il suffit de passer l'option \option{protestant} à l'appel du package. Le package uniformise les références.

\begin{minted}{latex}
\usepackage[protestant]{bibleref}
\end{minted}

\subsection{Styles de références}

Les éditions de la Bible divergent parfois quant à la manière d'afficher une référence biblique. En outre, les références peuvent s'afficher tantôt en version abrégée, tantôt en version complète. C'est pourquoi il existe  plusieurs styles de références bibliques. 

On choisit son style par la commande \csp{biblerefstyle}\marg{style}, qu'il est conseillé de mettre en préambule.


La liste des styles est présentée dans le tableau ci-dessous : 

\begin{longtable}{|l|p{0.5\textwidth}|}
\hline
Style & Exemple\\
\hline
\endhead
\hline
\endfoot
default & \biblerefstyle{default}\bibleverse{IICor}(12:1-5,7) ; \bibleverse{Qo}(1:3)(2:5) ; \bibleverse{Is}(10:)\\
jerusalem & \biblerefstyle{jerusalem}\bibleverse{IICor}(12:1-5,7) ; \bibleverse{Qo}(1:3)(2:5) ; \bibleverse{Is}(10:)\\
colombe & \biblerefstyle{colombe}\bibleverse{IICor}(12:1-5,7) ; \bibleverse{Qo}(1:3)(2:5) ; \bibleverse{Is}(10:)\\
colombefull & \biblerefstyle{colombefull}\bibleverse{IICor}(12:1-5,7) ; \bibleverse{Qo}(1:3)(2:5) ; \bibleverse{Is}(10:)\\
NBS & \biblerefstyle{NBS}\bibleverse{IICor}(12:1-5,7) ; \bibleverse{Qo}(1:3)(2:5) ; \bibleverse{Is}(10:)\\
NBSfull & \biblerefstyle{NBSfull}\bibleverse{IICor}(12:1-5,7) ; \bibleverse{Qo}(1:3)(2:5) ; \bibleverse{Is}(10:)\\
TOB & \biblerefstyle{TOB}\bibleverse{IICor}(12:1-5,7) ; \bibleverse{Qo}(1:3)(2:5) ; \bibleverse{Is}(10:)\\
TOBfull & \biblerefstyle{TOBfull}\bibleverse{IICor}(12:1-5,7) ; \bibleverse{Qo}(1:3)(2:5) ; \bibleverse{Is}(10:)\\
text & \biblerefstyle{text}\bibleverse{IICor}(12:1-5,7) ; \bibleverse{Qo}(1:3)(2:5) ; \bibleverse{Is}(10:)\\
\end{longtable}

En l'absence de style précisé, c'est le style \forme{default} qui l'emporte.
Il est évidemment possible de créer ses propres styles : nous renvoyons au manuel de \package{bibleref} pour en savoir plus\footcite{bibleref-styles}.


\section{Référencer un passage biblique}

La syntaxe pour référencer un passage biblique varie selon que l'on souhaite référencer un livre en entier, un chapitre ou certains versets seulement. Voici l'ensemble des syntaxes disponibles\footnote{Nous ne faisons ici que reprendre le manuel.}

\begin{longtable}{|l|l|}
Syntaxe & Résultat \\
\hline
\endhead
\hline
\endfoot
\begin{english}\verb|\bibleverse{Ex}|\end{english} & \bibleverse{Ex}\\
\begin{english}\verb|\bibleverse{Ex}(20:)|\end{english} & \bibleverse{Exodus}(20:)\\
\begin{english}\verb|\bibleverse{Ex}(20:17)|\end{english} & \bibleverse{Exod}(20:17)\\
\begin{english}\verb|\bibleverse{IICo}(12:21)|\end{english} & \bibleverse{IICo}(12:21)\\
\begin{english}\verb|\bibleverse{IICor}(12:21-32)|\end{english} & \bibleverse{IICor}(12:21-32)\\
\begin{english}\verb|\bibleverse{Ex}(20:17)(21:3)|\end{english} & \bibleverse{Ex}(20:17)(21:3)\\
\begin{english}\verb|\bibleverse{Ex}(15:)(17:)(20:)|\end{english} & \bibleverse{Ex}(15:)(17:)(20:)\\
\begin{english}\verb|\bibleverse{Ap}(1:2,5,7-9,11)|\end{english} & \bibleverse{Rev}(1:2,5,7-9,11)\\
\begin{english}\verb|\bibleverse{ICh}(1:3)-(2:7)|\end{english} &
\bibleverse{IChronicles}(1:3)-(2:7)\\
\begin{english}\verb|\bibleverse{Lc}(1,4-5,8:)|\end{english} &\bibleverse{Luke}(1,4-5,8:)
\end{longtable}

Voici la liste des abréviations de livre.

\begin{longtable}{|l|l|}
\hline
Abréviation & Livre
\hline
\endhead
\hline
\endfoot
Gn	&	Genèse\\
Ex	&	Exode\\
Lv	&	Lévitique\\
Nb	&	Nombres\\
Dt	&	Deutéronome\\
Jos	&	Josué\\
Jg	&	Juges\\
Rt	&	Ruth\\
IS	&	1 Samuel\\
IIS	&	2 Samuel\\
IR	&	1 Rois\\
IIR	&	2 Rois\\
ICh	&	1 Chroniques\\
IICh	&	2 Chroniques\\
Esd	&	Esdras\\
Ne	&	Néhémie\\
Est	&	Esther\\
Tb	&	Tobit\\
Jdt	&	Judith\\
IM	&	1 Maccabées\\
IIM	&	2 Maccabées\\
Jb	&	Job\\
Ps	&	Psaumes\\
Pr	&	Proverbes\\
Qo	&	Qohéleth\\
Ct	&	Cantiques des cantiques\\
Sg	&	Sagesse\\
Si	&	Siracide\\
Is	&	Isaïe\\
Es	&	Esaïe\\
Jr	&	Jérémie\\
Lm	&	Lamentations\\
Ba	&	Baruch\\
Ez	&	Ezéquiel\\
Dn	&	Daniel\\
Os	&	Osée\\
Jl	&	Joël\\
Am	&	Amos\\
Ab	&	Abdias\\
Jon	&	Jonas\\
Mi	&	Michée\\
Na	&	Nahum\\
Ha	&	Habaquq\\
So	&	Sophonie\\
Ag	&	Aggée\\
Za	&	Zacharie\\
Ml	&	Malachie\\
Mt	&	Matthieu\\
Mc	&	Marc\\
Lc	&	Luc\\
Jn	&	Jean\\
Ac	&	Actes\\
Rm	&	Romains\\
ICo	&	1 Corinthiens\\
IICo	&	2 Corinthiens\\
Ga	&	Galates\\
Ep	&	Éphésiens\\
Ph	&	Philippiens\\
Col	&	Colossiens\\
ITh	&	1 Thessaloniciens\\
IITh	&	2 Thessaloniciens\\
ITm	&	1 Timothée\\
IITm	&	2 Timothée\\
Tt	&	Tite\\
Phm	&	Philémon\\
He	&	Hébreux\\
Jc	&	Jacques\\
1P	&	1 Pierre\\
2P	&	2 Pierre\\
IJn	&	1 Jean\\
IIJn	&	2 Jean\\
IIIJn	&	3 Jean\\
Jude	&	Jude\\
Ap	&	Apocalypse\\
\end{longtable}

\begin{plusloins}
Les livres proposées correspondent au canon catholique. On peut ajouter d'autres livres : nous renvoyons à la documentation de \package{bibleref}\footcite{bibleref_nouveauxlivres}.


\end{plusloins}

\section{Indexer une référence}

On peut remplacer la commande \cs{bibleverse} par la commande \cs{ibibleverse}. Celle-ci en plus d'afficher une référence biblique l'indexe.

\subsection{Un index à part}

Par défaut, les références sont indexées dans l'index principal. Il est toutefois possible de dire de les indexer ailleurs.

Pour ce faire, avec \package{splitindex}\renvoi{splitindex}, il faut d'abord déclarer un nouvel index.

\begin{minted}{latex}
\newindex[Index des références bibliques]{bible}
\end{minted}

Il faut ensuite redéfinir la commande \cs{biblerefindex} de la manière suivante :
\begin{minted}{latex}
\renewcommand{\biblerefindex}[0]{\sindex[sources]}
\end{minted}

Et à l'endroit où l'on souhaite afficher cet index :
\begin{minted}{latex}
\printindex[bible]
\end{minted}

\section{Espaces insécables}

Le package \package{bibleref-french} prévoit d'insérer des espaces insécables :
\begin{enumerate}
\item Entre le nombre et le nom quand le nom d’un livre contient un nombre ;
\item Entre le nom du livre et le numéro de chapitre ;
\item Après un tiret demi-cadratin (–) dans un intervalle de versets ; 
\item Après les virgules de séparation.
\end{enumerate}

Si beaucoup de références sont indiquées, cela peut amener à des fins de lignes dépassante.
Pour éviter cela, on peut autoriser temporairement la coupure des espaces entre un livre et un numéro de chapitre. La commande \cs{BRallowhypbch} marque le début de l'autorisation ; \cs{BRforbidhypbch} en marque la fin.

