\chapter{Édition critique et bilingue : les packages \packagenoidx{ledmac}\sindex[pkg]{ledmac} et \packagenoidx{ledpar}\sindex[pkg]{lepar}}
\label{ledmac}
\sindex[pkg]{verse}

attention > version donnée par la textLive 2011 n'est pas la bonne: il faut mettre à jour  (aller sur site du ctan)

\section{Faire une édition critique avec \package{ledmac}}

\subsection{Numéroter un texte}

Le texte dont on veut numéroter les lignes est encadré entre les commandes \cs{beginnumbering} et \cs{endnumbering}. Il faut ensuite faire une distinction entre les textes en prose et les textes en vers. 

\subsubsection{Numéroter un texte en prose}


On débute la numérotation, au sein de cet environnement numbering, avec la commande \cs{pstart} et on la termine avec la commande \cs{pend} (le texte contenu entre ces deux commande est mis dans une \enquote{boite}), comme dans l'exemple suivant:

\begin{minted}{latex}
\beginnumbering
\pstart %commencer la numérotation
Lorem ipsum dolor sit amet, consectetuer adipiscing elit ?
Morbi commodo ; ipsum sed pharetra gravida !
Nullam sit amet enim. Suspendisse id : velit vitae ligula.
Aliquam erat volutpat.

Sed quis velit. Nulla facilisi. Nulla libero. 
Quisque facilisis erat a dui. 
Nam malesuada ornare dolor.
Cras gravida, diam sit amet rhoncus ornare, 
erat  elit consectetuer erat, id egestas pede nibh eget odio.
\pend %terminer la numérotation
\endnumbering
\end{minted}

On obtiens ainsi : 

\begin{minipage}{10cm}
\beginnumbering
\pstart %commencer la numérotation
Lorem ipsum dolor sit amet, consectetuer adipiscing elit ?
Morbi commodo ; ipsum sed pharetra gravida !
Nullam sit amet enim. Suspendisse id : velit vitae ligula. 
Aliquam erat volutpat.\pend

Sed quis velit. Nulla facilisi. Nulla libero. 
 Quisque facilisis erat a dui. % onreprend la numérotation
Nam malesuada ornare dolor.
Cras gravida, diam sit amet rhoncus ornare, 
erat  elit consectetuer erat, id egestas pede nibh eget odio.
\pend
\endnumbering
\end{minipage}

%la doc dit qu'il faut à chaque paragraphe remettere pstart/pend , et qu'on peut empêcher cette inconvénient en créant un groupe et avec la commande autopar. Pourtant marche très bien sur plusieurs paragarphes à la suit sans passer par cette commande.. à voir!

Si l'on veut interrompre puis reprendre la numérotation où on l'a interrompu, il suffit de faire deux \enquote{boites} encadrée entre \cs{pstart} et \cs{pend}:  tout ce qui sera entre ces deux boites ne sera pas numéroté.

\subsubsection{\package{ledmac} et les vers}

La  numérotation dans les textes en prose ne se fait pas de la même façon. Les commandes \cs{pstart} et \cs{pend}. A la place, on découpe son texte en strophes (\enquote{stanza}). Pour commencer la numérotation, on met donc la commande \cs{stanza}. Chaque vers se termine par un & , et le dernier vers de la strophe par \cs{&}.  Mais il faut aussi indiquer l'indentation de chaque vers de la strophe (faute de quoi on obtiens le message \enquote{Missing number}). On utilise pour cela la commande \cs{setstanzaindents}\marg{$n_1 n_2 n_x$}. L'argument \arg{$n_1$} indique l'indentation du rejet si le vers est trop long pour tenir sur une seule ligne; \arg{$n_1$} correspond au premier vers, \arg{$n_2$} au deuxième et ainsi de suite.


exemple: 

\begin{minted}{latex}
\beginnumbering
\setstanzaindents{0,0,4,0,0,0,2,2,2,8}
\let\endstanzaextra=\bigbreak
\stanza
Berlin setz an.&
Es Speien die Geschäfte&
die wackern Knaben und die Mädchen aus.&
Die großen Fraun sind ganz auf neu gemalen.&
Wer wird heut abend wohl dem zimt bezahlen?&
Sie lächeln lieb. Das Auto summt heran.&
Berlin setz an.\&
\stanza
Berlin brummt auf.&
Wo ich die Paare anseh:&
Hier wird ein harter Dienst straff absolviert.&
Ein Riesenrummel von Grünau bis Wannsee ---&
und alles tadellos organisiert.&
Um jeden Schnapstich fühlst du es bestätigt:&
Marie stark Geld --- heute wird das Ding getätigt!&
Die Spesen fest. Planmäßig der Verlauf ---&
Berlin brust auf.&
\endnumbering
\end{minted}

(berliner abend, kurt tuchovsky)

La commande
\cs{\endstanzaextra} permet d'ajouter quelque chose à la fin de chaque strophe.
Dans l'exemple qui précède, on a ainsi mis 
\verb|\let\endstanzaextra=\bigbreak|: cela permet d'avoir 
 un espace vertical entre chaque  strophes.
 
(on peut ajouter bcp de chose: voir le manuel)
les nouvelles commande de maïeul > pour avoir le crochet

si on ne fait que des vers sans indentation> 0 partout (ex: faire des "stanza" de 10 vers .. ) S'il y a moins de vers qu'indiqué dans stanzaindents, pas grave.

\subsubsection{Aller plus loin}

Comme on peut le voir dans les exemples précédents, Ledmac numérote par défaut  une ligne sur cinq. On peut changer cela très simplement avec les commande suivantes:
\cs{firstlinenum}\marg{nbre} et \cs{linenumincrement}\marg{nbre}.
L'argument \arg{nbre} de la commande \cs{firstlinenum} indique quelle est  la première ligne qui sera numérotée ; l'argument \arg{nbre} de la commande \cs{linenumincrement} définit la fréquence avec laquelle les lignes seront numérotés. On place ces commande après \cs{beginnumbering}. 

Ainsi, si l'on veut numéroter toutes les lignes en commençant par la première, il faut mettre \cs{linenumincrement}\marg{1} et \cs{firstlinenum}\marg{1} 




-faire des "sous-numérotation
> avec \cs{firstsublinenum} 
et \cs{sublinenumincrement}

changer le numéro des lignes
\cs{setline}\marg{num} and \cs{advanceline}\marg{num} > modifier le numéro d'une ligne 
setline> la ligne d'après> + 1.
\cs{skipnumbering} > ne pas prendre en compte dans la numérotation telle ligne
-
où apparissent les numéro des ligne et comment

Un moyen pour ne pas encombrer la mémoire qd texte trop long : faire de plus petites sections numérotées. Mais si on met endnumbering ,  avec beginnumbering on recommencera la numérotation à zéro.  Utiliser à la place pausenumbering - resumenumbering > avec resume numbering la nmérotation reprend où elle s'était arrêté. Si on utilise ces commande essentiellement pour des question de mémoire conseil de faire une commande exprès: \verb|\newcommand{\memorybreak}{\pausenumbering\resumenumbering}|> plus qu'à mettre memorybreak de temps en temps dans le texte ..

\cs{lineation}\marg{arg} > \arg{page} > numérotation recommence à chaque page \arg{section} (par défut) : numérotation recommence à chaque section.
Possible aussi de changer l'endroit où apparit le numéro de page (par défaut ds la marge de gauche): \cs{linenummargin}\marg{arg}, dt les arguments peuvent être left, right, inner, ou outer

\cs{startsub} \cs{endsub} > ligne numérotés 10.1, 10.2, 10.3,.. et non 10, 11, 12 

On peut aussi changer façon dont apparaissent es numéro des lignes (voir doc)



\subsection{L'apparat critique}

Notes de bas de page se présentent de cette façon:  \cs{Afootnote}\marg{texte}. Il y en a 5 type: de A à E


>ces notes appellent un lemme : le mot du texte qui apparaîtra ds la  note de bas de page. On fait une note de cette façon: \cs{edtext}\marg{lemme}\marg{commande} (l'argument \arg{commande} contient l'appel à la note: \cs{Afootnote}, \cs{Bfootnote}, etc.
l'argument \arg{lemme} est la partie du texte annoté qui est reprise dans la note de bas de page, précédée du numéro de la ou des lignes correspondantes. 

Il est aussi possible d'utiliser  des  notes de bas de page "normales", appelées par un numéro dans le texte. Il y en a cinq sortes  possibles: de \cs{footnoteA} à \cs{footnoteB}.

Lorsque ce \enqute{lemme} est trop grand, on peut choisir quel lemme on veut voir imprimé. On utilise pour cela la commande suivante: 
\cs{edtext}\marg{le lemme trop long}\{\cs{lemma}\marg{le lemme raccourci}\cs{Bfootnote\marg{la note}}\}. 

\begin{plusloins}
Il est aussi possible de mettre des notes de fin. Là encore, cinq sortes possibles:  de \cs{Aendnote} à \cs{Eendnote}.
\end{plusloins}

Dans l'exemple suivant, j'ai utilisé deux types de notes avec lemme (dont une où j'ai mis un lemme \enquote{raccourci}),et  un type de note sans lemme:


\begin{minted}{latex}
\setstanzaindents{0,0,1,0,1,0,1,0,1}
\stanza
Quando erit illa dies, cum nostrum intrabis in \edtext{ortum}{\Afootnote{hortum, \textit{F}}},&
 Atque leges nostras ungue libente rosas?\footnoteA{In exilio}&
Et tua magna sitis mage seu \edtext{mage crescere}{\Afootnote{mage se accrescere \textit{G}}} gliscet,&
 Dum quod semper amas, carmine plenus eris.&
\edtext{Si qua istis fuerint, ut erunt, uitiosa camenis}{\lemma{Si \dots camenis}\Bfootnote{\textit{Ov. Trist. IV},1 Si qua meis fuerint, ut erunt, vitiosa libellis, parce}},&
 Parce, precor : scriptor non mihi doctus inest.&
Quaeso, tuum nobis fidum transmitte ministrum,&
 Qui tua grata mihi perferat orsa, uale.\&
\end{minted}

On obtiens alors ceci: 


\begin{minipage}{\textwidth}
\beginnumbering 
\setlinenum{224}
\firstlinenum{225}
\linenumincrement{5}
\setstanzaindents{0,0,1,0,1,0,1,0,1}
\stanza
Quando erit illa dies, cum nostrum intrabis in \edtext{ortum}{\Afootnote{hortum, \textit{F}}},&
 Atque leges nostras ungue libente rosas?\footnoteA{In exilio}&
Et tua magna sitis mage seu \edtext{mage crescere}{\Afootnote{mage se accrescere \textit{G}}} gliscet,&
 Dum quod semper amas, carmine plenus eris.&
\edtext{Si qua istis fuerint, ut erunt, uitiosa camenis}{\lemma{Si \dots camenis}\Bfootnote{\textit{Ov. Trist. IV},1 Si qua meis fuerint, ut erunt, vitiosa libellis}},&
 Parce, precor : scriptor non mihi doctus inest.&
Quaeso, tuum nobis fidum transmitte ministrum,&
 Qui tua grata mihi perferat orsa, uale.\&
\endnumbering
\end{minipage}
\bigbreak

\begin{attention}
Il faut compiler deux fois pour obtenir le numéro des lignes dans l'apparat critique.
\end{attention}



Il est conseillé de clarifier tout cela en créant ses propres commandes, par exemple en indiquant dans son préambule: 

\verb|\let\variantes=\Afootnote| \\
\verb|\let\citations=\Bfootnote| \\
\verb|\let\eclaircissements=\footnoteA| \\

Et ainsi de suite\dots 








\section{Mettre deux textes en vis-à-vis: le \package{ledpar}}
\prealable{Le package \package{ledpar}, qui sert à mettre un texte et sa traduction en vis-à-vis, fonctionne avec \package{ledmac}}
