\chapter{Édition critique et textes parallèles} \sindex[pkg]{ledpar}\sindex[pkg]{ledmac} \label{ledmac}


\begin{intro}
Nous allons aborder ici deux packages particulièrement utiles en sciences humaines : \package{ledmac} pour faire des éditions critiques et \package{ledpar} (qui ne peut être utilisé séparément de \package{ledmac}) pour éditer un texte et sa traduction en parallèle.

Ces packages contiennent moult fonctionnalités : nous n'en présentons ici que quelque unes et renvoyons aux manuels pour les autres\footcites{ledmac}{ledpar}. 
\end{intro}

\begin{attention}
Ces packages ont été améliorés durant la rédaction de ce livre. Veillez à les mettre à jour.\renvoi{maj}
Dans le gestionnaire de packages, vous ne trouverez une entrée que pour  \package{ledmac},  pas pour \package{ledpar}.  En mettant à jour \package{ledmac} vous mettez toutefois aussi à jour  \package{ledpar}.
\end{attention}

\begin{attention}
Un certain nombre de fonctionnalités de ces packages nécessitent pour fonctionner correctement une double compilation.
\end{attention}



\section{Faire une édition critique avec \packagenoidx{ledmac}}\sindex[pkg]{ledmac}

\subsection{Numéroter les lignes d'un texte}

La première fonctionnalité utile de \package{ledmac} est de numéroter les lignes d'un texte. On commence par  entourer celui-ci des commandes \csp{beginnumbering} et \csp{endnumbering}. Il faut ensuite procéder de deux façons différentes  selon que le texte est en prose ou en vers.

\subsubsection{Texte en prose}


La numérotation des lignes commence, entre ces deux commandes, avec la commande \csp{pstart} et se termine avec la commande \csp{pend}, comme dans l'exemple suivant:

\inputminted{exemples/shs/ledmac/prose.tex}

Nous obtenons ainsi : 

\begin{minipage}{0.8\textwidth}
\beginnumbering
\pstart %commencer la numérotation
Lorem ipsum dolor sit amet, consectetuer adipiscing elit ?
Morbi commodo ; ipsum sed pharetra gravida !
Nullam sit amet enim. Suspendisse id : velit vitae ligula.
Aliquam erat volutpat.

Sed quis velit. Nulla facilisi. Nulla libero. 
Quisque facilisis erat a dui. 
Nam malesuada ornare dolor.
Cras gravida, diam sit amet rhoncus ornare, 
erat  elit consectetuer erat, id egestas pede nibh eget odio.
\pend %terminer la numérotation
\endnumbering

\end{minipage}



Tout texte qui n'est pas situé entre \cs{pstart} et \cs{pend} n'est pas numéroté. Cela peut donc être utile pour interrompre temporairement la numérotation :

\begin{latexcode}
\beginnumbering
\pstart %commencer la numérotation
Lorem ipsum dolor sit amet, consectetuer adipiscing elit ?
Morbi commodo ; ipsum sed pharetra gravida !
Nullam sit amet enim. Suspendisse id : velit vitae ligula.
Aliquam erat volutpat.
\pend

Texte non numéroté

\pstart Sed quis velit. Nulla facilisi. Nulla libero. 
Quisque facilisis erat a dui. 
Nam malesuada ornare dolor.
Cras gravida, diam sit amet rhoncus ornare, 
erat  elit consectetuer erat, id egestas pede nibh eget odio.
\pend %terminer la numérotation
\endnumbering
\end{latexcode}

La numérotation recommence à zéro à chaque \cs{beginnumbering}. Il est toutefois possible d'indiquer de recommencer automatiquement la numérotation  à chaque page \emph{via} la commande \csp{lineation}\verb={page}=. \label{lineation}

\begin{attention}
Si vos sections numérotées sont très longues, il peut arriver l'erreur de compilation \forme{TeX capacity exceeded}. Il faut alors fractionner vos sections en utilisant de temps en temps la commande \csp{pausenumbering} suivie de \csp{resumenumbering} au sein de  la section numérotée\footnote{On peut aussi régler la configuration de \TeX pour lui allouer plus de mémoire.}. 

Cette séquence de commandes correspond \emph{grosso-modo} à la séquence \cs{endnumbering}\cs{beginnumbering} mais sans ré-initialisation de la numérotation.
\end{attention}

\subsubsection{Texte en vers}

La  numérotation dans les textes en vers est un peu plus complexe. Les commandes \cs{pstart} et \cs{pend} disparaissent. Il faut à la place découper son texte en strophes (\enquote{stanza}). Pour commencer la numérotation, nous insérons donc la commande \cs{stanza}. Chaque vers se termine par un \ampersand, et le dernier vers de la strophe par \cs{\ampersand}.  

Il faut indiquer l'indentation de chaque vers de la strophe --- faute de quoi on obtient le message \enquote{Missing number} --- grâce à la commande \cs{setstanzaindents}\marg{$n_0, n_1, n_2, n_x$}. L'argument \arg{$n_0$} indique l'indentation du rejet si le vers est trop long pour tenir sur une seule ligne; \arg{$n_1$} correspond au premier vers, \arg{$n_2$} au deuxième et ainsi de suite.



\inputminted{exemples/shs/ledmac/vers.tex}

donne ainsi\footcite{tucholsky}: 


\beginnumbering
\setstanzaindents{0,0,4,0,0,0,2,2,2,8}
\let\endstanzaextra\bigbreak
\stanza
Berlin setz an.&
Es speien die Geschäfte&
die wackern Knaben und die Mädchen aus.&
Jetz kommt der Feierabend --- aber defte!&
Wir springen nur noch eben rasch nach Haus.&
Die großen Fraun sind ganz auf neu gemalen.&
Wer wird heut abend wohl den Zimt bezahlen?&
Sie lächeln lieb. Das Auto summt heran.&
Berlin setzt an.\&

\stanza
Berlin brummt auf.&
Wo ich die Paare anseh:&
Hier wird ein harter Dienst straff absolviert.&
Ein Riesenrummel von Grünau bis Wannsee ---&
und alles tadellos organisiert.&
Um jeden Schnapstich fühlst du es bestätigt:&
Marie stark Geld --- heute wird das Ding getätigt!&
Die Spesen fest. Planmäßig der Verlauf ---&
Berlin braust auf.\&
\endnumbering





La commande \csp{endstanzaextra} permet d'ajouter un contenu à la fin de chaque strophe.
Dans l'exemple qui précède, l'espace vertical entre chaque  strophe est ainsi obtenu par la ligne 
\cs{let}\cs{endstanzaextra}\cs{bigbreak}. 

La commande\csp{bigbreak}  insère un espace vertical. 
 La commande \csp{let}\cs{cmd1}\cs{cmd2} est une commande \TeX qui  copie la commande \cs{cmd2} dans \cs{cmd1}.
\begin{plusloins}
Pour obtenir un crochet indiquant les rejets de vers, selon la typographie française traditionnelle, il faut redéfinir la commande \csp{hangingsymbol}.

\begin{latexcode}
\renewcommand{\hangingsymbol}{[\,}
\end{latexcode}

La commande \csp{,} sert à insérer une espace fine insécable.
\end{plusloins}

Si vos vers ne possèdent pas d'indentation, mettez \verb=0= partout dans la commande \cs{stanzaindents}, sauf pour l'indentation de rejet. S'il y a moins de vers qu'indiqué dans \cs{stanzaindents}, cela n'est pas problématique, l'inverse en revanche produit une erreur de compilation.



\subsection{Aller plus loin}

\subsubsection{Numéroter les paragraphes}

Il est possible d'insérer un nombre à chaque \cs{pstart} en utilisant la commande \csp{numberpstarttrue}\label{numberpstarttrue}, ce qui permet ainsi de numéroter les paragraphes. À chaque commande \cs{beginnumbering}, la numérotation recommence.

\begin{minipage}{0.8\textwidth}
\numberpstarttrue
\beginnumbering
\pstart 
Lorem ipsum dolor sit amet, consectetuer adipiscing elit ?
Morbi commodo ; ipsum sed pharetra gravida !
Nullam sit amet enim. Suspendisse id : velit vitae ligula. 
Aliquam erat volutpat. \pend


\pstart Sed quis velit. Nulla facilisi. Nulla libero. 
 Quisque facilisis erat a dui.
Nam malesuada ornare dolor.
Cras gravida, diam sit amet rhoncus ornare, 
erat  elit consectetuer erat, id egestas pede nibh eget odio.
\pend
\endnumbering
\end{minipage}

Pour arrêter la numérotation des \cs{pstart}s,  utilisez la commande \csp{numberpstartfalse}.

\begin{plusloins}
Si chaque \cs{pstart} correspond à un paragraphe, il est possible d'automatiser l'insertion des \cs{pstart}s et \cs{pend}s en insérant la commande \csp{autopar} après \cs{beginnumbering}.
\end{plusloins}

\begin{plusloins}
La numérotation se fait à l'aide d'un compteur \compteur{pstart}. Nous expliquons plus loin ce qu'est un compteur, comment en modifier l'apparence et la valeur.\renvoi{apparencecompteur}\renvoi{manipcompteurs} Avec ces informations vous pourrez aisément changer la numérotation des \cs{pstart}s.
\end{plusloins}

\subsubsection{Fréquence de numérotation des lignes}

Comme on peut le voir dans les exemples précédents, par défaut \cs{ledmac} numérote  une ligne sur cinq. Nous pouvons changer cela très simplement avec les commandes suivantes:

\cs{firstlinenum}\marg{nbre} et \cs{linenumincrement}\marg{nbre}.

L'argument \arg{nbre} de la commande \cs{firstlinenum} indique quelle est  la première ligne qui sera numérotée 
; celui de la commande \cs{linenumincrement} définit la fréquence avec laquelle les lignes seront numérotées. Ces commandes se placent après \cs{beginnumbering}. 

Ainsi, si nous souhaitons numéroter toutes les lignes en commençant par la première, il faut mettre :

\begin{latexcode}
\linenumincrement{1}
\firstlinenum{1}
\end{latexcode}



\subsubsection{Changer le numéro de certaines lignes}

La commande \cs{setline}\marg{num} permet de changer le numéro d'une ligne. La numérotation des lignes suivantes  continue à partir de ce nombre ; \cs{advanceline}\marg{num} permet d'ajouter \arg{num} au numéro de la ligne.

La commande \cs{skipnumbering} permet de ne pas prendre en compte la ligne courante dans la numérotation.

\subsubsection{Sous-numéro}

Il est possible d'indiquer  des lignes comme \enquote{sous-lignes}, c'est-à-dire de stopper temporairement la numérotation et de commencer une sous-numérotation.

La sous-numérotation commence avec \csp{startsub} et  finit avec \csp{endsub}.
Si la première commande se situe au milieu d'une ligne, la sous-numérotation commence à la ligne suivante.

Les commandes \cs{subfirstlinenum}\marg{nbre} et \cs{sublinenumincrement}\marg{nbre} permettent de régler la première sous-ligne numérotée et la fréquence de sous-numérotation.

\inputminted{exemples/shs/ledmac/subline.tex}

\begin{minipage}{0.8\textwidth}
\firstlinenum{1}
\linenumincrement{1}
\firstsublinenum{1}
\sublinenumincrement{1}
\beginnumbering
\pstart
Lorem\startsub ipsum dolor sit amet, 
consectetuer adipiscing elit ?
Morbi commodo ; ipsum sed pharetra gravida !
Nullam sit amet enim. Suspendisse id : velit vitae ligula. 
Aliquam erat volutpat.\endsub


Sed quis velit. Nulla facilisi. Nulla libero. 
 Quisque facilisis erat a dui. 
Nam malesuada ornare dolor.
Cras gravida, diam sit amet rhoncus ornare, 
erat  elit consectetuer erat, id egestas pede nibh eget odio.
\pend
\endnumbering
\end{minipage}

\subsubsection{Style et position des numéros des lignes}


Les numéros des lignes apparaissent  par défaut dans la marge de gauche. La commande \cs{linenummargin}\marg{arg} redéfinit cette position. L'argument peut prendre la valeur :

\begin{description}
\item[\contenuarg{left}] qui est la valeur par défaut ;
\item[\contenuarg{right}] pour avoir les numéros  à droite ;
\item[\contenuarg{inner}] pour les avoir dans la marge intérieure ;
\item[\contenuarg{outer}] pour les avoir dans la marge extérieure.
\end{description}

Il existe   cinq styles pour numéroter les lignes, sélectionnés par la commande 
\cs{linenumberstyle}\marg{style}, \cs{sublinenumberstyle}\marg{style} pour la sous-numérotation :

\begin{description} 
\item[\contenuarg{arabic}] les numéros des lignes sont en chiffres arabes, c'est le style par défaut ; 
\item[\contenuarg{Alph}] pour des lettres majuscules ;
\item[\contenuarg{alph}] pour des lettres minuscules ;
\item[\contenuarg{Roman}] pour des chiffres romains majuscules ; 
\item[\contenuarg{roman}] pour des chiffres romains minuscules. 
\end{description}

La distance \cs{linenumsep} représente la distance entre le texte et le numéro de la ligne\renvoi{unite}, par défaut, 1~pica. 
Elle est  modifiable par la commande \cs{setlength}\verb|{|\cs{linenumsep}\verb|}|\marg{distance}\renvoi{setlength}. 

De la même façon nous pouvons modifier, via \cs{renewcommand}, \cs{numlabfont}, qui indique la taille du numéro et qui est ainsi définie:
 \cs{newcommand}\verb|{|\cs{numlabfont}\verb|}{|\cs{normalfont}\cs{scriptsize}\verb|}|. 
 

\subsection{L'apparat critique}

Les notes de bas de page se présentent de cette façon:  \cs{Afootnote}\marg{texte}. Il en existe cinq sortes, de \csp{Afootnote} à \csp{Efootnote}.


Ce type de notes appelle un \enquote{lemme}, c'est-à-dire une partie du texte annoté reprise dans la note de bas de page, précédée du numéro de la ou des lignes correspondantes. 

La note de bas de page est appelée de la façon suivante:  

\csp{edtext}\marg{lemme}\marg{commande}. 

L'argument \arg{commande} contient l'appel à la note: \csp{Afootnote}, \csp{Bfootnote}, etc.

Il est aussi possible d'utiliser  des  notes de bas de page \enquote{normales}, appelées par un numéro dans le texte. Il  en existe là aussi cinq sortes: de \csp{footnoteA} à \csp{footnoteE}.

Dans les notes de type \cs{Afootnote}\dots, lorsque le \enquote{lemme} est trop grand, nous pouvons choisir de l'abréger. La syntaxe est la suivante: 

\cs{edtext}\marg{le lemme trop long}\verb|{|\cs{lemma}\marg{le lemme raccourci}\cs{Bfootnote}\marg{la note}\verb|}|. 

\begin{plusloins}
Il est aussi possible de mettre des notes de fin. Là encore, il y en a cinq sortes :  de \csp{Aendnote} à \csp{Eendnote}.

La commande \cs{marginpar} ne fonctionne pas dans les textes numérotés. À la place, le package propose les commandes \csp{ledleftnote}\marg{note} et \csp{ledrightnote}\marg{note} pour des notes de marge gauche et droite.

Il existe aussi une commande \csp{ledsidenote}\marg{note} qui insère par défaut la note dans la marge de droite, mais \csp{sidenotemargin}\verb={left}= permet de dire de l'insérer à gauche.
\end{plusloins} 

Dans l'exemple suivant, nous avons utilisé deux types de notes avec lemme, dont l'une où nous avons mis un lemme \enquote{raccourci}, et  un type de note sans lemme:


\inputminted{exemples/shs/ledmac/critique.tex}

Nous obtenons ceci\footcite{theodulf}: 



\begin{minipage}{0.8\textwidth}
\setstanzaindents{0,0,1,0,1,0,1,0,1}
\stanza
Quando erit illa dies, cum nostrum intrabis in \edtext{ortum}
{\Afootnote{hortum, \textit{F}}},&
 Atque leges nostras ungue libente rosas?\footnoteA{In exilio}&
Et tua magna sitis mage seu \edtext{mage crescere}
{\Afootnote{mage se accrescere \textit{G}}} gliscet,&
 Dum quod semper amas, carmine plenus eris.&
\edtext{Si qua istis fuerint, ut erunt, uitiosa camenis}
{\lemma{Si \dots camenis}\Bfootnote{\textit{Ov. Trist. IV},
1 Si qua meis fuerint, ut erunt, vitiosa libellis, parce}},&
 Parce, precor : scriptor non mihi doctus inest.&
Quaeso, tuum nobis fidum transmitte ministrum,&
 Qui tua grata mihi perferat orsa, uale.\&

\end{minipage} 



\begin{attention}
Il faut compiler deux fois pour obtenir le numéro des lignes dans l'apparat critique.
\end{attention}



Pour plus de facilité vous pouvez changer le nom des types de notes en indiquant par exemple dans son préambule: 
\begin{latexcode}
\let\variantes\Afootnote
\let\citations\Bfootnote
\let\eclaircissements\footnoteA
Et ainsi de suite…
\end{latexcode}

Mais il est aussi conseillé de créer des commandes personnelles pour simplifier   --- et surtout clarifier --- plus encore la rédaction d'un apparat critique; par exemple:

\begin{latexcode}
\usepackage{ifthen, xargs}
\newcommandx*{\variantes}[3][2]{%
    \edtext{#1}{%
        \ifthenelse{\equal{#2}{}}{}{\lemma{#2}}%
    \Afootnote{#3}}%
    }
\end{latexcode}

Cette commande est un peu plus complexe parce qu'elle fait appel aux packages \package{xargs} et \package{ifthen}. 

La commande \csp{newcommandx}, fournie par \package{xargs}, permet de créer une commande avec des arguments optionnels. Ici, la commande crée appelle trois arguments: le premier argument est le lemme ; le deuxième argument, optionnel --- ce qu'indique,  dans notre code, l'argument \verb=[2]= --- contient le lemme raccourci ; enfin le dernier argument contient l'annotation.

Ligne 4, la commande \csp{ifthenelse}\arg{test}\arg{sioui}\arg{sinon} du package \package{ifthen} permet d'effectuer un test : ici nous regardons si le deuxième argument est vide (\csp{equal}\verb={#2}{}=) : si ce n'est pas le cas, alors nous l'utilisons  comme argument d'une commande \cs{lemma}.

 L'utilisation de la commande que nous venons de créer peut donc se résumer ainsi : 
 
 \cs{variantes}\marg{lemme}\oarg{lemme plus court}\marg{annotation}. 
 
Ce qui est  plus simplement utilisable que :

\cs{edtext}\marg{le lemme trop long}\verb|{|\cs{lemma}\marg{le lemme raccourci}\cs{Bfootnote}\marg{la note}\verb|}|

\begin{plusloins}
Les packages \package{ifthen} et \package{xargs} sont extrêmement utiles, mais nous ne pouvons détailler ici leurs utilisation. Nous renvoyons à leurs documentations\footcites{ifthen}{xargs}.
\end{plusloins}





 






\section{Mettre deux textes en vis-à-vis: le package \packagenoidx{ledpar}}\sindex[pkg]{ledpar}


Le package \package{ledpar}, qui sert à mettre un texte et sa traduction en vis-à-vis, fonctionne avec \package{ledmac}. Tout ce qui concerne l'apparat critique et la numérotation dans \package{ledmac} est ainsi valable ici aussi. Nous considérons donc, pour éviter les redites, que vous avez lu ce qui précède.



\subsection{Principes}

Le package \package{ledmac} permet d'obtenir très facilement deux textes en vis-à-vis; c'est donc un package particulièrement utile lorsque l'on veut faire de l'édition bilingue.

Son principe de base est simple:  des textes numérotés  --- inscrits entre \cs{beginnumbering} et \cs{endnumbering}  --- sont placés dans les environnements \enviro{Leftside} pour le texte de gauche,  \enviro{Rightside} pour le texte de droite.

Le package \package{ledpar} fait correspondre en vis-à-vis chaque \enquote{boîte} d'un texte à la boîte correspondante de l'autre texte (il faut donc qu'il y ait le même nombre de boites des deux côtés). Pour la poésie,  chaque vers  (terminé par \ampersand ), est une boîte;  en prose, les boîtes sont délimitées par  \cs{pstart} et  \cs{pend}.  Il est recommandé, en prose, de mettre chaque paragraphe dans une boîte, pour obtenir une synchronisation la plus fine possible --- donc de débuter chaque paragraphe par \cs{pstart} et de le finir par \cs{pend}. 


Le package \package{ledpar} peut mettre les textes en vis-à-vis sur deux colonnes ou sur deux pages  : bien sûr dans ce dernier  cas le texte de gauche est toujours imprimé sur une page paire.
Nous utilisons dans le premier cas l'environnement \enviro{pairs} ; l'environnement \enviro{pages} dans le second.

Les commandes \cs{Columns} ou \cs{Pages}, selon qu'il s'agisse de l'environnement \enviro{pairs} ou \enviro{pages}, indiquent à \LaTeX  d'imprimer les boîtes jusqu'ici gardées en mémoire. 



Résumons!

Pour mettre deux textes en vis-à-vis sur deux pages :

\begin{latexcode}

\begin{pages}  % Des textes en vis-à-vis sur deux pages
\begin{Leftside} % Texte de la page de gauche
\beginnumbering % Début de la numérotation
\pstart % Ici, prose : donc \pstart et \pend
 
Le texte dans une langue 
 
\pend
\endnumbering  % Fin de la numérotation
\end{Leftside} % Fin du texte de gauche 
 
\begin{Rightside}  % La même chose à droite
\beginnumbering
\pstart
 
Le texte dans une autre langue
 
\pend
\endnumbering
\end{Rightside} 
\Pages

\end{pages} % Fin la composition en parallèle.
\end{latexcode}

\begin{attention}
Si l'un des textes n'est pas dans la langue principale du document, il faut préciser le changement de langue\renvoi{i18n} \emph{à l'intérieur} de l'environnement \enviro{Leftside} ou \enviro{Rightside}.

\begin{latexcode}
\begin{Leftside}
\begin{<langue>}
\end{<langue>}
\end{Leftside}
\end{latexcode}
\end{attention}

\begin{attention}
Comme avec \package{ledmac}\renvoi{lineation}, la numérotation recommence à chaque \cs{beginnumbering}. On peut toutefois faire recommencer la numérotation  à chaque page \emph{via} les commandes \cs{lineation} et \csp{lineationR}, qui ont la même syntaxe, vue plus haut. \renvoi{lineation}
\end{attention}

Nous obtenons parfois le message d'erreur \enquote{Too many \cs{pstart} without printing}. Pas de panique ! Cela signifie qu'il y a trop de boîtes en mémoire: \LaTeX  doit en imprimer quelques-unes avant de pouvoir continuer.  Il y a deux solutions possibles à ce problème.

La première solution consiste à  augmenter le nombre de boîtes que \package{ledpar} peut garder en mémoire, en mettant dans le préambule la commande \cs{maxchunks}\marg{nbre}, où \arg{nbre} est le nombre de boîtes en mémoire. Par défaut, ce nombre est de dix. Il suffit donc de l'augmenter  pour qu'il soit égal ou supérieur au nombre de boîtes de son  texte. 

Toutefois \TeX ne dispose pas d'une place illimitée pour les boîtes, c'est pourquoi il est souvent préférable d'utiliser la seconde solution.


Celle-ci, que nous pouvons d'ailleurs combiner avec la première, consiste à  couper le texte en plusieurs grands ensembles de boîtes --- c'est-à-dire en plusieurs environnements \enviro{Leftside} et \enviro{Rightside}, se terminant chacun par \cs{Columns} ou \cs{Pages}.  Cette solution est préférable pour les longs textes. Les divisions internes du texte se reflètent alors dans les environnements \enviro{Leftside} et \enviro{Rightside}. Cependant,  en mettant simplement :

\begin{latexcode}
\begin{pages}  
\begin{Leftside}  \beginnumbering 
La première partie du texte en langue originale
\endnumbering  \end{Leftside} 
 
\begin{Rightside} \beginnumbering
La traduction de la première partie
\endnumbering \end{Rightside} 
 \Pages
 
\begin{Leftside} \beginnumbering  
La deuxième partie du texte 
\endnumbering  \end{Leftside} 
 
\begin{Rightside}  \beginnumbering
La traduction de la deuxième partie
\endnumbering \end{Rightside} 
 \Pages

  \end{pages}
\end{latexcode}

nous nous retrouvons avec une numérotation qui recommence à zéro pour la deuxième partie du texte. 

La commande \cs{memorydump} permet de résoudre ce problème. Elle équivaut en fait à un \cs{pausenumbering} immédiatement suivit d'un \cs{resumenumbering}. Il faut donc mettre: 

\begin{latexcode}
\begin{pages}  
\begin{Leftside}  \beginnumbering 
La première partie du texte en langue originale
\end{Leftside} % pas de \endnumbering ici!...
 
\begin{Rightside} \beginnumbering
La traduction de la première partie
\end{Rightside} % et là non plus...
\Pages
 
\begin{Leftside} \memorydump % et, ici, pas de \beginnumbering..  
La deuxième partie du texte 
\endnumbering  \end{Leftside}
    
\begin{Rightside}  \memorydump %...car la numérotation continue!
La traduction de la deuxième partie
\endnumbering \end{Rightside}   % Ne pas oublier de mettre 
% tout de même \endnumbering quand on arrive à la fin ..
 
 \Pages

\end{pages}
\end{latexcode}

Ainsi, la numérotation du texte se fait en continue entre la première partie et la deuxième partie.


\begin{plusloins}
Il est bien sûr possible de mettre en parallèle un texte en vers et sa traduction en prose. 

Par ailleurs, si vous utilisez la commande \cs{numberpstarttrue}\renvoi{numberpstarttrue} pour numéroter les \cs{pstart}s, sachez que la numérotation gauche et la numérotation droite fonctionnent avec deux compteurs différents : \compteur{pstartL} et \compteur{pstartR}.\renvoi{apparencecompteur}\renvoi{manipcompteurs}
\end{plusloins}




\subsection{Affiner la présentation}


\subsubsection{Vers de longueurs inégales}

Si des vers sur des pages parallèles prennent un nombre de lignes inégales, par exemple si le vers de gauche contient un rejet et non  celui de droite, alors \package{ledpar} introduit un blanc après le vers le droite, pour que le vers suivant soit exactement au même niveau dans la page de droite et dans la page de gauche.

Il est possible de dire à \package{ledpar} de ne pas introduire ce blanc, mais de produire un léger décalage des vers, décalage rattrapé à chaque changement de pages. Pour ce faire, il faut passer l'option \option{shiftedverses} au chargement du package.

\begin{latexcode}
\usepackage[shiftedverses]{ledpar}
\end{latexcode}
 
 \subsubsection{Tailles et séparateurs des colonnes}

Par défaut, les colonnes font 45\% d'une largeur normale de texte. Nous pouvons modifier ce réglage en redéfinissant les longueurs \csp{Lcolwidth} et \csp{Rcolwidth} respectivement pour la colonne gauche et droite. Nous utilisons pour cela la commande \cs{setlength}\renvoi{setlength}\renvoi{longueurrelative}. Par exemple pour avoir 47\% de la largeur normale d'un texte, il faut mettre:

\begin{latexcode}
\setlength{\Lcolwidth}{0.47\textwidth}
\setlength{\Rcolwidth}{0.47\textwidth}
\end{latexcode}

De même nous pouvons définir un filet de séparation grâce à la longueur \csp{columnrulewidth}\renvoi{unite}.

\begin{latexcode}
\setlength{columnrulewidth}{0.4pt}
\end{latexcode}


\subsubsection{Numérotation des lignes de droite}

Par défaut, les numéros des lignes de droite sont suivies d'un \forme{R}, correspondant au terme \cs{Right}. Pour obtenir à la place un \forme{D}, il suffit de redéfinir la commande \csp{Rlineflag} :

\begin{latexcode}
\renewcommand{\Rlineflag}{D}
\end{latexcode}

Si nous ne souhaitons  rien, mettre tout simplement :


\begin{latexcode}
\renewcommand{\Rlineflag}{}
\end{latexcode}




